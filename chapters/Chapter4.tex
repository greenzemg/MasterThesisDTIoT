% Chapter 4
% ======================================================================================================
% NOTES, TODOS
% limitation 
% Observation -> blockchain for data integrity and sharing Machine learning for intrusion detection cloud -> deploying dt in cloud 
% Privacy preserving 
% DT is a simulation platform 
% ======================================================================================================
\chapter{Discussion, Limitation and Conclusion}
\label{Chapter4} % For referencing the chapter elsewhere, use \ref{Chapter1} 

In this literature review part of an ongoing project we conducted a systematic way of reviewing literature  on the use of DT technology in various domains to enhance security requirements. The study was conducted using Kitchenham and Charter's three-phase approach for systematic literature review. The aim was to investigate how DT is used to enhance Iot applications more specifically in industry 4.0. In addition, we explore the literature on what security scheme or mechanism is used to protect the integrity and confidentiality of data flow between DT and (I)IoT sensor devices.
% at giving more indications for wise design  choices, a summary of the pros and cons of the various categories of algorithms, a vision of the future evolution of these algorithms

A total of 523 articles were identified from online digital databases, such as ScienceDirect, SpringerLink, Scopus, IEEExplore, ACM, and Web of Science. After applying inclusion and exclusion criteria and removing duplicates, we left with 74 research papers from which we selected 56 papers where we perform analysis to answer our research question.

\section{Observation and Findings}
As a result of a thorough review of the literature on the use of DT technology for securing (I)IoT applications and securing digital communication between DT and IoT devices, few findings have been identified.

\begin{itemize}
    \item \textbf{\textit{DT as integrated platform}}: We  identified DT as an integrated platform of a virtual model with enabling technology to process collected data from the operating environment through (I)IoT sensors in order to gain insight for monitoring, optimization and prediction. Real-time and uncorrupted data is main requirement of DT for its proper functioning. 


    \item\textbf{\textit{Core enabling technology}}: Machine learning and data analytics are the core technology used to power up Digital Twin to function as a security-enhancing tool. Detection and protection security service are realized mainly using machine learning and data analytics that operate on large data collected through sensors. 
    
    \item\textbf{\textit{Lack of discussion on Lightweight encryption}}: In our review of selected papers, it became evident that most of the papers give little emphasis on ensuring the authenticity and integrity of the sensor data that is fed into the Digital Twin station. Even though a handful of papers do discuss securing the data transmission channel, their recommendations rely on traditional encryption and authentication mechanisms such as RSA and AES. This is concerning because in most use cases data generating sensors are power constraints where it is not feasible to deploy traditional encryption algorithms to secure them. Hence, It is essential to give focus on lightweight algorithms to protect data integrity and authenticity of data used in Digital Twin-based solutions.
    
\end{itemize}












\section{Future Direction}

\textbf{\textit{Efficient lightweight encryption algorithms}}: As the development of Digital Twin technology progresses, it is expected that it will become accurate in replicating physical objects and processes. To achieve this level of accuracy, a large number of tiny, resource-constrained IoT sensors will need to be deployed on a massive scale to measure every aspect of the physical status being replicated. This presents future research directions for designing and implementing  efficient encryption algorithms that can be deployed on resource constraint devices.


\textbf{\textit{Machine learning}}: AI is expected to play a crucial role in the future development of DT technology. Specifically, there are two potential areas where AI can be utilized: for performing analytics on collected data and to create AI-enabled simulations. Hence, future research could explore the ways in which machine learning technology is used in conjunction with DT models. This could involve conducting a systematic literature review to better understand how machine learning has been integrated within DT technology in previous research studies.

\textbf{\textit{Remote access control for DT}}: One area of research we have identified as a gap in the literature is the secure remote access control to the virtual counterpart of an ICS component for vendors to perform troubleshooting and testing. In the traditional real-world industry setup, vendors of ICS components have remote access control to the physical object of the industry for various reasons. However, it is not clear how this is going to be handled on the DT yet. One potential direction for research is to explore and investigate how secure remote access can be achieved to one or more components of the DT.

% \textbf{\textit{DT based ransomware detection}}
\textbf{\textit{Human computer interaction}}: Finally, future research could explore the human-computer interaction (HCI) aspect of DT technology. This could involve examining the ways in which users interact with DT models and exploring new and innovative ways to improve the user experience. By improving the HCI aspect of DT technology, it may be possible to enhance the accuracy and reliability of the models by ensuring human error is minimized.

\section{Limitations Of The Study}

One limitation of this study is that only papers with "[Dd]igital [Tt]win[s]?" in their title were selected for review. While the authors argue that Digital Twin-focused research will likely use the term in the title, it is not always the case. Additionally, only one person was involved in the paper selection process, which may have resulted in some subjectivity.

\section{Conclusion}
Overall, the finding of this systematic literature review suggests that DT technology is evolving to become crucial technology, particularly in smart industries, such as the power grid, automotive industry, water treatment plants, transportation systems, and satellite internet, for providing real-time cybersecurity insights and for use as an emulation environment for threat detection and response, vulnerability assessment, security awareness training, and threat intelligence without disrupting actual industry operations. 

Machine learning and data analytics are the two primary technologies widely used by study authors to enable digital twin security features. By analyzing large amounts of data generated by digital twins, machine learning algorithms can detect anomalies and identify potential security threats. Data analytics can also be used to identify vulnerabilities in digital twin systems and to optimize security configurations.  
- While digital twin technology provides many advantages, it also poses several security challenges. Even worse, most authors neglect the security concerns related to the data used by digital twins during transmission and while at rest. Offcourse it is challenging to secure the data used by digital twins, especially in device-constrained environments. Encryption methods such as AES and RSA can provide a level of security for digital twin data. However, these methods may not be feasible for deployment in devices with limited processing power and memory.  

% To address the security related issues of DT deployment, some researchers have proposed using blockchain technology to maintain the integrity and reliability of data when it is shared by a network of digital twins. Others suggested to use the tradition cryptographic mechanisims like RSA and AES to secure a data communication channel between data source the DT station hub. 
