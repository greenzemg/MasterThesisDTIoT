% Chapter 6
% ======================================================================================================
% NOTES, TODOS

% ======================================================================================================

\chapter{Conclusion}
\label{Chapter6} %

This final chapter presents the conclusion and takeaways of the research. The first part of the chapter provides the conclusion of the research effort of the systematic literature review, which explored the potential of Digital Twin (DT) as a security tool in Industry 4.0. The second part of the chapter summarizes the contribution and findings from the implementation of the proposed solution to fill the research gap identified in the first part. The last two paragraphs of the chapter present the takeaways of the research.



In the first part of this work, we approach the first research question (\hyperref[lbl:rq1]{RQ1}) in \hyperref[Chapter2]{Chapter 2} by exploring the security application of Digital Twin in Industry 4.0 through a systematic literature review of 67 papers retrieved from six (6) academic databases. The result of our review, presented in section \ref{sec:rq1dtsec}, indicates that Digital Twin is rapidly being adopted in various Industry 4.0 sectors including critical infrastructures such as \textit{Power Grid}, \textit{Automotive Industry}, \textit{Intelligent Transport System}, \textit{Water Treatment}, to provide security services such as intrusion detection, penetration testing, cyber-range, and so on. In addition, our result shows that \textit{Machine learning}, \textit{Blockchain (Smart Contract)}, \textit{Cloud and Edge}and \textit{5g Networks} are the major integrated tool used to equip Digital Twin with various functionality beyond mere modelling. 

On the other hand, in the quest of exploring the second research question (\hyperref[lbl:rq1]{RQ2}), the result of the literature review, presented in section \ref{sec:rq2secmech}, revealed a lack of discussion and recommendations on efficient security mechanisms that meet the speed and security requirement of Digital Twin application in Industry 4.0. Moreover, the literature review did not identify any security solutions that took into consideration the limited power, storage, and computation of (I)IoT devices, which play a key role in sending environmental data and receiving commands. While the majority of papers that raise the security issue in the digital communication between (I)IoT and Digital Twin propose Blockchain to provide integrity with the aim of addressing the privacy of data. The remaining papers mention recommending traditional encryption algorithms such as AES, SHA-256 and RSA and security mechanisms that require expensive hardware setup and sophisticated technology like Trusted Execution Environment and Quantum Communication. However, these solutions are computationally demanding and infeasible to run on resource-constrained sensor devices practically.


To answer the third research question (\hyperref[lbl:rq1]{RQ3}) with the aim of addressing the previously mentioned and discussed gap in section \ref{sec:gap}, we proposed a communication scheme presented in section \ref{sec:prosolution} and implemented in \hyperref[Chapter3]{Chapter 3}. It is based on lightweight Authenticated Encryption with Associated Data (AEAD) family of algorithms known as ASCON that ensure confidentiality, integrity, and authenticity in one operation of encryption and decryption. In this regard, our contribution is to design a resource-efficient communication scheme based on the idea of lightweight encryption algorithms and payload encryption techniques. To show the practicality of our proposed solution in securing communication between a hardware device on the field and the Digital Twin hosted on the cloud, a proof of concept demonstrated in section \ref{sec:sendingauth} by sending an authenticated and encrypted payload over MQTT protocol.

In response to the last research question (\hyperref[lbl:rq1]{RQ4}), a performance analysis of our proposed solution comparing two implementations, one based on ASCON (lightweight) and the second based on AES (AES-GCM) is conducted in \hyperref[Chapter4]{Chapter 4}. The result revealed that lightweight encryption algorithms (ASCON) are efficient in terms of speed, memory, and power consumption with an equal level of security as the heavyweight-based algorithms (AES-GCM) implementation. 


To summarize, in recent years, Digital Twin has been widely adopted as a security tool in Industry 4.0, including critical infrastructure. This is because Digital Twin is addressing one of the biggest challenges in security testing which is performing tests without disrupting operations. By creating a virtual representation of the physical system, Digital Twin allows security analysts to test and analyze the system without having to interact with the real system. This can be a significant advantage in critical infrastructure systems, where even a short disruption can have serious consequences.

On the other hand, lightweight encryption algorithms are suitable for resource-constrained (I) IoT devices including sensors, actuators, and RFID that are used in Industry 4.0. These algorithms offer better performance compared to traditional encryption algorithms, while still providing adequate security. However, it is important to note that other components within device applications, such as Wi-Fi and MQTT application codes and so on, also contribute significantly to resource consumption. As a result, it is important to carefully consider the overall system design when selecting encryption algorithms for (I)IoT devices.
