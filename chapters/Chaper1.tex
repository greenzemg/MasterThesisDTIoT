% Chapter 1

\chapter{Introduction} % Main chapter title

\label{Chapter1} % For referencing the chapter elsewhere, use \ref{Chapter1} 

%----------------------------------------------------------------------------------------
% Define some commands to keep the formatting separated from the content 
\newcommand{\keyword}[1]{\textbf{#1}}
\newcommand{\tabhead}[1]{\textbf{#1}}
\newcommand{\code}[1]{\texttt{#1}}
\newcommand{\file}[1]{\texttt{\bfseries#1}}
\newcommand{\option}[1]{\texttt{\itshape#1}}

%----------------------------------------------------------------------------------------
% 
% \hajj{Chapter One introduces the topic of the thesis to the reader. The critical part of writing Chapter One is to establish the statement of the problem and research questions. Basically, you are justifying to the reader why it is nec- essary to study this topic and what research question(s) your study will answer. Usually, the topic is based around a particular problem area that you want to focus on. However, before you introduce the reader to the specific topic and problem, you have to first provide the reader with the broader context (the general problem) and consequences related to the topic. In other words, before you discuss the specific problem, you need to contextualize your topic within the larger problem. For example, you would first discuss the problems related to the topic.
% Chapter One of the thesis includes a section on the Statement of the Problem (information about the specific problem), Background and Need (the background literature related to the problem), the Purpose of the Study (the focus and goal of the study), Research Questions (what questions the study proposes to answer), and other significant sections. In this chapter, you need to support all of your claims and positions using citations
%  check this reference please:
%  https://www.wcupa.edu/business-publicManagement/geographyPlanning/documents/thesisGuideline.pdf}
% \begin{enumerate}
    %item state the general topic and give some background
    %\item provide a review of the literature related to the topic
        % \item define the terms and scope of the topic
        % \item outline the current situation
        % evaluate the current situation (advantages/ disadvantages) and identify the gap
    % \item identify the importance of the proposed research
    % \item state the research problem/ questions
    % \item state the research aims and/or research objectives
    % \item state the hypotheses. DONE
    % \item outline the order of information in the thesis. DONE
    % \item outline the methodology. DONE
% \end{enumerate}


% General Research topic 
\textbf{Research Topic}:
% Data exchange and connectivity 
% Sensors and actuators 
% Industrial process is becoming intelligent through digital twin
% Machines are communicating with each other to get data for processing 
% 
% We are in an era where it is difficult for industries and organisations to operate without connectivity and data exchange. With the advent of Industry 4.0, Digital Twins(DT) and Internet of Things(IoT) are the two enabling technologies that provide unprecedented levels of connectivity and data exchange to monitor and optimise business operation. Both technologies have been around for a while, but the integration of of them is just getting started.
Connectivity and data exchange are two of the characteristics of Industry 4.0. Industrial Internet of Things ((I)IoT) sensors are a vital component of this paradigm, facilitating the collection and transmission of environmental data from the physical system to the central station for processing and analysis(Digital Twin). However, although sensors play a critical role in this process, they are not inherently equipped to run strong encryption mechanisms to secure the data they transmit over wired or wireless channels. This research aims to explore the security challenges posed by power and storage limitations of (I)IoT sensors in the application of Digital Twin within Industry 4.0 use cases. 



% \textbf{Literature Review}:
% % Methodology Summary 
% % The core concept of Digital Twin
% % How digital twin is used to securing IoT application 
% % How data is secure 
% This study conducted a systematic literature review to investigate the use of Digital Twin in enhancing the security of IoT applications in industry 4.0. The review followed a three-phase approach, including the use of automated tools for the review process. A total of 523 papers were initially collected from seven digital libraries, and after applying inclusion and exclusion criteria and quality assessment, 56 relevant papers were selected for review. 

% The definition of digital twins in the literature varies depending on the context, but generally, it consists of three components: physical and digital states, interconnectivity, and a process for collecting and examining data.

% The use cases for digital twins are vast, including threat detection and response, vulnerability assessment, security awareness training, and threat intelligence. A wide range of industry 4.0 sectors benefit from this technology, including the power grid, automotive industry, water treatment plants, transportation systems, and satellite internet, among others. Researchers have deployed security-enabled digital twins to improve the security and safety of operations.  
% Machine learning and data analytics are the two primary technologies widely used by study authors to enable digital twin security features. Other technologies, such as blockchain, cloud computing, and federated learning, are also used. However, most authors neglect the security concerns related to the data used by digital twins during transmission and while at rest.  
% Some researchers have proposed encryption methods, such as AES and RSA, to secure digital twin data. However, these methods may not be feasible for deployment in device-constrained devices. Other researchers have suggested using blockchain to maintain the integrity and reliability of data shared by a network of digital twins.  

\textbf{Niche and Scope}:
% Industry 4.0 smart manufacturing, 
% IIoT application 
% Digital Twin use case
% 
The (Industrial) Internet of Things is a system of interconnected physical devices, such as sensors and actuators, that are embedded in technology to allow the communication and exchange of data through wireless networks \cite{maillet-contozEndtoendSecurityValidation2020}. The adoption of the Industrial Internet of Things (IIoT) in industrial settings aims to improve efficiency, productivity, and safety by gathering and analyzing data from connected devices, sensors, and machines \cite{kumarBlockchainDeepLearning2022}. However, the gap between enterprise Information Technology (IT) and Operational Technology (OT) has been a challenge in implementing (I)IoT due to different standards, protocols, and security requirements \cite{adrienbacueDigitalTwinsEnhanced2022}.

Digital Twin, on the other hand, is software-defined digital representation of the physical object that operates on a large set of environmental sensor data to simulate and monitor operation in real-time \cite{williamdanilczykANGELIntelligentDigital2019, danilczykSmartGridAnomaly2021}. Simulation of an operation, visualization of product in real-time, troubleshooting remote equipment \cite{alcarazDigitalTwinComprehensive2022, veledarDigitalTwinsDependability2019, vargheseDigitalTwinbasedIntrusion2022}, and managing assets in industry are a few of the use cases among the others. 

Together, those two technologies are transforming the way we manage resources and operations in various industry use cases.


% The importance of this research
% The challenge in the storytelling scenario 
\textbf{Importance of the Research}:
Digital Twin and (I)IoT technologies have recently emerged to play important roles across various industries, including aerospace engineering, electric grid, car manufacturing, petroleum industry, healthcare, and more \cite{tao_digital_2019}. For these technologies to function properly in an industrial setting, authentic and un corrupted data \cite{fuller_digital_2020} in real-time is required \cite{yuchenziqianzhangningtangApplicationDigitalTwin2022}. 

However, in most cases, the data transmitted over wired or wireless from the source ((I)IoT) to the central processing station (DT) is vulnerable to a variety of cyber threats, such as eavesdropping, tampering, and replay attacks \cite{hussainiTaxonomySecurityDefense2022}. These attacks can result in the unauthorized access, modification, or theft of sensitive data, potentially causing significant harm to organizations that rely on these technologies.

Therefore, it is critical to secure the data channels used for interaction and communication between DT  and (I)IoT technologies to ensure the integrity and authenticity of data. Authentication is a fundamental security principle that can address this concern by allowing only authorized endpoints to access and consume data. In the context of DT and (I)IoT, authentication can help ensure that only known (I)IoT devices can send sensor data to the DT hub and only known Digital Twins can send commands to remote devices. In addition to authentication, other security measures, such as encryption and data integrity checks, can be employed to further strengthen the security of DT and (I)IoT applications.

Further, in industry 4.0 use cases, (I)IoT devices, such as sensors and actuators, are typically located in remote areas that are difficult to access. As a result, it is also crucial to implement an efficient and reliable cryptographic solution that can increase the lifespan of these power-constraint devices and avoid frequent repairs or replacements.     


% Problem statement 
\textbf{Problem Statement}:
As noted previously, manufacturing facilities, including critical infrastructure, are using an (I)IoT devices to collect and send sensor measurements and operating conditions to the Digital Twin station to monitor and optimize the overall operation of the business process. However, due to the storage and processing constraints that (I)IoT devices have \cite{williams_survey_2022,noauthor_lightweight_nodate}, it is challenging to adopt traditional cryptographic security mechanisms to ensure the confidentiality and integrity of the data flow between (I)IoT and DT. If proper security measures (authentication, authorization, encryption, and so on) are not used, an attacker may be able to perform a man-in-the-middle attack with the intent of intercepting sensitive sensor data or disrupting a system by injecting crafted faulty data \cite{salimBlockchainEnabledSecureDigital2022}. In recent literature and blog of standard institutes, it is mentioned to utilize security schemes that can fit into constraint devices to address the security requirement.  

% Proposed Solution and Hypothesis 
\textbf{Proposed Solution}:
The National Institute of Standards and Technology (NIST) has developed a set of standards for lightweight cryptography that are designed to provide cryptographic security with limited storage and computing resources, such as those found in small devices like sensors or smart cards.

In this study, we propose an authentication/encryption scheme based of one of the NIST standard lightweight cryptography algorithms. According to NIST \cite{noauthor_lightweight_nodate}, the algorithms have a small code size and low power consumption, while still providing a high level of security.  This NIST-standardised lightweight-based authentication scheme will enhance the confidentiality and integrity of data exchanged via the communication channel between the DT and its counter-physical component.


\textbf{Research Objective and Aims}:
The aim of this research is two-fold. First, conduct a systematic literature review to analyze the concept of Digital Twins and how they can enhance the security of IIoT/IoT applications in industry 4.0 use cases. Secondly, to implement a NIST standard lightweight authentication scheme to improve the security of digital communication between power, storage, and processing constraint devices and DT stations hosted on the cloud or local premises.


\textbf{Hypothesis}:
% More security
% efficiency increases speed of data flow 
% Getting a real-time of data is one requirement of the DT application 
Implementing a lightweight encryption/authentication scheme based on the NIST standard can enhance the security and integrity of data flow between DT and power constraint (I)IoT sensors. This solution can reduce memory footprint, power consumption, and latency. 

The proposed solution in this paper-implementing ligthweight based authentication- can facilitate the secure exchange of data between DT and IIoT sensors while preventing unauthorized access to sensitive information. The scheme utilizes symmetric key cryptography, which is computationally efficient and suitable for power-constrained devices while supporting mutual authentication to verify the identities of both the DT and the IIoT sensors before exchanging data.




% It supports mutual authentication to verify the identities of both the DT and the IIoT sensors before exchanging data, thus preventing unauthorized access to sensitive information.



% Methodology 
\textbf{Methodology}:
This study has two major parts; a literature review to identify a gap and an implementation of the proposed solution to fill the gap. For the literature review, we conducted a systematic way of reviewing previous literature with the aim of analyzing how Digital Twin is used to securing IoT applications within Industry 4.0 use cases. 

To implement our proposed solution, which is a lightweight authentication/encryption scheme, we used a Digital Twin open-source platform called Ditto for the software part, for the hardware part we used one of ESP32 family microcontrollers from Espressif systems.As we mentioned before, Ditto is an open-source software design pattern, developed and maintained by Eclipse Foundation to facilitate the integration of Digital Twin and IoT devices\cite{noauthor_eclipse_nodate}. On the other hand, ESP32 popular microcontrollerchip used in IoT applications. It has low power consumption, WiFi and Bluetooth connectivity, and a wide range of peripheral interfaces.  

After we implemented the proposed solution, the performance and efficiency of the proposed solution is compared with the traditional  method of providing encryption and authentication mechanisms in terms of power consumption, execution time, and storage complexity(memory footprint). 
% The following diagram provides the general setting of our proposed solution.

% \begin{figure}[H]
%     \label{fig:ps-scheme}
%     \caption{General Setting of Proposed Solution}
%     \centering
%     \includesvg[width=\textwidth]{images/svg/ps-scheme-final.svg}
    
% \end{figure}

% outline the order of information in the thesis 
\textbf{Report outline}:
The remainder of this study report is organized as follows. In the following section, we present the methodology we employed for the systematic literature review process. Then we provide details results of the literature review. Following that, we discuss the summary of the result and limitations of the study. Finally, we draw our conclusion on the basis of the evidence we gathered.   

