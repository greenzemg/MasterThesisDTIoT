% ==============================================================================
% NOTES, TODOS, Outline
% Outline 
%   background and context
%  Propose statement 


% A. Background and Context
% Problem Statement/Research Question
% Significance and Relevance of the Research
% Overview of the Existing Literature
% Importance and sifnificance of the research
% =============================================================================
% New outline 
% Start discussing Digital Twin. How it originated, how is being used in Industry 4.0
% Then talk about the communication channel, the data it needs should be in real-time secure 
% Then talk about IoT and IIOT. How they can be used to send data to Digital Twin? though they are resource constrained. 
% Then talk about lightweight solutions like ASCON. They provide enough security but they are efficient 
% Also it is good to talk about MQTT protocol and payload techniques 

Industry 4.0 is the fourth industrial revolution, which is characterized by the use of cyber-physical systems, the Internet of Things(IoT),  cloud computing, and big data analytics to automate, monitor, and optimize manufacturing processes \cite{abikoye_application_2021}. This has led to a significant increase in the connectivity of industrial systems through a number of attached actuators/sensors and smart devices, which has also made them more vulnerable to cyberattacks \cite{sousaELEGANTSecurityCritical2021}.

In addition, due to the complexity and tightly coupled legacy systems in Industry 4.0 of critical infrastructure, it is very challenging to perform vulnerability assessment or penetration tests. It is so challenging that even a very simple \textit{nmap} scan can result in downtime or disruption of the whole system. These systems are often critical to the functioning of society where a few seconds of downtime or disruption can have a major impact.


Luckily, we can address the aforementioned challenge using a technology known as Digital Twin (DT). A Digital Twin is a virtual representation of a physical system or process that can be used to simulate its behaviour \cite{eckhartEnhancingCyberSituational2019}. This can be used to perform vulnerability assessment and penetration testing in a safe and controlled environment \cite{sousaELEGANTSecurityCritical2021}. By simulating attacks on the Digital Twin, it is possible to identify and fix vulnerabilities without causing disruption or downtime. Digital Twin can also be used to improve the security of Industry 4.0 systems in other ways. For example, it can be used to monitor IoT-based healthcare systems \cite{pirbhulalNovelFrameworkReinforcing2022} and detect anomalies that could indicate a cyberattack \cite{saadImplementationIoTBasedDigital2020, chukkapalliCyberPhysicalSystemSecurity2021}. Additionally, they can be used to control access to systems and data \cite{glenandbensonjamesandguptamaanakandsandhuravicatheyEdgeCentricSecure2021}, and to enforce security policies \cite{giovannipaolosellittoEnablingZeroTrust2021}.

One of the integral parts of Digital Twin is (Industrial) Internet of Things ((I)IoT). (I)IoT refers to a network of interconnected physical devices, such as sensors and actuators, that are embedded in technology and equipped with interconnectivity that enables them to communicate and exchange data either via wired or wireless channel \cite{maillet-contozEndtoendSecurityValidation2020}. In this regard, the communication channel between the Digital Twin and (I)IoT requires a critical security consideration. This channel not only needs to be secure in order to prevent attackers from tampering with or disrupting the data exchange but also needs to be efficient to support the real-time requirements of the Industry 4.0 system \cite{fuller_digital_2020} and resource constraint of (I)IoT devices. 


However, integrating (I)IoT devices into the Industry 4.0 use case has a few challenges. One of the challenges is due to the fact that they are inherently limited with power and computation resources to secure them using strong traditional encryption algorithms \cite{williams_survey_2022, noauthor_lightweight_nodate}. Ensuring the confidentiality and integrity of data flow between Digital Twin and resource-constrained (I)IoT is an important aspect that should be taken into consideration before deploying and integrating (I)IoT with Digital Twin. If, for example, security measures such as authentication, authorization, encryption, and integrity checks are not in place, attackers can exploit vulnerabilities to perform man-in-the-middle attacks. These attacks might involve intercepting sensitive sensor data or injecting malicious data to disrupt system operations \cite{salimBlockchainEnabledSecureDigital2022}. Therefore, it is important to use a lightweight cryptographic solution like ASCON to secure the communication channel without sacrificing efficiency. 



% The integration of (I)IoT in Industry 4.0 aims to enhance efficiency, productivity, and safety by collecting and analyzing data from connected devices like sensors and actuators \cite{kumarBlockchainDeepLearning2022}. 

% 2) Due to the disparity between enterprise Information Technology (IT) and Operational Technology (OT) that arises from differing standards, protocols, and security requirements \cite{adrienbacueDigitalTwinsEnhanced2022}. In this context, IT refers to general-purpose technologies utilized for data processing, information management, and computing tasks. On the other hand, OT systems are specifically designed to control and oversee industrial processes within industrial environments \cite{dietzUnleashingDigitalTwin2020}.


% ------------------------------------------------------


% IT is end user centric usually connected to the internet, where as OT is an isolated network of industry assets. 

% Digital Twin (DT), on the other hand, is a digital representation of a physical object that leverages environmental sensor data to simulate and monitor real-time operations \cite{williamdanilczykANGELIntelligentDigital2019, danilczykSmartGridAnomaly2021}. It has various applications, including simulating operations, visualizing products in real-time, remotely troubleshooting equipment \cite{alcarazDigitalTwinComprehensive2022, veledarDigitalTwinsDependability2019, vargheseDigitalTwinbasedIntrusion2022}, and managing industrial assets. These technologies, (I)IoT and Digital Twin, are revolutionizing resource and operational management across multiple industries.






In this regard, the NIST (National Institute of Standards and Technology) is playing a vital role in reviewing and standardizing cryptography algorithms for resource-constrained devices. They have been conducting a competition for decades for secure and lightweight algorithms. They recommend employing security schemes and encryption algorithms specifically designed for constrained devices to address these security requirements of low-power devices \cite{noauthor_nist_2023}.

Digital Twin and (I)IoT technologies are being integrated and utilized across various industries, including aerospace engineering, power grid, automobile manufacturing, oil and gas industry, healthcare, and more \cite{tao_digital_2019}. In this context, while (I)IoT devices are used to collect environmental sensor measurements and send them,  Digital Twin is used to process and analyze data for insight and intelligence. This research explores how Digital Twin is used to secure applications in Industry 4.0 and it also proposes a lightweight solution for secure communication between constrained devices and digital twins.

% This enables the monitoring and optimization of overall business operations. Real-time, authentic, and uncorrupted data is required to fully leverage these technologies and prevent undesired behaviour \cite{fuller_digital_2020, yuchenziqianzhangningtangApplicationDigitalTwin2022}.


\section{Motivation}
The motivation behind this research stems from the widespread adoption of Digital Twin technology in Industry 4.0 \cite{atalay_digital_2022} and it's heavily dependent on (I)IoT devices (sensors and actuators). These (I)IoT devices are often resource-constrained devices which are not equipped with the same security features as more powerful devices. This makes them more vulnerable to attack.

% During the systematic literature review, we identify a challenge related to securing communication between Digital Twin and resource-constrained (I)IoT devices. This communication channel carries sensitive data that should be protected and authenticated through an efficient lightweight encryption communication scheme. 

As these technologies--Digital Twin and (I)IoT--become increasingly integrated into critical infrastructure, it becomes crucial to secure the data channels used for interaction and communication between them using a lightweight encryption algorithm. The less resource-demanding nature of lightweight cryptography solutions can also increase the lifespan of sensor power as it uses less power. For example, in Industry 4.0 applications like oil refineries, (I)IoT devices (sensors and actuators ) are often deployed in remote and inaccessible areas. If less power-consuming solutions are implemented, the need for frequent repairs and replacement can be reduced or avoided \cite{williams_survey_2022, noauthor_lightweight_nodate}.


Considering the pressing nature of these issues and their potential impact on Industry 4.0 applications, our research attempts to address the challenges associated with securing the communication channel between Digital Twin and constrained (I)IoT devices. By developing a lightweight communication scheme based on the latest NIST standard lightweight encryption algorithm, we aim to bridge the gap and provide a practical solution that ensures secure and efficient data transfer in resource-constrained (I)IoT environments. To show the efficacy of the proposed solution, we implemented the solution and evaluated its performance in a real-world setting. The results of our evaluation show that our solution is effective in securing the communication between Digital TWin and (I)IoT devices, while also being lightweight and robust.

Through this research, we seek to contribute to the advancement of secure communication practices within the integration of Digital Twin and (I)IoT technologies for enhanced operation in the Industry 4.0.

% In the context of Digital Twin (DT) and (I)IoT, using authenticated encryption the integrity and confidentiality of the data communicated between can be assured. In other words, these security measures help safeguard the confidentiality of data and verify its integrity throughout the communication process \cite{salimBlockchainEnabledSecureDigital2022}.


\section{Methodology}
This research has two aims:-
\begin{itemize}
    \item \textit{First}, to conduct a systematic literature review and explore how Digital Twin is used to enhance security in Industry 4.0 use cases. In addition, we investigate the security methods discussed in previous studies to secure the communication between DT and (I)IoT devices.

    \item \textit{Second}, to implement a resource-efficient security mechanism for constrained (I)IoT end devices to ensure the confidentiality, integrity and authenticity of data communicated with Digital Twin ( or any cloud-based solution ). 
\end{itemize}
% This research has two major parts: a systematic literature review of Digital Twin security services in Industry 4.0 and the technical implementation of proposed solutions to secure the communication between Digital Twin and (I)IoT devices.

For the literature review, we adopted a three-phase approach following the guidelines outlined by Kitchenham and Charter \cite{kitchenham_guidelines_2007}. This systematic process involved planning the review protocol, executing the review itself, and thoroughly reporting the results. To ensure comprehensiveness, we collected a total of 67 papers from six reputable digital archives. To facilitate the literature review process and enhance information retrieval efficiency, we employed two valuable tools: Parsifal, an online tool specifically designed for automating systematic literature reviews, and Logseq, a note-taking application known for its ability to connect ideas and retrieve stored information effectively.

Based on the research gaps we identified from the systematic literature review, we proposed a lightweight resource-efficient communication scheme. In order to validate the effectiveness of our proposed communication scheme, we conducted a comparative analysis between two different implementations: one based on ASCON lightweight encryption and the other using AES-based encryption. The evaluation primarily focused on three resource-related critical metrics: power consumption, execution time, and storage complexity (memory footprint). This analysis allowed us to gain insight into the advantages and benefits of our proposed solution in terms of resource utilization and system performance. By leveraging these metrics, we could ascertain the feasibility of our approach and its potential practical application in real-world Industry 4.0 scenarios.



\section{Research Questions}
Both, the systematic literature review and the implementation of the proposed solution presented in this research have answered four research questions. While the first two research questions are answered through the literature review, the last two research questions are answered through technical implementation. The research questions of the paper are listed as follows: 
\begin{itemize}
    \item \textbf{RQ1: How Digital Twin is used to enhance the security of Industry 4.0 applications?} \label{lbl:rq1} This research question aims to identify in what way Digital Twin is used to provide security services such as intrusion detection, vulnerability assessment and so on to enhance the security aspect of the Industry 4.0 process.
    \item \textbf{RQ2: What are the security mechanisms (ways) presented in the literature to ensure the confidentiality, integrity, and authenticity of data (message ) communicated between Digital Twin and its mapped physical devices?} This research question focuses on the identification of cryptographic or any other security solutions that are used to improve the security of digital channels for data communication between Digital Twin and (I)IoT devices. 
    \item \textbf{RQ3: How to ensure the security requirement of a communication channel between a Digital Twin and resource-constrained (I)IoT device?} This research question aims to provide an answer to the challenge of implementing security solutions for resource-constrained devices such as sensors and actuators that send and receive data to and from Digital Twin. 
    \item \textbf{RQ4: How efficient is the proposed solution?} This research question aims to provide insight into the performance difference between the lightweight-based (ASCON - recently standardized lightweight encryption) implementation of the proposed solution and the traditional-based (AES - existing and conventional) implementation. 
\end{itemize}

Research question 1 (\textbf{RQ1}) and 2 (\textbf{RQ2}) are answered in chapter \ref{Chapter2}. The answer for research questions 3 (\textbf{RQ3}) and 4 is provided in chapter \ref{Chapter3} and chapter \ref{Chapter4} respectively. 

\section{Hypothesis}
\label{sec:hypo}
We hypothesize that there exists a research gap in how to secure the communication between the Digital Twin and resource-constrained devices using lightweight solutions. This is because lightweight encryption algorithms have only recently emerged and are still being standardized. As a result, we expect limited research on how to use them to secure communication between Digital Twins and resource-constrained devices. 


\section{Contribution}
This research has four contributions which can be summarized as follows:
\begin{itemize}
    \item \textit{Exploration of Digital Twin's Role in Securing Industry 4.0 Business Process:} Through a systematic literature review, this research digit into the application of Digital Twin in enhancing the security of Industry 4.0 processes. Besides, by investigating existing security mechanisms discussed to secure data communicated in the Digital Twin application, a research gap is presented.   
    \item \textit{Development of a Lightweight Communication Scheme for Resource-Constrained (I)IoT Devices}: In response to the research gap identified, this paper presents a new and efficient communication scheme specifically tailored for resource-constrained (I)IoT devices. By leveraging the latest NIST standard lightweight encryption algorithm, the proposed solution ensures secure communication between Digital Twin and resource-limited (I)IoT devices. 
    \item \textit{Practical Implementation and Validation of the Proposed Solution:} The research also provide practical implementation of the proposed communication scheme in a real-world application through various hardware and software configurations. 
    \item \textit{Performance Analysis of the Proposed Solution:} Finally, an assessment and evaluation of three (no encryption, AES-based, and ASCON-based) implementation of the proposed solution are conducted through performance measurement using metrics, including power consumption, latency, and memory usage.
\end{itemize}

\section{Outline}
This paper is organized into 6 chapters outlined as follows. 
\subsection{Chapter 2: Literature Review}
Chapter \ref{Chapter2} presents a systematic literature review on Digital Twin in the context of Industry 4.0. We discuss the methodology employed for the literature review, including search criteria, data sources, and the selection process. The chapter includes an analysis of the findings, identifying existing gaps in the research, and exploring existing security mechanisms discussed in the literature to secure the digital communication between Digital twin and (I)IoT.

\subsection{Chapter 3: Proposed Solution}
In Chapter \ref{Chapter3}, we delve into the details of our proposed solution for addressing the research gaps we identified in Chapter \ref{Chapter2}. This chapter also discusses the background of the key components of the proposed solution and their relevance to the study. Additionally, we provide implementation details, showcasing how the proposed solution can be practically applied.

\subsection{Chapter 4: Performance Evaluation}
Chapter \ref{Chapter4} focuses on the performance evaluation of three different implementations of the proposed solution. We present a detailed analysis of the results obtained from these implementations and provide a comprehensive comparison. This chapter provides insight into the speed, size and power consumption of each implementation.

\subsection{Chapter 5: Discussion and Analysis}
In Chapter \ref{Chapter5}, we discuss the implications of the proposed solution based on the findings from the literature review and the performance evaluation. Moreover, this chapter addresses any limitations encountered during the research process with future research direction.


\subsection{Chapter 6: Conclusion}
In the final chapter (\ref{Chapter6}), we draw conclusions based on the result achieved from the literature review and the implementation of the proposed solution. We summarize the key contributions of this study and discuss its implications along with future research and practical applications.



