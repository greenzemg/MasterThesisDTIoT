\section{Problem Statement}
% Version one 
% Traditional security measures are no longer effective due to the low power nature of many IoT sensor nodes\cite{williams_survey_2022}, and problems like communications taking place without sufficient encryption or authentication are being reported in the literature. The large-scale data collecting that is made possible by these devices is another frightening problem. If proper security measures (authentication, authorization, encryption, and so on) are not used, an attacker may be able to perform a man in the middle attack with the intent of intercepting sensitive sensor data or disrupting a system by injecting his own fault data\cite{nguyen_survey_2021}.
% The innovative idea of Digital Twins (DTs) has been adopted by research and industry more frequently since 2017 in a variety of disciplines, including industrial production, the process industry, building management, the health care sector, and smart cities. 


% Version 2
Manufacturing facilities, including critical infrastructure,  are using IoT sensors to collect and send sensor measurement and operating conditions to Digital Twin station to monitor and optimize the overall operation of the business process \.  However, due to storage and processing constraint IoT sensors have[ \cite{williams_survey_2022}, \cite{noauthor_lightweight_nodate}], it is challenging to adopt traditional security cryptographic mechanisms and ensure the confidentiality and integrity of data flow between IoT and DT. If proper security measures (authentication, authorization, encryption, and so on) are not used, an attacker may be able to perform a man-in-the-middle attack with the intent of intercepting sensitive sensor data or disrupting a system by injecting his crafted faulty data. In previous literatures and standard bodys, it is mentioned to utilize security schemes that can fit into constraint devices to address the security requirement. Hence, in this paper, we propose a lightweight mutual authentication scheme to ensure the security of IoT applications in Digital Twin.