% ======================================================================================================
% NOTES, TODOS
% ======================================================================================================

\subsection{Defining PICOC}
PICOC stands for Population, Intervention, Comparison, Output and Context. It is a widely used technique in medical and social science studies to define the focus of research \cite{carrera-rivera_how-conduct_2022}. However, Kitchenham and Charter in \cite{carrera-rivera_how-conduct_2022} and Carrera in \cite{kitchenham_guidelines_2007} showed that this technique can also be applied to computer science related research to formulate and structure research questions. 

In this subsection, we define our PICOC criteria for this systematic literature review as follows:-

\textit{Population:} The motivation to conduct this research is the security-related problem we identified in the communication between Digital Twin and constrained  (I)IoT devices deployed in the smart industry to collect sensor data. Hence, the problem domain or "Population" for this research is (I)IoT devices used with Digital Twin to enhance security in Industry 4.0. Industries that use Digital Twin and (I)IoT devices, such as smart cities, smart homes, smart grids, smart health, smart manufacturing, etc. In this sense, the "Population" part of PICOC in this review refers to the following terms: Digital Twin, (Industrial)Internet of Things, Industry 4.0, Smart Manufacturing, Cyber-physical Systems, and Critical Infrastructure. 

\textit{Intervention:} Our intervention to address the aforementioned problem - the security issue of digital communication between Digital Twin and (I)IoT - is to implement a lightweight NIST standard cryptographic authenticated encryption algorithm for power, storage and computation constraint (I)IoT devices. In this regard, we use the term "authentication" as an intervention.

\textit{Comparison:} Before designing and implementing an intervention for a specific problem, it is important to identify the existing solution in the literature. The results of reviewing, comparing and analysing the existing solution discussed in the relevant research literature can be used as input to design and implement the intervention methodology. With this regard, this study will identify and compare authentication schemes or security mechanisms used in securing a data flow between Digital Twin and (I)IoT. 

\textit{Outcome:} This study has two broad categories of outcomes. The expected outcome of the literature review is to provide insight into the potential benefits of Digital Twin in securing operations in Industry 4.0. From a technical implementation perspective, the expected outcome is ensuring data integrity and a secure communication channel with an efficient and performant cryptographic authentication and encryption scheme for constrained devices.

\textit{Context:} This systematic literature review is focused on the Industry 4.0 environment, targeting Digital Twin solutions deployed in smart industries to enhance security. On the other hand, the second part of this study is focused on implementing authenticated encryption algorithms in constrained physical devices and measure the performance of lightweight and traditional algorithms-based implementation in terms of power consumption, execution time and memory usage. 