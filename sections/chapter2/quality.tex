% ======================================================================================================
% NOTES, TODOS
% ======================================================================================================
% describe how to perform the detail assessment
% use numeral scale 
% Are the aim of the article clearly stated?
% Is the implementation detail explained adequately 
% does the study has direct link to research question 1 and/or question 2. 
% 
% \subsection{Quality Assessment Checklist}
% After selecting papers using the inclusion and exclusion criteria, we also evaluate the papers using a quality assessment checklist. This evaluation takes place during the review phase, with the goal of identifying and eliminating articles that fail to meet the checklist criteria. In other words, a paper is removed if it does not satisfy any of the criteria on the checklist.


% The quality assessment checklist that defines the detailed assessment criteria is outlined below.
% % Besides the inclusion and exclusion criteria, it is important to evaluate the quality of the research study\cite{kitchenham_guidelines_2007}.
% \begin{itemize}
%     \item \textbf{QA1:} Abstract: Is the research question (the aim) clearly stated?
%     % \item \textbf{QA2:} Abstract: Does the study propose any new security solutions or improvements to existing ones? ensuring the security (I)IoT application using Digital Twin?
%     % \item \textbf{QA1:} 
%     \item \textbf{QA3:} Methodology: Does the study adequately explain a methodology used in the paper?
%     \item \textbf{QA4:} Methodology: Does the study conduct an experiment (test bed) or use case to validate the hypothesis?
%     % \item \textbf{QA5:}Methodology: Does the study provide detailed procedures to ensure secure communication between DT and IoT?
%     % \item \textbf{QA5:}Result: Are the results of the study clearly presented and supported by the data?
%     \item \textbf{QA5:} Discussion: Does the research provide a discussion and analysis of the implications and potential future work in the field?
% \end{itemize}
% The quality assessment questions outlined above were evaluated using a numerical metric scoring system, with a range of 1-5, where 1 represents an irrelevant article and 5 represents a highly relevant and qualified article. The level of agreement in answering the quality checklist questions was determined through a categorical classification system, comprising of "Agree," "Somewhat Agree," "Neutral," "Somewhat Disagree," and "Disagree." These categories were assigned corresponding weight values, with "Agree" being assigned the highest weight value of 5 and "Disagree" being assigned the lowest weight value of 1. This scoring system helped to ensure that we evaluated research studies in a fair and consistent way so that the studies that were the most relevant and of the highest quality were selected for the review.

% The number of articles and their scores is depicted in the diagram [blab bla]. 
