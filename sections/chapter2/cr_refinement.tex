% ======================================================================================================
% NOTES, TODOS
% ======================================================================================================
\subsection{Study Selection and Refinement}

% After the screening of 452 papers using the inclusion/exclusion criteria, we were left with 83 papers. However, during the review phase, we excluded 16 more papers and we ended with 67 papers for further review and analysis. 

We conducted a systematic literature review to identify relevant studies on the topic of Digital Twin security and Industry 4.0. The initial search of electronic databases yielded 727 papers. We then applied the inclusion and exclusion criteria listed in Table 2.2, which resulted in 452 papers. We manually screened the titles and abstracts of these papers to identify 83 papers that met our criteria. We then conducted a full-text review of the 83 papers and excluded 16 papers that were not relevant to our research question. The final set of 67 papers was included in our analysis.



We had to exclude several papers during our review process for the following reasons:
\begin{itemize}
   \item \textbf{Irrelevant to (I)IoT and Industry 4.0}: Some papers were not relevant to securing applications related to \textbf{(I)IoT} in an \textbf{Industry 4.0} context. For instance, we came across a study that used a \textbf{Digital Twin} to secure a data center, which did not fit within our scope.

    \item \textbf{Duplicate Content}: We identified instances where the same study was submitted to different journals with different metadata, yet contained nearly identical content. Unfortunately, the tools we employed couldn't always catch these duplicates.
    
    \item \textbf{Non-Conference Source}: We excluded studies that were sourced from book chapters rather than conference proceedings, as they didn't align with our criteria.
    
    \item \textbf{Lack of Relevance to Research Questions}: Some studies simply weren't relevant to any of the research questions we were addressing.
    
    \item \textbf{Unrelated to Industry Use Case}: Studies that focused on securing \textbf{(I)IoT} devices without any connection to an industry use case were also among the excluded papers.
    
\end{itemize}

As a result of this refinement and selection process, the final set of 67 papers was identified for in-depth data extraction and analysis. In the subsequent section, we present a review of these papers, with a focus on addressing two research questions:\begin{itemize}
    \item[-] How is Digital Twin used to enhance security in Industry 4.0?
    \item[-] And what security mechanisms are used to secure the communication channel between Digital Twin and (I)IoT? 
\end{itemize} 

 




