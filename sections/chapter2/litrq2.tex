% ======================================================================================================
% NOTES, TODOS
%
% ======================================================================================================
%
% In\cite{xuGametheoreticApproachSecure2020} Xu et al. describe a system with two channels for communication between sensors and other components. The first channel is used for communication between sensors and a Digital Twin and is protected using cryptography like Message Authentication Code (MAC) or Digital Signature (DS). This channel is considered secure because cryptography helps protect the integrity of the data flow between the sensors and the digital twin. The second channel, used for communication between sensors and a physical estimator, does not use cryptography because it would negatively impact the performance of the physical system. 


% In Olivares-Rojas et al.'s recent study \cite{olivares-rojasCybersecuritySmartGrid2022}, a Digital Twin framework was proposed to avoid sniffing attacks and ensure secure communication between physical and virtual objects. To achieve this, the authors utilized RSA signatures, which provide robust encryption that is difficult to compromise.

% \subsubsection{Securing DT and IoT Communication}
\subsection{RQ2: (I)IoT-DT Security: Literature's Security Mechanisms}
\label{sec:rq2secmech}
This subsection provides an answer to the second research question of this paper which is to identify the security mechanism presented in the literature to ensure secure data communication between (I)IoT and Digital Twin.

To ensure the reliability and security of Digital Twin based systems, it is essential to have secure communication between the physical and digital components. The computational, power and storage limitation of those physical components ((I)IoT) has to be taken into consideration. In this regard, we analyzed 14 papers that discuss data confidentiality, integrity, and privacy in the Digital Twin ecosystem. Table \ref{tbl:security-mechanism} provides a summary of  the security mechanisms employed in the literature. 

The reviewed studies cover topics such as access control systems, cryptography, authentication protocols, privacy protection mechanisms, quantum networking, and blockchain-based data sharing. Our aim is to provide an overview of the current state of research concerning securing communication in cyber-physical systems based on Digital Twin and (I)IoT components.

Gehrmann et al.\cite{gehrmannDigitalTwinBased2020} discuss the implementation of a single central access control system that is based on policies defined using standard frameworks such as XACML and tokens like SAML and OAuth. These policies help regulate who has access to what information and ensure the security of the communication.  


To solve the security problems such as communication trust and privacy protection, the authors in\cite{xuEfficientAuthenticationVehicular2021} propose a secured vehicular digital twin communication framework that utilizes anonymous authentication. To achieve this, the authors present a concrete authentication protocol based on a secret-handshake scheme and group signature, which solves the issues of unforgeability and conditional traceability. The proposed framework provides secure communication between iTwins(DT) and their physical lords, as well as between iTwins(DT) themselves, ensuring the privacy and security of the information transmitted. The proposed protocol has been validated and found to meet basic security requirements while having low computation costs.  

Jingyi Wu et al.\cite{wuDeepLearningDriven2022} presents a method that focuses on the privacy and confidentiality of data used for training detection models in drones of cyber-physical systems. The authors use differential privacy techniques to improve the accuracy and efficiency of the analysis of drone data while ensuring the protection of sensitive information. 

Kumar et al.\cite{kumarBlockchainDeepLearning2022} suggest a blockchain-based data transmission scheme that employs a Proof-of-Authentication (PoA) mechanism, which is implemented through the use of smart contracts. This helps to validate the legitimacy and integrity of data collected from Internet of Things (IIoT) nodes, improving communication security and data privacy within a decentralized IIoT network.  

In \cite{salimBlockchainEnabledSecureDigital2022} Salim's work involves securing the communication between IoT devices and Digital Twins using a private blockchain, smart contracts, and deep learning for network traffic monitoring. The private blockchain and smart contracts help ensure the data flow between physical devices and DTs is secure and tamper-proof. The deep learning model helps detect early signs of botnet behaviour and alerts the security vendor to take action to isolate infected devices, maintaining the security of the communication and the integrity of the data.  


A study conducted by Zhigan Lv et al.\cite{lvDigitalTwinsBased2022} aims to enhance the communication security between industrial Internet of Things devices (IIoT) and Digital Twins (DTs) by using quantum communication technologies. The authors introduce a channel encryption scheme based on quantum communication using entanglement states and quantum teleportation. Further, they propose an Adaptive Key Residue algorithm based on a quantum key distribution mechanism. The goal is to improve the security of communication between IIoT devices and DTs.


Lai et al.\cite{chengzhelaiSPDTSecurePrivacyPreserving2022} present a scheme for secure and privacy-preserving traffic control data sharing using digital twins. The scheme incorporates a group signature with time-bound keys for data source authentication and efficient member revocation during the data uploading phase, ensuring secure data storage on the cloud service provider. Moreover, the scheme includes an attribute-based access control technique for flexible and efficient data sharing during the data sharing stage. The primary objective of this scheme is to guarantee effective and secure data sharing for traffic control purposes.

in\cite{debenedictisAdoptionSecureCyber2022} De Benedictis addresses the security and trustworthiness of the communication between the digital twin and physical device through various technologies and HW and SW solutions such as Trusted Execution Environment platforms and Physically Unclonable Functions (PUFs) for device authentication. In addition, blockchain technology, which provides secure, immutable, and auditable data storage for the exchanged critical data is investigated by authors. 


The authors in this paper \cite{chenDigitalTwinBasedHeuristic2023a} proposed a secure smart manufacturing framework through the integration of Digital Twin (DT) and Blockchain technologies. The framework aims to facilitate efficient and secure multi-party collaborative information processing in heterogeneous IIoT environments. Notably, the paper demonstrates that the proposed authentication mode outperforms the standard protocol in terms of time efficiency. Although the paper does not provide detailed information on other methods employed in the framework, it highlights simulation results. In conclusion, the authors suggest the future inclusion of quantum computing technology to further enhance the overall efficiency and security of the proposed framework.


Zhen et al. \cite{zhengBlockchainBasedTrustworthy2022a} proposed a data security sharing architecture based on a dual Blockchain network to solve the security problems of the Internet of Things. The first blockchain, which is called, authorization Blockchain is used for permission control and consensus, and the other which is called storage Blockchain is used for the storage of data body. The proposed architecture is applied to the Internet of Things system based on Digital Twin to solve the data security transmission between the physical system, digital twin system and IoT application system. However, the authors in this study provide only data authentication. They assume the data from IoT devices is encrypted on transmission.  

In \cite{danilczykBlockchainChecksumEstablishing2021a} this paper, a novel use of the lightweight SHA-256 hash algorithm is proposed to create a blockchain of sensor readings, ensuring trustworthy communication between the control center and remote sensors. By chaining the checksums of current and previous readings, the implementation establishes trust based on the unbroken linked list length. The authors in this paper claim that this approach can strengthens the security and trustworthiness of sensor data in digital twin applications, particularly in high-value domains such as the power grid.

Lie et.al \cite{liuBlockchainBasedSecureCommunication2022a} proposed BC-Based IoV Secure Communication Framework presents an architecture designed to enhance secure communication in the context of the Internet of Vehicles (IoV). By leveraging blockchain technology, the authors claim the framework securely stores vital data such as public keys and communication history. It consists of five key modules: BC network, access control, secure transmission protocol, vehicle Ad Hoc, and a Sybil attack detection mechanism. To combat the rising prevalence of Sybil attacks in IoV scenarios, the framework utilizes regular location certificates issued by base stations, which serve to validate vehicle location accuracy. This proposed framework offers a viable solution to enhance communication security in IoV environments.

In \cite{pervezSIGNEDSmartCIty2023a} this paper the authors introduce a framework called SIGNED, which aims to enable secure and verifiable exchange of digital twin data in a smart city context. The framework focuses on data ownership, selective disclosure, and verifiability principles using Verifiable Credentials. It consists of five functional components: Cyber \& Physical Layer, Workflow Designer, Analysis Layer, Traceability Layer, and Digital Wallet. The Traceability Layer, integrated with a blockchain-based Verifiable Data Registry, maintains the public credentials and tracks registered assets. The authors present a proof of concept using a smart water management use case to demonstrate the effectiveness of SIGNED in ensuring trusted and verifiable data exchange, with minimal performance impact. Overall, the framework provides enhanced security and privacy when sharing data between different functional units in a smart city.

A contribution by Feng et al. \cite{fengSensibleSecureIoT2021a} presented work to enhance IoT communication security in digital twin networking. They proposed an interference source location scheme with a mobile tracker to reduce attacks, improve resistance, and enhance Attribute-Based Encryption (ABE). They use access control policy and symmetric encryption to secure key exchange. To address observation noise through an unscented Kalman filter, the paper modifies interference source location. The authors in this work conclude that utilizing Jamming Signal Strength (JSS) information with the untracked Kalman filter algorithm can effectively estimate the interference source location and other related state information.



\begin{table}[H]
\footnotesize
% \centering
\caption{\label{tbl:security-mechanism} Security mechanism of securing the data in DT and (I)IoT communication.}
% \resizebox{\linewidth}{!}{
\begin{NiceTabular}{|p{1.0cm}|p{5.5cm}|p{6.0cm}|}
\CodeBefore
% \rowcolors[gray]{2}{0.8}{}[cols=1-2,restart]
\Body
\toprule

\textbf{Ref}  & \textbf{Security mechanism(s)}& \textbf{Goal(s)}  \\
    \midrule

    \cite{gehrmannDigitalTwinBased2020} & Central access control system based on OAuth and XACML &  Secure access control\\
    \hline
    \cite{xuEfficientAuthenticationVehicular2021} & Anonymous communication based on secret-handshake scheme and group signature & Unforgeability and conditional
traceability (privacy)\\
    \hline
    \cite{wuDeepLearningDriven2022} & Differential privacy techniques & Privacy and confidentiality of data \\
    \hline
    \cite{kumarBlockchainDeepLearning2022} & Blockchain and Smart contract based Proof-of-Authentication(PoA) & Validate the legitimacy and integrity of data collected from (I)IoT nodes. \\
    \hline
    \cite{salimBlockchainEnabledSecureDigital2022} & Blockchain, Smart contract and Deep learning & Integrity of data, detect botnet behaviour\\
    \hline
    \cite{lvDigitalTwinsBased2022} & Quantum communication technologies & Improve overall security of communication between DT and IIoT \\
    \hline
    \cite{debenedictisAdoptionSecureCyber2022} & Trusted Execution Environment and Unclonable Functions(PUFs)& Security and Trustworthiness of communication \\
    \hline
    \cite{chengzhelaiSPDTSecurePrivacyPreserving2022} & Attribute-based Access Control  & Secure data storage \\
    \hline
    \cite{chenDigitalTwinBasedHeuristic2023a} & Blockchain  & To authenticate data generated from cluster before they are used in DT \\
    \hline
    \cite{zhengBlockchainBasedTrustworthy2022a} & Authorization Blockchain and Storage Blockchain & Secure data sharing through authorization \\
    \hline
    \cite{danilczykBlockchainChecksumEstablishing2021a} & Blockchain and SHA-256 hash for chained checksum & To increase the security and trustworthiness of sensor reading for Digital Twin application.\\
    \hline
    \cite{liuBlockchainBasedSecureCommunication2022a} & Blockchain, access control secure transmission protocol & Improve the communication security of Interent of Vehicles(IoV).\\
    \hline
    \cite{pervezSIGNEDSmartCIty2023a} & framework based on verifiable data register(VDR) and credentials  & Secure and protect privacy of data exchange in Digital Twin ecosystem.\\
    \hline
    
     \cite{fengSensibleSecureIoT2021a} & Attribute-Based Encryption (ABE) and Symmetric encryption scheme & To ensure the secure communication of Digital Twin and IoT.\\

    
\bottomrule
\end{NiceTabular}
% }
\end{table}