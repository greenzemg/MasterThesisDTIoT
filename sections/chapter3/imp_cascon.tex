

\subsection{Implementation of ASCON and AES-GCM for device}





The implementation of both algorithms on the hardware IoT device was carried out using C and C++ programming languages within Arduino for esp-idf embedded development framework. We opted for C and C++ because those two choices are more suitable for low-level programming such as for embedded resource constraint devices. The main application for the IoT device was developed in C++, while the algorithm for ASCON and AES-GCM was implemented in C and incorporated through the use of the "extern" macro in C++ main application.


The counterpart of the algorithms in the digital twin was implemented in Java. This is because the connectivity microservice of Ditto is implemented in Java. This allowed us to extend the connectivity module using Java to incorporate an extension for encryption and decryption of the payload that comes from IoT devices.

It is worth noting that we neither altered nor introduced optimizations in the design of these algorithms. For both algorithms (ASCON\footnote{https://github.com/ascon/ascon\_collection} and AES-GCM\footnote{https://github.com/usnistgov/Lightweight-Cryptography-Benchmarking/tree/main/implementations/\_reference\_/crypto\_aead/aes-gcm/mbedtls}), we selected the optimized reference implementations tailored for the ESP32 device chip. However, to enhance the security of our implementation, we incorporated a function to generate a nonce. This aspect is crucial in addressing vulnerabilities such as replay attacks, which involve the repeated use of encrypted information.
