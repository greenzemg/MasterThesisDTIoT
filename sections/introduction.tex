\section{Introduction}
The emergence of Digital Twins(DT) and the Internet of Things(IoT) has opened up a new opportunity for businesses to leverage technology to gain insights and optimize performance. To maintain the security and privacy of data, these technologies comes, however, with their own unique set of security challenges that must be addressed. In this paper, we will explore the current state of security, particularly the authentication scheme used to ensure the confidentiality and integrity of data flow between the virtual model, which is the DT, and the physical devices, which could be sensors or actuators. We will also provide lightweight DT based authentication scheme along with how to implement for IoT application. Finally, we will conclude our work with a recommendation on how to protect a DT and IoT network with efficient and performant cryptographic authentication schemes. 

% ----- Archive
% Over the years data has become crucial part for proper functioning of  business process. Due to the growing adoption of IoT in daily life, it is being created on a massive scale. Maintaining the confidentiality and integrity of data that is sent across communication channels is critical for the security and safety of any system. Recent years have seen an increase in interest among researchers in applying Digital Twin (DT) to challenges in a range of disciplines, including cyber security . DT are simulations of real-world physical systems and operations