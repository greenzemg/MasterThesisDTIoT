% ======================================================================================================
% NOTES, TODOS
% ======================================================================================================

\subsection{Research Question}
Before embarking on the process of identifying studies and extracting data, it is crucial to identify and clearly define research questions or objectives, as they serve as guiding principles for conducting a literature review\cite{carrera-rivera_how-conduct_2022}. Therfore, in this systematic literature review we have established two research questions as follow:

\begin{itemize}

    % RQ1
    \item \textbf{RQ1: How is Digital Twin used to enhance the security of (I)Iot applications in the industry 4.0 use cases ?} - 
    this research question aims to identify the potential benefit of DT in improving the security and safety of smart industries that using IoT devices, including sensors and actuators.
    % \begin{itemize}
    %     % I need a comment from Mohammed on this -> with regad to use case -> is to braod and vogue?
    %     \item \textbf{RQ1.1: What is the concrete concept of Digital Twin} - 
    %     under subcategory of the above research question, the concept of digital and its use cases are explored.
    % \end{itemize}

    % RQ 1
    % \item RQ1. What mututal authentication schemes for DT and IoT application are discussed in the literature? 
    % \item How can we use Digital Twin to enhance security issue in IoT/IIoT application? 
    %  Replace schemes by  mechansims.
    \item \textbf{RQ2: What are the security methods presented in the literature to ensure the authentication between Digital Twin and its mapped physical devices?} - 
    this question focus on the identifying  authentication/encryption mechanisms used to ensure the secure communication betwen DT and (I)IoT devices.
\end{itemize}