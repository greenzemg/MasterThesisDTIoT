% ======================================================================================================
% NOTES, TODOS
% ======================================================================================================

\subsection{Research Question}
Research question or objectives are the guiding lines for reviewing a literature. Therefore, it is important to identify and define a clear research question prior to starting study identification and data extraction\cite{carrera-rivera_how-conduct_2022}. In this sense, this systematic review of the literature seeks to provide research answers for the following questions.
\begin{itemize}

    % RQ1
    \item \textbf{RQ1: How is Digital Twin used to enhance the security of IoT/IIot applications in the industry 4.0 use cases ?} - 
    this question aims to identify how digital twin is used to improve the security of industries -- that use IoT devices including sensors and actuators -- to achieve OT(operational technology) security goals:Safety,Reliability, and Availability.It is worth not that safety related to ensuring the security of industry components.
    \begin{itemize}
        % I need a comment from Mohammed on this -> with regad to use case -> is to braod and vogue?
        \item \textbf{RQ1.1: What is the concrete concept of Digital Twin} - 
        under subcategory of the above research question, the concept of digital and its use cases are explored.
    \end{itemize}

    % RQ 1
    % \item RQ1. What mututal authentication schemes for DT and IoT application are discussed in the literature? 
    % \item How can we use Digital Twin to enhance security issue in IoT/IIoT application? 
    %  Replace schemes by  mechansims.
    \item \textbf{RQ2: What are the security schemes presented in the literature to ensure the authentication between Digital Twin and its mapped physical devices?} - 
    this question focuses on the identification of cryptographic authentication schemes that are used to improve the security of digital twin and IoT communication.
    
\end{itemize}