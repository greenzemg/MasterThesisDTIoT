% ======================================================================================================
% NOTES, TODOS
% ======================================================================================================

\subsection{Research Question}
Before beginning the process of study identification and data extraction, it is crucial to identify and clearly define a research question or objectives, as they serve as the guiding principles for conducting a literature review\cite{carrera-rivera_how-conduct_2022}. This systematic literature review aims to address the following research questions:

\begin{itemize}

    % RQ1
    \item \textbf{RQ1: How is Digital Twin used to enhance the security of IoT/IIot applications in the industry 4.0 use cases ?} - 
    this question aims to identify how Digital Twin is used to improving the security of industries that use IoT devices including sensors and actuators to achieve OT (operational technology) security goals: Safety, reliability and availability.
    % \begin{itemize}
    %     % I need a comment from Mohammed on this -> with regad to use case -> is to braod and vogue?
    %     \item \textbf{RQ1.1: What is the concrete concept of Digital Twin} - 
    %     under subcategory of the above research question, the concept of digital and its use cases are explored.
    % \end{itemize}

    % RQ 1
    % \item RQ1. What mututal authentication schemes for DT and IoT application are discussed in the literature? 
    % \item How can we use Digital Twin to enhance security issue in IoT/IIoT application? 
    %  Replace schemes by  mechansims.
    \item \textbf{RQ2: What are the security schemes presented in the literature to ensure the authentication between Digital Twin and its mapped physical devices?} - 
    this question focuses on the identification of authentication mechanisms that are used to ensure the security of DT and (I)IoT communication.
\end{itemize}