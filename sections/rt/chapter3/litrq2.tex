% ======================================================================================================
% NOTES, TODOS
%
% ======================================================================================================
%
\section{RQ2: What are the security schemes presented in the literature }

To ensure the reliability and security of Digital Twin based systems, it is essential to have secure communication between the physical and digital components. In this section, we will examine the various methods proposed in recent studies to secure the communication between Digital Twin and its physical counterpart. The studies reviewed will cover topics such as access control systems, cryptography, authentication protocols, privacy protection mechanisms, quantum networking, and blockchain-based data transmission with the aim to provide an overview of the current state of research in securing communication in cyber-physical systems based on Digital Twin and IIoT components.

Gehrmann et al.\cite{gehrmannDigitalTwinBased2020} discuss the implementation of a single central access control system that is based on policies defined using standard frameworks such as XACML and tokens like SAML and OAuth. These policies help regulate who has access to what information and ensure the security of the communication.  

In\cite{xuGametheoreticApproachSecure2020} Xu et al. describe a system with two channels for communication between sensors and other components. The first channel is used for communication between sensors and a Digital Twin and is protected using cryptography like Message Authentication Code (MAC) or Digital Signature (DS). This channel is considered secure because cryptography helps protect the integrity of the data flow between the sensors and the digital twin. The second channel, used for communication between sensors and a physical estimator, does not use cryptography because it would negatively impact the performance of the physical system.  

To solve the security problems such as communication trust and privacy protection, the authors in\cite{xuEfficientAuthenticationVehicular2021} propose a secured vehicular digital twin communication framework that utilizes anonymous authentication. To achieve this, the authors present a concrete authentication protocol based on a secret-handshake scheme and group signature, which solves the issues of unforgeability and conditional traceability. The proposed framework provides secure communication between iTwins(DT) and their physical lords, as well as between iTwins(DT) themselves, ensuring the privacy and security of the information transmitted. The proposed protocol has been validated and found to meet basic security requirements while having low computation cost.  

Jingyi Wu et al.\cite{wuDeepLearningDriven2022}presents a method that focuses on the privacy and confidentiality of data used for training detection models in drones of cyber-physical systems. The authors use differential privacy techniques to improve the accuracy and efficiency of the analysis of drone data while ensuring the protection of sensitive information. 

Kumar et al.\cite{kumarBlockchainDeepLearning2022} suggest a blockchain-based data transmission scheme that employs a Proof-of-Authentication (PoA) mechanism, which is implemented through the use of smart contracts. This helps to validate the legitimacy and integrity of data collected from Internet of Things (IIoT) nodes, improving communication security and data privacy within a decentralized IIoT network.  

In \cite{salimBlockchainEnabledSecureDigital2022} Salim's work involves securing the communication between IoT devices and Digital Twins using a private blockchain, smart contracts, and deep learning for network traffic monitoring. The private blockchain and smart contracts help ensure the data flow between physical devices and DTs is secure and tamper-proof. The deep learning model helps detect early signs of botnet behavior and alerts the security vendor to take action to isolate infected devices, maintaining the security of the communication and the integrity of the data.  


A study conducted by Zhigan Lv et al.\cite{lvDigitalTwinsBased2022} aims to enhance the communication security between industrial Internet of Things (IIoT) devices and Digital Twins (DTs) by using quantum communication technologies. The authors introduce a channel encryption scheme based on quantum communication, using entanglement states and quantum teleportation. Further, they propose an Adaptive Key Residue algorithm based on quantum key distribution mechanism. The goal is to improve the security of communication between IIoT devices and DTs.

Lai et al.\cite{chengzhelaiSPDTSecurePrivacyPreserving2022} present a scheme for secure and privacy-preserving traffic control data sharing using digital twins. The scheme incorporates a group signature with time-bound keys for data source authentication and efficient member revocation during the data uploading phase, ensuring secure data storage on the cloud service provider. Moreover, the scheme includes an attribute-based access control technique for flexible and efficient data sharing during the data sharing stage. The primary objective of this scheme is to guarantee effective and secure data sharing for traffic control purposes

in\cite{debenedictisAdoptionSecureCyber2022} De Benedictis addresses the security and trustworthiness of the communication between the digital twin and physical device through various technologies and HW and SW solutions such as Trusted Execution Environment platforms and Physically Unclonable Functions (PUFs) for device authentication. In addition,  Blockchain technology, which provides secure, immutable and auditable data storage for the exchanged critical data is investigated. 

In Olivares-Rojas et al.'s recent study \cite{olivares-rojasCybersecuritySmartGrid2022}, a Digital Twin framework was proposed to avoid sniffing attacks and ensure secure communication between physical and virtual objects. To achieve this, the authors utilized RSA signatures, which provide robust encryption that is difficult to compromise.

A framework called Trusted Twins for Securing Cyber-Physical Systems (TTS-CPS) has been suggested by Sabah Suhail et al.\cite{suhailSituationalAwareCyberphysical2022}. They have integrated blockchain with Digital Twins (DTs) to create a trustworthy environment using Integrity Checking Mechanisms (ICMs) and to track the accountable entity for adding or updating Safety and Security (S\&S) rules. The TTS-CPS framework aims to establish a secure and situational-aware environment in the CPS, supporting a simulated network topology, Human Machine Interfaces (HMIs), Programmable Logic Controllers (PLCs), and physical devices (e.g., robotic arm, motor). The feasibility of the framework has been demonstrated through a prototypical implementation in the automotive industry, and formal verification has been performed to validate its effectiveness.