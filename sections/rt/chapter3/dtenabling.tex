% ======================================================================================================
% NOTES, TODOS
% Modeling and simulation 
% AI and Machine learning
%     Deep learning
% Data Analytic
% Big data 
% Cloud and Edge Computing
% Block Chain

% ======================================================================================================
%
\subsection{Enabling Technologies That Power Digital Twin}
The concept of Digital Twin can be described as a warehouse of data and a virtual representation model of a real-world object with enabling or augmenting technologies to process the data. Its true power lies primarily due to the utilization of enabling technologies\cite{sousaELEGANTSecurityCritical2021}. The use of machine learning, blockchain, cloud computing, and big data analytics has enhanced the capabilities of digital twins, leading to better decision-making processes and improved security. In this section, we will examine the primary technologies that have been explored in the literature for enhancing the capabilities of Digital Twins to provide security services.


% =============================================================================
\subsubsection{AI and Machine Learning}
Researchers have been exploring the integration of machine learning into Digital Twin based cybersecurity solutions. By using machine learning algorithms, they aim to enhance security training, detect power system faults, address security issues in drone networks, detect malicious activities in network traffic, and classify anomalous behavior in smart grids. These studies have shown the effectiveness of using deep learning techniques such as Convolutional Neural Networks (CNN), Generative Adversarial Networks (GANs), Long Short-Term Memory (LSTM) models, and Support Vector Machines (SVM) in detecting security threats such as botnet activities, DoS and DDoS attacks, and intrusion in industrial control systems. The integration of machine learning in digital twin environment provides an innovative and effective approach to tackle the various security challenges in the modern digitalized industry. With that in mind, let's take a closer look at how researchers are leveraging machine learning for various use case scenarios.  

Integration of ML tools into cybersecurity training is one of contributions discussed in a paper by Vakaruk et.al.\cite{vakarukDigitalTwinNetwork2021}, the authors propose to enhance the training processes of the SPIDER cyber range, a playground for cybersecurity training, by integrating ML models into the platform. The authors describe the process of collecting network traffic, labeling it, and training supervised ML models, which are then integrated into the SPIDER platform for deployment in emulation scenarios. The ML models classify the input traffic as either malicious or benign and provide a confidence level for the prediction, which is reported to a dashboard for inspection and analysis by a human user.  

Incorporating deep learning techniques into digital twin systems for power system fault detection is another contribution. Danilczyk et.al.\cite{danilczykSmartGridAnomaly2021} utilized a Convolutional Neural Network (CNN) module for detecting power system faults and found it to be highly effective, with 95\% accuracy. This was better than other ML methods like MLP and LSTM RNN, leading to its selection for its accuracy and efficiency.  
Wu et.al.\cite{wuDeepLearningDriven2022} also used ML in the field of drone security, deploying ML with digital twin to address security issues and detect different attacks and intrusions. The proposed model was tested and compared against existing models, and it was found to have higher accuracy as evidenced by a low root mean square error.  

In a study by Rebecchi et. al.\cite{rebecchiDigitalTwin5G2022}, the application of Generative Adversarial Networks (GANs) in the field of ML was proposed for the generation of synthetic flow-based network traffic that can imitate both normal and malicious activities. The generated data serves as training input for ML models, which are used by "Blue Teams" to detect security threats, such as cryptomining attacks.  

The use of ML algorithms in the simulation and optimization environment of a smart grid is another contribution discussed by Atalay et.al.\cite{atalayDigitalTwinsApproach2020}. The authors suggest ML algorithms for the classification of abnormal behavior during the system's lifecycle.  

In a paper by Salim et.al.\cite{salimBlockchainEnabledSecureDigital2022}, deep learning techniques were used to analyze packet headers for detecting botnet activities that are associated with external IP addresses. The capabilities of deep learning algorithms to identify suspicious activities and malicious nodes within a network allow digital twins to accurately monitor and detect malicious behavior.  

In a study by Sousa et. al.\cite{sousaELEGANTSecurityCritical2021}, ML was used for security analysis with the primary objective of detecting anomalous activity related to Denial of Service (DoS) and Distributed Denial of Service (DDoS) attacks. The developed model examines incoming traffic and categorizes it as either normal or attack traffic.  

In \cite{jiaqiliSpaceSpiderHyper2022} Li et.al. introduced AI as part of a competition to improve the evaluation and enhance the training effect. AI was used as part of advanced technologies, including big data, to provide a better overall educational experience for those participating in the competition.  

Varghese et.al.\cite{vargheseDigitalTwinbasedIntrusion2022} used machine learning techniques for intrusion detection in an industrial control system (ICS). The study used eight supervised ML-based Intrusion detection system(IDS ) techniques (SVM, KNN, NB, RF, LR, ANN, DT, and Gradient Boosting) and evaluated their performance using a labeled dataset. The dataset was split into 70\% for training and 30\% for testing the models. The performance of the algorithms was evaluated based on accuracy, precision, recall, and F1-score. The study also used an ensemble(staccking multiple classifier) approach, to design a signature-based IDS. The IDS used three individual classifiers in Level 0 and a neural network (MLP) classifier in Level 1 to make the final inference.  

 In\cite{akbarianIntrusionDetectionDigital2020} this paper, Akbarian et.al. the authors propose a digital twin-based intrusion detection technique using machine learning. Attack classification approach used in the paper is Support Vector Machine (SVM), which can handle both binary and multi-class classification. The goal is to classify the state of the system into Normal, Scaling attack, or Ramp attack. In short, the intrusion detection mechanism involves a combination of a Kalman filter for attack detection, a particle swarm optimization algorithm for noise estimation, and a SVM algorithm for attack classification.  

 
% =============================================================================
\subsection{Blockchain and Smart Contract}
The use of blockchain technology in combination with Digital Twin has become an interest of research next to Machine Learning. In this section, we compile three studies that use blockchain to improve security and accountability in the industrial internet of things (IIoT) and smart grid environments.  

A paper by Kumar et.al\cite{kumarBlockchainDeepLearning2022}, proposes a digital twin empowered IIoT network that consists of various layers such as IoT Device Layer, Edge-Blockchain, and Cloud-Blockchain Layer. The IIoT devices gather and measure sensor data, while the Edge-Blockchain Layer consists of powerful nodes that are responsible for initial processing and creating tamper-resistant blocks using smart contract-based consensus techniques. The Cloud-Blockchain Layer consists of distributed cloud servers that share historical blockchain data. The primary objective of blockchain in this study is to ensure the integrity and authenticity of data transmitted over public communication channels and leverage smart contracts for secure transmission. In this study, the Ethereum test network was employed for scalability analysis, and IPFS off-chain storage system was incorporated to store encrypted IIoT transactions securely.  

Salim et.al\cite{salimBlockchainEnabledSecureDigital2022}, explore the use of blockchain technology in securing IIoT environments against cyberattacks, particularly botnets. A Blockchain-enabled Digital Twin Framework is proposed, which includes the design of digital twins in the edge layer that synchronize and authenticate data with their corresponding IoT devices using a private blockchain network. The framework leverages smart contracts to authenticate the communication between the digital twins and packet auditors, ensuring the security of data flow in a Smart Factory environment.  

Lopez et.al\cite{lopezDIGITALTWINSINTELLIGENT2021}, focus on how blockchain technology can be used to enhance the security and accountability of data in the smart grid. The security of data at rest is considered very important, as well as the accurate record-keeping of all actions performed by devices within the grid. Blockchain solutions are seen as a solution for this, as they can synchronize information between partners in a secure manner, ensuring data ownership and traceability.  

% =============================================================================
\subsubsection{Cloud and Edge Computing}
In this section we briefly dicuse various studies that have explored the deployment of digital twin systems using cloud computing. 

In \cite{akbarianSecurityFrameworkDigital2021}, a digital twin is deployed in the cloud to leverage its unlimited computational and storage resources. The proposed intrusion detection system is located close to the controller in the cloud, allowing it to access the signals sent to the controller and evaluate its health.
In \cite{grasselliIndustrialNetworkDigital2022} Grasselli et.al., describe how the virtual components of the digital twin are deployed as Virtual Network Functions (VNFs) on Virtual Machines (VMs), replicating elements of the real infrastructure such as firewalls, intrusion detection/prevention systems, and industrial application gateways. The deployment is supported by a cloud infrastructure managed with OpenStack and OSM, providing dynamic control over the compute, storage, and network resources needed for the instantiation of the digital twin.
Deployment of digital twins on edge devices is proposed by Gupta et.al.\cite{guptaHierarchicalFederatedLearning2021} to reduce the gap between physical objects and their digital representations hosted on cloud servers. The authors design a Federated Learning (FL) based Anomaly Detection (AD) model and deploy it on an edge cloudlet associated with the patient's digital twin. Further, the use of edge cloudlet computing enhances data privacy and improves model performance.
In \cite{saadImplementationIoTBasedDigital2020} Saad et.al, a digital twin solution is introduced that is composed of two parts where the second part is built as a function on the cloud.

% =============================================================================
\subsubsection{Big Data and Data Analytics}
By incorporating data analytics techniques, digital twins can provide powerful insights\cite{li}. into the behavior and performance of physical systems. In this section, we provide a review of studies that leverage big data processing and analytics to augment digital twin.  
Salvi et. al.\cite{salviCyberresilienceCriticalCyber2022} study introduces a conceptual model that integrates data analytics and causal analysis to detect potential attacks on critical cyber infrastructures (CCIs). This model aims to improve the security of CCIs by detecting anomalies and potential threats in real time. 
Another contribution is from Wang et.al.\cite{wangDTCPNDigitalTwin2022}, who proposed a platform that integrates big data processing capabilities to efficiently obtain, store, and search data of specific objects. This platform is designed to support network application playback for simulation, allowing for the analysis of complex data sets and the improvement of network security. Hussaini et.al.\cite{hussainiTaxonomySecurityDefense2022} discussed how data analytics is used to enhance the digital twin (DT) system in detecting anomalies and improving security in critical process systems (CPS). They argue, advanced data analytics, which are based on machine learning and deep learning, strengthens the DT's evaluation and monitoring capabilities against various attacks, such as evasion, inference, and model poisoning. 
Li \cite{jiaqiliSpaceSpiderHyper2022} used big data and artificial intelligence to process and analyze large amounts of data to provide deeper insights into network security scenarios and improve the training experience for users.  
Dietz et.al. \cite{dietzIntegratingDigitalTwin2020} present security analytics tools for processing log data for security incident detection. The log data is collected by Filebeat from the digital twin simulation, normalized by Logstash into semi-structured JSON documents, processed by a correlation engine (Dsiem) to detect incidents and trigger alarms, stored in Elasticsearch, and visualized and analyzed using Kibana. The authors in the study done by Almeabed et.al.\cite{almeaibedDigitalTwinAnalysis2021} present a framework based on data processing and analytics capability for vehicular digital twins.  
