
% ======================================================================================================
% NOTES, TODOS
% Telecommunication -> 1, 2
%     5G network 
% Power grid, smart grid Energy sector-> 3, 4, 5
% Transportation -> 1, 2
% Agriculture sector
% Automotive -> 1, 2
% Water treatment -> 1
% Space industry -> 1
% General
%     Industry Control system -> 1, 2
%     Cyber-Physical System -> 1, 2, 3, 
%     IIoT network -> 1
%     Digital Enterprises  -> 1
%     smart manufacturing -> 1

% ======================================================================================================



\subsection{The Impact of Digital Twin on Key Industry Sectors}

\subsubsection{Power Grid}
In the study by Danilczyk et.al.\cite{williamdanilczykANGELIntelligentDigital2019}, a framework named "Automatic Network Guardian for Electrical Systems (ANGEL)" was proposed that models both the cyber and physical layers of a microgrid, providing real-time data visualization with the goal of enhancing cyber-physical security. Another study by Saad et al.\cite{saadImplementationIoTBasedDigital2020} presents a digital twin for networked microgrids with the objective of detecting cyber attacks such as false data injection, denial of service, and coordinated attacks. Hossen et al.\cite{hossenDigitalTwinSelfSecurity2021} proposes a method to maintain the security of smart grids using digital twin and a "knowledge-based self-security algorithm". This approach checks incoming commands for power setpoints to protect the smart grid from man-in-the-middle attacks. Atalay et al.\cite{atalayDigitalTwinsApproach2020} present a digital twin-based approach for the entire lifespan of a smart grid with the aim of tackling the current challenge of insufficient standards for evaluating the security of smart grids. The digital twin approach is intended to create a unified platform for standardized models and to simulate the behavior of the physical grid precisely, avoiding any service interruptions caused by security testing on the actual grid. Giovanni et al.\cite{giovannipaolosellittoEnablingZeroTrust2021} introduce a dynamic, maintenance-aware approach for building a cybersecurity digital twin for smart grids, with the ability to enforce security policies in line with topological changes and enabling a zero-trust architecture (ZTA). Salvi \cite{salviCyberresilienceCriticalCyber2022} targets the electrical energy sector with the aim of increasing the cyber-resilience of Critical Control Infrastructures (CCIs) in the context of the European Union's electric power ecosystem based on the implementation of integrated digital twin. In\cite{danilczykSmartGridAnomaly2021} Danilczyk's  study, a deep learning architecture was developed for anomaly detection in a distributed smart grid infrastructure using state-of-the-art digital twin techniques to detect and locate faults, thereby improving the performance and cost of power distribution.  
% \subsection{Transportation}

\subsubsection{Agriculture}
In\cite{chukkapalliCyberPhysicalSystemSecurity2021} Chukkapalli et al. introduce a security surveillance system for a smart farm that keeps track of the data generated by sensors and alerts the farm owners. The system includes the collected sensor data, a smart farm ontology for creating knowledge graphs, and digital twin modules for anomaly detection. The researchers first use the collected data to generate knowledge graphs with the smart farm ontology and then use the digital twin to train the anomaly detection model using Principal Component Analysis. The authors demonstrate that the digital twin-based anomaly detection model can be used to detect various anomalies in the smart farm.
% \subsection{Health}

\subsubsection{Automotive}
In \cite{almeaibedDigitalTwinAnalysis2021}, Almeaibed et.al proposed a standard framework for the creation of vehicular digital twins that streamlines data collection, processing, and analytics. The authors also highlight the importance of digital twin security through a case study that showcases how radar sensor readings can be altered by hackers, potentially leading to collisions. The paper concludes by providing insights on the implementation of digital twins in the autonomous vehicle industry and addressing privacy, safety, security, and cyber attack mitigation concerns. Another research that focuses on the Automotive industry to tackle safety and security issues in connected cars and Autonomous Driving is presented by \cite{veledarDigitalTwinsDependability2019}. The authors use Digital Twins as a virtualization system to enhance stakeholder confidence in technology and detect risks early. A study\cite{marksteinerUsingCyberDigital2021} funded by Austrian Research Promotion Agency (FFG) and the ECSEL Joint Undertaking, with support from the European Union's Horizon 2020 program, proposes an automated approach for cybersecurity testing in a black box setting. The methodology combines pattern matching-based binary analysis, translation mechanisms, and model checking techniques to generate meaningful attack vectors with minimal prior knowledge of the system being tested. It is designed to meet the security requirements outlined by UNECE regulation R155 for vehicular systems  
\subsubsection{Water Treatment}
In \cite{sugumarAssessmentMethodDetecting2019}, Sugumar et.al aimed to create a digital twin for the Secure Water Treatment plant, using formal methods to design, model, and assess the performance of an anomaly detection system.
\subsubsection{Space Industry}

In \cite{adrienbacueDigitalTwinsEnhanced2022} this research, the authors highlights the utilization of digital twins (DTs) in the aerospace manufacturing industry where the Industrial Internet of Things (IIoT) is being integrated in Airbus Defence and Space factories. The study's results demonstrate that DTs can effectively aid the industry in enhancing cybersecurity while adopting connected and collaborative manufacturing techniques.
Hou et.al. \cite{houDigitalTwinRuntime2022} proposes a method that integrates digital twins with runtime verification for the secure monitoring and verification of satellites. The approach employs state synchronization and encryption for secure communication and a Linear Temporal Logic (LTL)-based verification engine.
In \cite{jiaqiliSpaceSpiderHyper2022} Jiaqi et.al proposed the construction of the Space Spider, a hyper-large scientific infrastructure to simulate the life cycle elements of the space internet and build a system for attack and defense.
