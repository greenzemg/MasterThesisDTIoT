
% ======================================================================================================
% NOTES, TODOS
% Telecommunication -> 1, 2
%     5G network 
% Power grid, smart grid Energy sector-> 3, 4, 5
% Transportation -> 1, 2
% Agriculture sector
% Automotive -> 1, 2
% Water treatment -> 1
% Space industry -> 1
% General
%     Industry Control system -> 1, 2
%     Cyber-Physical System -> 1, 2, 3, 
%     IIoT network -> 1
%     Digital Enterprises  -> 1
%     smart manufacturing -> 1

% ======================================================================================================



\subsection{Digital Twin Use-Case Based on Targeted Industry}
In our study, we classified the papers based on the industry they targeted to identify domain-specific challenges. Our approach to classifying the papers was carefully designed to minimize any potential biases. Our finding revealed that Digital Twin based security solution has been widely adopted in several industries ranging from Smart Grid(SG) to Smart Homes(SH) for various purpose. Intrusion detection, anomaly detection, training and testing, botnet detection etc are among  the main function provided through this digital technology. 


To avoid bias classification, we followed a systematic way of identifying the target industry based on the following condition. 

\begin{itemize}
    \item If the authors explicitly specify the industry to which their proposed solution is targeted for, we consider this to be the target industry of the paper.
    \item In cases where the target industry was not explicitly mentioned, we looked at the experiments and use cases presented in the paper to identify the targeted industry.
    \item If we couldn't find the target industry through the previous methods, we searched for any discussions of Industrial Control Systems (ICS) or Cyber-Physical Systems (CPS) in the paper, and if found, we categorised the paper under ICS/CPS.
\end{itemize}


It is important to note that our classification methodology excluded review papers and studies that solely focused on providing security mechanisms for digital communication between DT and (I)IoT. 



Using the above criteria, we found the following main industry sectors illustrated in Figure \ref{fig:use-dt}. \\
% Power grid, Agriculture, Helath, Smart home, Transportation, Autonomous Vehicle, Water Treatment, Space industry, Smart Factory, Automotive industry, 

\begin{center}
 

\smartdiagram[constellation diagram]{
  Digital Twin,
  Power grid, Agriculture, Helath, Smart home, Transportation, Autonomous vehicle, Water Treatment, Space industry, Smart Factory, Automotive industry
}
\captionof{figure}{Use Cases Of Digital Twin}
\label{fig:use-dt}
\end{center}
\subsubsection{Power Grid}
In the study by Danilczyk et al.\cite{williamdanilczykANGELIntelligentDigital2019} proposed a framework named "Automatic Network Guardian for Electrical Systems (ANGEL)" which uses real-time data visualization to enhance the security and resiliency of microgrids. The framework models both the cyber and physical layers of the microgrid, allowing it to detect discrepancies between simulated and physical systems under various operating conditions. The framework's two-way coupling between the simulation and the physical system enables it to update and improve its simulations, detect unnatural changes, and evaluate meter data accuracy, thereby improving security. ANGEL also has self-healing capabilities that can mitigate component failures and cyber-attacks. While the ANGEL framework is promising, it has limitations, including potential false positives and difficulty detecting some types of malicious attacks. Additionally, the framework is still in development and has not yet been tested on a real-world microgrid system for further evaluation.

Another study by Saad et al.\cite{saadImplementationIoTBasedDigital2020} presents an IoT-based digital twin (DT) for microgrids that aims to improve their resilience against cyber attacks. The proposed framework is implemented as a cloud-based DT platform that acts as a central hub for the networked microgrid system. The digital twin is designed to model both the physical and cyber layers of the microgrid, allowing it to detect false data injection (FDIA) and denial of service (DoS) attacks. The framework utilizes observer-based What-If scenarios to take corrective action when an attack is detected, ensuring the safe and seamless operation of the networked microgrids. The proposed DT framework is validated using a practical setup of the distributed control system and Amazon Web Services (AWS), and is able to quickly detect and mitigate a range of cyber attacks. The authors argue may the fusion of deep learning and Luenberger Observer(LO) enhances the speed, accuracy, and predictability of attacks. In general, the proposed IoT-based digital twin framework presents a practical solution to improve the resilience of microgrids against cyber attacks.

In\cite{hossenDigitalTwinSelfSecurity2021} this paper, Hossen et al. propose a knowledge-based self-security algorithm that evaluates the incoming power setpoints for safety before implementation. The algorithm utilizes the steady-state and dynamic behaviours of the inverter, which were experimentally determined using laboratory equipment to create a digital twin of the inverter. The study demonstrates that this technique can help protect smart grids from man-in-the-middle attacks by thoroughly examining incoming commands via the digital twin before engaging them in the local controller.

The study undertaken by Atalay et al.\cite{atalayDigitalTwinsApproach2020} focuses on providing an overview of smart grid cybersecurity standards and reviews major threats to smart grid environments at the physical, network, and application layers. In this study, the authors argue that despite the prevalence of smart grids in energy distribution networks, there is a lack of standards for comprehensive security assessment, which is a critical shortcoming. With aim to address this gap, the authors propose a Digital Twins based approach for the security testing lifecycle of smart grids, by accurately modelling the functioning of the physical grid and running security testing on the actual grid without causing disruption. This approach has the potential to become an important tool for standardisation. While the paper presents an innovative framework for security testing, it lacks experimental validation and implementation details for real-event scenarios.


In their study, Sellitto et al.\cite{giovannipaolosellittoEnablingZeroTrust2021} proposed a methodology to build a cybersecurity digital twin of a Smart Grid based on its architectural blueprint. The goal of the methodology is to enable the adoption of Zero Trust Architecture (ZTA) and dynamic enforcement of security policies for devices connecting to the grid. The authors presented a novel approach to dynamically align the digital twin with its real-world counterpart, creating a maintenance-aware model for the Smart Grid. This was achieved by adopting an architectural view that gets dynamically aligned with the state of the real-world counterpart during deployment and operation time. The authors laid the foundation for a digital twin model that allows dynamic enforcement of security policies that reflect Smart Grid topology changes over time. 


Salvi et al.\cite{salviCyberresilienceCriticalCyber2022} targets the electrical energy sector with the aim of increasing the cyber-resilience of Critical Control Infrastructures (CCIs) using a digital twin implementation to address risks associated with the integration of computational, communication, and physical aspects of CCIs. It seeks to provide increased situational awareness, a common understanding of incidents, and enhanced response capacity to minimize response time and reduce the impact of cyber-attacks on organizations and society. However, the study is limited by the fact that it only focuses on the conceptual model, rather than the implementation of the digital twin, which may require further validation through proof of concepts in different CCI contexts. Nevertheless, this research addresses the needs expressed by key stakeholders in the electrical energy sector and presents design principles that can be applied in disaster management contexts.


A study by Danilczyk et al.\cite{danilczykSmartGridAnomaly2021} presents a deep learning convolutional neural network (CNN) as a module within the Automatic Network Guardian for Electrical Systems (ANGEL) Digital Twin environment to detect physical faults in a power system. The approach uses high-fidelity measurement data from the IEEE 9-bus and IEEE 39-bus benchmark power systems to detect if there is a fault in the power system and to classify which bus contains the fault. The anomaly detection CNN algorithm was able to identify the existence of a fault with near-perfect accuracy and classify the location of the fault with an accuracy of nearly 95\% for both systems. The long-term goal of this project is to have the Digital Twin with the anomaly detection CNN running alongside the physical smart grid. However, the study's limitation is that, due to the small timescales present in power systems, the inference speed of the network will be of critical importance. For real-time implementation, more powerful hardware would be beneficial to the overall performance of the integrated system. Despite this limitation, deep learning algorithms show significant promise in the detection and location of power system faults and can improve performance and reduce the cost of power distribution.

To overcome limitations in security studies of Smart Grids (SG) in physical test-beds Kandasmy et al.\cite{kandasamyElectricPowerDigital2022} build a digital power twin that enables the deployment of real-world attacks and countermeasures while allowing easy modification of components and configurations. The tool presented by the author named EPICTWIN, a digital twin for a power physical test-bed, allows users to validate the security and safe operation of critical components in a more realistic environment, reducing the gap between physical and simulated test-bed environments. They claim their tool provides an attacker designer(AD) and attack launcher(AL), unique tools that enable researchers to validate and improve defence mechanisms even without expertise in offensive security testing. Finally, the authors highlight the uniqueness of their contributions in building a digital twin of an existing cyber-security test-bed, presenting a procedure that can be extended to any type of system, and providing unique tools for launching systematic attacks on the twin.

\subsubsection{Smart Factory}
Lopez et al.\cite{lopezDIGITALTWINSINTELLIGENT2021} aim in their research to analyze the evolution of digital twins in smart grid infrastructures and their role in implementing intelligent authorization policies. The authors study the application of AI technologies, including machine learning and blockchain, in the context of digital twins to manage dynamic information flows and detect cybersecurity issues in real time. They provide a mid-term and long-term analysis of the pending challenges of DTs and discuss the three-stage process of DT evolution, starting from monitoring systems with limited analysis capabilities to fully semantic, self-learning platforms. The contribution of this article lies in the analysis of the future smart grid through the evolution of digital twins, pointing out the most relevant challenges they face. The authors conclude that digital twins will play a fundamental role in driving the progress of the electricity grid toward a fully decentralized and autonomous model, governed by intelligent authorization systems. However, standardization and information security efforts are necessary, along with deep research into machine learning specifically applied to critical infrastructures and smart cities.


 In\cite{shitoleRealTimeDigitalTwin2021} paper presented by Shitole et al. aims to develop a low-cost Real-Time Digital Twin (RTDT) of an interconnected and distributed Residential Energy Storage System (RESS) controlled and monitored via Cloud based Energy Management System (CEMS), in order to analyse the cyber-security of such systems and develop appropriate Intrusion Detection Systems against cyber attacks. The proposed RTDT allows for flexibility in modifying, scaling, and replicating the system without compromising its real-time fidelity. The development procedure can be easily replicated to develop RTDT of any Cyber-Physical System (CPS) or micro-grid test-beds. The paper presents a systematic procedure for the development of the RTDT and verifies its performance through an experimental case study. The RTDT is developed using a low-cost single board computer with Simulink Desktop Real-Time, which reduces overall development cost. Overall, this paper presents a reliable and economical solution for cyber security studies on RESS through the development of an RTDT.


 Salim et al.\cite{salimBlockchainEnabledSecureDigital2022} propose a secure blockchain-enabled digital framework for the early detection of botnet formation in a smart factory environment. The proposed framework integrates a digital twin (DT), a packet auditor (PA), deep learning models, blockchain, and smart contracts(SC) for securing the data flow of a smart factory environment. The DT is designed to collect device data and inspect packet headers for connections with external unique IP addresses with open connections. Data are synchronized between the DT and the PA for detecting corrupt device data transmission, and smart contracts authenticate the DT and PA to ensure malicious nodes do not participate in data synchronization. Botnet spread is prevented using DT certificate revocation. A comparative analysis with existing research shows that the proposed framework provides data security, integrity, privacy, device availability, and non-repudiation.


In\cite{becueCyberFactorySecuringIndustry40with2018} paper by Bécue et al. discuss ITEA initiative CyberFactory\#1 project, which aims to develop a system of systems to optimize and ensure the resilience of digital factories and factories of the future (FoF) in the face of increasing digitization and connectivity. The project focusses on optimising the efficiency and security of the network of factories, proposing novel architectures and methodologies to address cyber and physical threats and safety concerns. It also integrates technical, economic, human, and societal dimensions. In this study Digital twins is used to support cybersecurity testing and training, together with cyber ranges, to enable risk anticipation and accurate impact prediction. The project demonstrates key capabilities in realistic environments and reflect the variety of possible new factory types and business model shifts.



\subsubsection{Health}
An automated framework for improving cybersecurity in IoT-based healthcare applications using DT that includes innovative healthcare security techniques such as system modelling, traffic and attack generation, impact assessment, attack and response strategies, and cyber-attack prevention processes proposed by Pirbhulal et al.\cite{pirbhulalNovelFrameworkReinforcing2022} The authors investigate the applicability of DT for cyber-attacks prevention, and present a strategic procedure for enhancing cybersecurity. The proposed framework can help update access control policies and enhance cybersecurity, and it provides an automated cybersecurity solution by incorporating system models and resolving known vulnerabilities and threats. However, the limitation of this research is that it is a theoretical study and needs to be validated through experiments and simulations. The authors conclude that DT is a valuable tool for enhancing cybersecurity in healthcare systems, as it provides analysis, design, and optimization of systems to improve accuracy, speed, and effectiveness, and it can simulate security breaches and develop decision-making and mitigative responses to simulated cyberattacks. 

\subsubsection{Smart Home}
The design and implementation of a DT system for smart metering systems (SMS) in a smart home setting are presented in the work by Olivares-Rajos et al.\cite{olivares-rojasCybersecuritySmartGrid2022}. The system is mainly composed of a DT framework controller (DTFC) responsible for mapping real objects with its virtual representation, exchange values between DTs, notify events and alarms between all the objects, and simulating cyber threats and attacks. The SMS is mainly composed of an Smart Meter(SM) in the premises of the end-users, data concentrator (DC) that collect the SM consumption/production data, and metering database management system (MDMS) server that store all the information collected by DC. The authors conclude that the use of digital twins (DTs) is feasible in various contexts of the smart grid, particularly in cybersecurity testing.

In\cite{xiaoCommandFenceNovelDigitalTwinBased2022} this papery,  Xiao et al. propose a novel digital-twin-based security framework, CommandFence, to protect smart home systems from malicious and benign apps with design flaws or logical errors that may cause harm to the user when executed. The framework uses an Interposition Layer to interpose app commands and an Emulation Layer to execute these commands in a virtual smart home environment and predict whether they can cause any risky smart home state when correlating with human activities and environmental changes. If a sequence of app commands can potentially lead to a risky consequence, they are treated as dangerous, and the framework drops them before any insecure situation can occur. The authors fully implemented the CommandFence framework and tested it on 553 official SmartApps on the Samsung SmartThings platform, 10 malicious SmartApps created by Jia et al., and 17 benign SmartApps with logic errors developed by Celik et al. The experiment successfully identified 34 potentially dangerous SmartApps out of 553 official SmartApps, 7 out of 10 malicious SmartApps, and achieved 100\% accuracy for the 17 benign SmartApps with logic errors.CommandFence is orthogonal to the well-received permission-based access control mechanisms and can be implemented as plug-in software without any hardware upgrades. 

\subsubsection{Transportation}

Cathey et al.\cite{glenandbensonjamesandguptamaanakandsandhuravicatheyEdgeCentricSecure2021} presented a novel edge-centric access control architecture for IoT environments using techniques called Tag Based Access Control(TBAC), which utilizes digital twins to separate data based on tags assigned on the fly, limiting access to authorized users and applications. The proposed architecture is lightweight, supports low-latency and real-time security mechanisms, and improves system security and efficiency by minimizing data sharing and granting an individual access to data subsets. The paper demonstrates the usefulness of TBAC in smart environments such as manufacturing and internet-connected vehicles.

A DT based tool called Testing and Simulation(TaS) presented in\cite{luongnguyenDigitalTwinIoT2022} paper by Nguyen et al. for testing and simulating IoT environments in order to improve testing methodologies and evaluate the possible impact of IoT systems on the physical world. The tool supports functional and non-functional testing and can be used to detect and predict failures in evolving IoT environments. The tool has been tested and validated through experiments performed in the context of the H2020 ENACT project. The contribution of the paper lies in the design of a tool that allows for the real-time connection of the physical system to a new software version deployed in the DT, enabling verification that changes made in the code do not impact existing software functionality. The tool has been applied in different domains, showing that it is generic and can be used to achieve different test objectives. Although TaS automates several steps in the test process, the author points to limitations regarding testing scenario generation that could be improved.


\subsubsection{Autonomous Vehicle}
In \cite{almeaibedDigitalTwinAnalysis2021}, Almeaibed et.al proposed a standard framework for the creation of vehicular digital twins that streamlines data collection, processing, and analytics. The authors also highlight the importance of digital twin security through a case study that showcases how radar sensor readings can be altered by hackers, potentially leading to collisions. The paper concludes by providing insights on the implementation of digital twins in the autonomous vehicle industry and addressing privacy, safety, security, and cyber attack mitigation concerns.

Another research that focuses on the autonomous vehicle to tackle safety and security issues in connected cars and Autonomous Driving is presented by Veledar et al.\cite{veledarDigitalTwinsDependability2019}.
With the scope of IoT4CPS, a guideline for secure integration of IoT into Autonomous Driving(AD), the authors suggested three main steps for designing Digital Twins to address security vulnerabilities in AD. The proposed three steps are: Firstly, identifying assets, modelling them, and defining security and safety objectives. Secondly, designing security and safety evaluation metrics. Lastly, performing threat modelling and test case demonstrators based on security and safety risk assessment and forecasting.

A study by Marksteiner et al. \cite{marksteinerUsingCyberDigital2021} which is funded by Austrian Research Promotion Agency (FFG) and the ECSEL Joint Undertaking, with support from the European Union's Horizon 2020 program, proposes an automated approach for cybersecurity testing in a black box setting. The methodology combines pattern-matching-based binary analysis, translation mechanisms, and model-checking techniques to generate meaningful attack vectors with minimal prior knowledge of the system being tested. It is designed to meet the security requirements outlined by UNECE regulation R155 for vehicular systems  

Xu et al.\cite{xuEfficientAuthenticationVehicular2021} have introduced a conceptual framework called the Vehicular Digital Twin (VDT), designed to aid in the fusion, calculation, and communication of data in autonomous vehicles (AVs). The VDT, which is stored on the cloud, is constantly updated in real-time to match the AV it represents. It can also connect with other digital twins to obtain necessary information. To maintain secure communication between the AV and the DT, the authors propose an authentication protocol that combines the secret handshake scheme and group signature. This protocol provides anonymity for honest members while allowing for traceability if necessary, and also ensures the authenticity of messages sent between the AV and the DT. The result of the performance analysis shows that the authentication protocol had a less computational cost while satisfying necessary security requirements effectively. 

\subsubsection{Water Treatment}
In \cite{sugumarAssessmentMethodDetecting2019}, Sugumar et al. examine the use of timed automata models for detecting cyber-attacks on critical infrastructure through a design-centric anomaly detection method. The authors create a digital twin model that replicates the behavior of real-world systems, such as water treatment plants, and use an attack launcher to test the deployed security method's effectiveness against various attacks, including scaling and pulse attacks. The experiment results demonstrate that the proposed approach can accurately detect cyber-attacks outperforming other methods that involve simulations or direct testing on operational testbeds. However, the authors do not address the potential limitations of their approach, such as scalability when applied to more complex systems with more intricate components. Additionally, they only evaluated a subset of potential attack templates, including scaling and pulse attacks and did not consider other types of attacks, such as random or ramp attacks, which could pose a threat to critical infrastructure systems. 

The authors In\cite{maillet-contozEndtoendSecurityValidation2020}, introduce an approach for the integration, verification, and validation of security in IoT devices. The approach is based on the digital twin concept and involves creating a comprehensive virtual representation of a physical device, composed of black box and white box models at different abstraction levels. By using this approach, the cost impact of adding security to physical devices is reduced, while still ensuring the security and functionality of the device. This approach provides a new way to think about integrating security in the IoT and has the potential to improve the overall security and efficiency of connected devices. To validate their approach they conducted two use case studies based on H2020 critical infrastructure of water management project.

\subsubsection{Space Industry}

In \cite{adrienbacueDigitalTwinsEnhanced2022} this research, the authors highlight the utilization of digital twins (DTs) in the aerospace manufacturing industry where the Industrial Internet of Things (IIoT) is being integrated with Airbus Defence and Space factories. They conducted a case study to show how DT based simulation solutions can be used for simulating attacks and designing countermeasures without affecting the internal operation of manufacturing. The study's results demonstrate that DTs can effectively aid the industry in enhancing cybersecurity while adopting connected and collaborative manufacturing techniques.

Hóu et al.\cite{houDigitalTwinRuntime2022} propose a method for improving the capability of detecting cybersecurity issues in satellite communication using run-time verification based on digital twins. This involves monitoring and evaluating software or hardware system against user-defined properties. The proposed method uses state synchronization and encryption for secure communication between the physical twin and digital twin and incorporates a cryptographic algorithm into their state synchronization protocol to guarantee the correctness of the state. However, the framework has some weaknesses, such as the lack of discussion on the security protocols used for secure communication and the absence of security and performance analysis.

% Hou et.al. \cite{houDigitalTwinRuntime2022} proposes a method that integrates digital twins with runtime verification for the secure monitoring and verification of satellites. The approach employs state synchronization and encryption for secure communication and a Linear Temporal Logic (LTL)-based verification engine.
In \cite{jiaqiliSpaceSpiderHyper2022} Li et al. claim adding contribution by defining characteristic hyper-large scientific infrastructures and evaluation indicators of traditional large scientific infrastructures. Due to security risks facing the space Internet, the paper proposes constructing a hyper-large scientific infrastructure called Space Spider, which simulates the space Internet's entire life cycle and creates a system for space Internet attack and defence. Additionally, the paper introduces Spiderland, an open experimental platform for studying space Internet applications and security.

\subsubsection{Enterprise Network}
 Wang et al.\cite{wangDTCPNDigitalTwin2022} suggest a Digital Twin Cyber Platform based on NFV (DTCPN) to address the challenges in developing large-scale networks, such as complex network management and operation, and high risk and overhead of on-the-fly optimization of product network. The DTCPN combines the advantages of digital twin and NFV technology to eliminate complex and inaccurate modeling process, support Real-Virtual interaction, and provide high fidelity. The platform is designed to facilitate the design, analysis, testing, and evaluation of network technologies and devices in a rapid, accurate, and efficient way. The article concludes that DTCPN has technical advantages that can play a significant role in network security, network management, and network applications. Further optimization and enrichment of the DTCPN's design and functions are planned for the future.

This\cite{hadarCyberDigitalTwin2020} research paper proposes a novel method for automatically gathering and prioritising security controls requirements (SCRs) for rapid risk reduction in active networks. It introduces a cyber digital twin, based on attack graph analytics, that associates network information with attack tactics, evaluates the efficiency of implemented SCRs, and automatically detects missing security controls. The paper presents a framework and methodology to construct a contextual cyber digital twin, ranking the risk impact of security controls, and prioritising SCRs to reduce risk impact as quickly as possible. The paper also provides visualizations of a field experiment conducted via an active network, demonstrating successful results in reducing cyber impact and identifying missing security controls for future implementation. The proposed cyber digital twin simulator offers several new risk reduction methods for automatically selecting SCRs and can be used as a valuable tool for existing cybersecurity evaluation and future cybersecurity budget proposals.

\subsubsection{ICS/CPS Environment Use Case}

This\cite{vargheseDigitalTwinbasedIntrusion2022} research from Varghese et al. introduces a digital twin-based security framework for industrial control systems (ICS) that can simulate attacks and defence mechanisms. Four process-related attack scenarios are tested on an open-source digital twin model of an industrial filling plant. The study proposes a real-time intrusion detection system based on a stacked ensemble classifier that combines predictions from multiple algorithms. This model outperforms previous methods in terms of accuracy and F1 Score, detecting intrusions in close to real-time (0.1 seconds). The proposed framework extends the capabilities of an existing ICS digital twin framework with an ML-based IDS module and provides a platform for developing intrusion detection and prevention systems.


In\cite{masiSecuringCriticalInfrastructures2023} Masi et al. discuss the  use of Digital Twin (DT) technology to improve the cybersecurity of critical infrastructures. The paper presents a Cybersecurity View that can be derived from an Enterprise Architecture (EA) approach to cybersecurity. This view facilitates the identification of adequate cybersecurity measures for the system while improving the overall system design. The methodology proposed in this paper can be applied to the whole system life-cycle: from design/construction to production/exploration and phaseout. The paper addresses two main challenges: the custom-built nature of Industrial Automation and Control Systems (IACS) and the impedance between the EA models used in industrial automation and the models used in visual threat modeling. To address these challenges, the paper proposes the adoption of a reference architecture framework suitable for IACSs and uses a set of rules to build a cybersecurity view of IACS that is amenable to translation into a visual threat modeling language.  The practical usefulness of the proposed methodology is demonstrated through two real-world use cases: the Cooperative Intelligent Transport System (C-ITS) and the Road tunnel scenario. 

Dietz et al.\cite{dietzEmployingDigitalTwins2022} discusses the security issues of industrial control systems (ICS) and proposes an approach for introducing security-by-design system testing with the help of a digital twin. The authors argue that proper system testing can reveal the system’s vulnerabilities and provide remedies, and that security measures should be carried out as early as possible, especially to render systems secure-by-design. The authors implement a digital twin representing a pressure vessel and demonstrate how to carry out each step of their proposed approach, identifying vulnerabilities and showing how an attacker can compromise the system by manipulating values of the pressure vessel with the potential to cause over-pressure, which, in turn, can result in an explosion of the vessel. Overall, the digital twin presented in this study as a tool for security-by-design system testing in industrial control systems.


In another study, Dietz et al.\cite{dietzUnleashingDigitalTwin2020} discuss the challenges and opportunities presented by Industry 4.0 (I4.0) in relation to industrial security. As traditional operational technology (OT) systems are increasingly integrated with general-purpose IT systems, which creates novel attack vectors in industrial ecosystems, the author argues that I4.0 technologies, such as digital twins (DTs), can contribute to industrial security by providing virtual entities that represent physical industrial systems. They also added DTs offer opportunities for security, such as simulation and replication of system behavior, and can play an important role in mitigating and avoiding risks associated with critical infrastructures. DTs can also provide comprehensive information about the asset's status, history, and maintenance needs, and can support an immediate reaction to security incidents. In conclusion, the author suggests that DTs can be an important tool to strengthen industrial security in the context of I4.0.


To enhance cyber-situation awareness for operators Eckhart et al.\cite{eckhartEnhancingCyberSituational2019} propose a digital-twin cyber situational awareness framework for cyber-physical systems (CPSs). The paper builds upon and extends the previous research on leveraging the digital-twin concept for securing CPSs. The proposed framework provides advanced monitoring, inspection, and testing capabilities that support the operations staff in gaining situation perception, comprehension, and projection. The framework enables real-time visualisation and a repeatable, thorough investigation process on a logic and network level. The technical use cases illustrate the added value of the proposed framework for improving cyber situational awareness regarding CPSs, such as risk assessment, monitoring, and incident handling. However, the paper acknowledges that further development effort is required to improve the visualization of digital twins and to complete the record-and-replay feature. 


Dietz et al.\cite{dietzIntegratingDigitalTwin2020} propose a security framework that leverages digital twin-based security simulations to enhance Security Operations Center (SOC) and Security Information and Event Management (SIEM) systems in mitigating the expanding attack surface in industrial environments. The authors demonstrate how the framework can simulate attacks, analyze their impact on virtual counterparts, and create technical rules for implementation in SIEM systems. In general, the framework comprises five activities: asset modeling, attack modeling, simulation execution, result analysis, and action implementation. The paper concludes by highlighting the contribution of the proposed framework  to SOC security strategies and suggests future work to evaluate its effectiveness and performance. Additionally, the authors recommend extending the framework to integrate with cyber threat intelligence (CTI) to provide more utility to SOC analysts.

The paper by Grasselli et al.\cite{grasselliIndustrialNetworkDigital2022} presents the implementation of a digital twin for industrial networks to facilitate cyber-security testing and validation without interfering with the real cyberphysical system. The proposed methodology involves the use of technologies such as Cloud Computing and Network Function Virtualization (NFV) and is supported by the ETSI NFV Management and Orchestration (MANO) framework to automate the deployment of the digital twin. The authors describe the different steps involved in the lifecycle management of the digital twin, which include the preparation phase, commissioning phase, operation phase, and de-commissioning phase. The paper also includes a quantitative evaluation of the time needed to perform these actions. Overall, the paper highlights the potential of digital twin technology in addressing cyber-security concerns in Cyber-Physical Systems.


In\cite{xuGametheoreticApproachSecure2020} paper, Xu et al. discuss the issue of data-integrity attacks in Cyber-Physical Systems (CPSs), particularly the Sensor-and-Estimation (SE) attack where the attackers tamper with sensing or estimated information of CPSs. The authors propose a framework that uses a Chi-square detector in a Digital Twin (DT) to monitor the estimation of the physical system and collect evidence to detect any attack. They also use a Signaling Game with Evidence (SGE) to find the optimal attack and defence strategies. The proposed framework is designed to mitigate the impact of the attack on physical performance and to guarantee the stability of CPSs. Analytical results show that proposed defensive strategies can effectively restrict attackers' ability to carry out stealthy estimate attacks.


Sousa et al.\cite{sousaELEGANTSecurityCritical2021} introduces an off-premises approach to designing and deploying digital twins (DTs) for securing critical infrastructures. The proposed solution involves the use of high-fidelity replicas of Programming Logic Controllers (PLCs), which provide a faithful environment for security analysis and evaluation of potential mitigation strategies. The authors highlight that while on-premises implementation can be costly, DTs offer a reliable option for security analysis and evaluation. However, adapting security and safety monitoring mechanisms to synchronize with the DT replica can be challenging. To address this issue, the paper presents an off-premises approach that uses real-time, high-fidelity emulated replicas of PLCs along with scalable and efficient data collection processes. The approach includes the development and validation of Machine Learning models to mitigate security threats such as Denial of Service (DoS) attacks. The results of the experiments demonstrate that DTs provide a faithful environment for security analysis and evaluation of potential mitigation strategies against high-impact threats such as distributed DoS attacks.

The use of digital twins as security enablers and data sharing for Industrial Automation and Control Systems (IACS) discuss in detail by Gehrmann et al.\cite{gehrmannDigitalTwinBased2020}. The authors identify design-driving security requirements for digital twin-based data sharing and control and propose a state synchronisation model to meet these requirements. They also evaluate the security and performance of the proposed architecture through a proof-of-concept implementation with a programmable logic controller (PLC) software upgrade case. The paper concludes that a digital twin-based security architecture can be a promising way to protect IACS while enabling external data sharing and access, but further research is needed to fully implement and evaluate the proposed architecture.

Motivated by the increasing connectivity of Industrial Control Systems(ICS) which makes them more vulnerable to cyber attacks Akbarian et al\cite{akbarianIntrusionDetectionDigital2020} proposes a Digital Twin based solution consisting of two parts: attack detection and attack classification. The intrusion detection mechanism uses a combination of a Kalman filter is used  to estimate the correct signals of the system and remove the destructive effects of attacks and noises, which helps detect the occurrence of attacks. Support Vector Machine (SVM) is then used for the classification of the system's state as Normal, Scaling attack or Ramp attack. The proposed anomaly detection algorithm is evaluated through Matlab simulation.

Akbarian et al.\cite{akbarianSecurityFrameworkDigital2021} propose a similar security framework to prior work\cite{akbarianIntrusionDetectionDigital2020} for industrial control systems (ICS) to address the vulnerability of these systems to cyber attacks, particularly when controlled over the cloud. Like their prior work, their proposed framework consists of two parts: attack detection and attack mitigation. The detection part is an intrusion detection system that is deployed in the digital domain, which can detect attacks in a timely manner. To mitigate the effects of attacks, a local controller is added to the factory floor close to the plant. The research paper also evaluates the proposed security framework using a real testbed, which shows that it can detect attacks on a real system in a timely manner and keep the system stable with good performance even during attacks.

A study by Francia et al.\cite{franciaDigitalTwinsIndustrial2021} proposes the use of digital twins in Industrial Control Systems (ICS) to enhance security testing, vulnerability assessment, and penetration testing at low cost and without disrupting operational physical systems. The authors identify key challenges to ICS security, including the convergence of IT and OT, supply chain insecurity, and the difficulty of OT security testing due to operational disruption. The study presents a proof-of-concept system involving a Programmable Logic Controller (PLC)-based bottle-filling system. The authors suggest future directions such as creating additional modular digital twins for various environments, expanding the digital twin testbed for more elaborate ICS integrations and security testing, and automating the process of creating security scenarios for effective utilisation of digital twins in security training and education.

A framework that utilizes Digital Twin as a simulation tool to generate Cyber Threat Intelligence (CTI) which can provide valuable threat information for organizations to improve their security posture is presented in this study\cite{dietzHarnessingDigitalTwin2022}. By combining a general CTI process with digital twin security simulation capabilities, the authors demonstrate the successive steps using the STIX2.1 standard and provide utility tools to assist the CTI generation process. They also conduct an attack simulation with a prototypical digital twin application to evaluate the framework and provide tool-based guidance on the CTI process steps. The experimental results show that STIX2.1 CTI report can be systematically constructed and customised according to the use case. 



\subsubsection{Miscellaneous Use Cases } 
% Todo:  what we are discussing in this section. 
This section discusses some of the miscellaneous use cases of Digital Twin in various fields such as Smart Home, 5G Network, IIoT Network, Smart Factory, and Drone Network. These use cases show how Digital Twin can be applied to enhance the security, privacy, and efficiency of different systems. From enhancing security for smart homes to predicting attacks in drone networks.
  

\textbf{\textit{5G Network}}: Vakaruk et al.\cite{vakarukDigitalTwinNetwork2021} discuss the need to train cybersecurity experts in preventing network attacks in the mission-critical industrial environment. They detailed the integration of machine learning (ML) tools into the SPIDER cyber range platform, which is a cybersecurity training system for experts in next-generation network cybersecurity. The SPIDER platform uses a highly virtualized synthetic traffic generation environment called Mouseworld to inject realistic traffic into a 5G network infrastructure, including attack activity within it. The Smart Traffic Analyzer (STA) component within the Mouseworld is used to train ML modules that can be utilised later as components of the SPIDER platform for detecting attacks in the injected traffic. The platform also applies Generative Adversarial Networks (GANs) to generate synthetic network traffic data (attacks and well-behaved connections) that reproduce the statistical distribution of real traffic. 

\textbf{\textit{IIoT Netwrok}}: To improve communication security and data privacy for DT powered Industrial Internet of Things (IIoT) network, Kumar et al.\cite{kumarBlockchainDeepLearning2022} introduces a framework that combines blockchain and deep learning. A New DT model is presented that  can simulate and replicate security-critical processes in a virtual environment, along with a blockchain-based data transmission scheme that uses smart contracts to ensure data integrity and authenticity. They also presents a Deep Learning scheme that utilizes Long Short Term Memory-Sparse AutoEncoder (LSTMSAE) technique to extract spatial-temporal representation and Multi-Head Self-Attention (MHSA)-based Bidirectional Gated Recurrent Unit (BiGRU) algorithm to detect attacks. The practical implementation of the framework demonstrates a significant enhancement in communication security and data privacy for the DT empowered IIoT network.



\textbf{\textit{Drone Network}}: With the aim of improving security of the CPS drone network Wu et al.\cite{wuDeepLearningDriven2022} study the use of Digital Twin as a simulation aid with deep learning.  
The author presents an attack prediction model  using improved long short-term memory (LSTM) networks and differential privacy frequent subgraph (DPFS) to ensure data privacy. The constructed model is simulated using the Tennessee Eastman process, and the results show that it has higher prediction accuracy and better robustness compared to other models. The Digital Twin technology is utilised to map the operating environment of the drone in the physical space, fully analyse the information security issues of the drone system in the virtual space, and detect multiple attacks and intrusions. However, the study has limitations due to the fact only 3 types of attacks(FDIA, replay attacks and DoS) took into consideration. in addition, only temperature sensor is under attack. Other factors like location, time and intensity of drone system are not considered for an attack.  




\textbf{\textit{Automotive industry}}: A framework called Trusted Twins for Securing Cyber-Physical Systems (TTS-CPS) that utilizes blockchain-based Digital Twins (DTs) to strengthen the security of Cyber-Physical Systems (CPSs) is presented by Suhail et al.\cite{suhailSituationalAwareCyberphysical2022}. The aim of TTS-CPS framework is to ensure the trustworthiness of data generated based on DT specification through Integrity Checking Mechanisms (ICMs). The authors argue that the framework helps to establish more understanding and confidence in the decisions made by underlying systems through storing and retrieving Safety and Security (S\&S) rules from the blockchain. In the paper, the authors demonstrate the feasibility of the TTS-CPS framework in an assembly line of automotive industry through a prototypical implementation supporting simulated network topology, Programmable Logic Controllers (PLCs), Human Machine Interfaces (HMIs), and physical devices. 

\textbf{\textit{Agriculture}}: In\cite{chukkapalliCyberPhysicalSystemSecurity2021} Chukkapalli et al. introduce a security surveillance system for a smart farm that keeps track of the data generated by sensors and alerts the farm owners. The system includes the collected sensor data, a smart farm ontology for creating knowledge graphs, and digital twin modules for anomaly detection. The researchers first use the collected data to generate knowledge graphs with the smart farm ontology and then use the digital twin to train the anomaly detection model using Principal Component Analysis. The authors demonstrate that the digital twin-based anomaly detection model can be used to detect various anomalies in the smart farm.
