% ======================================================================================================
% NOTES, TODOS
% Modeling and simulation 
% AI and Machine learning
%     Deep learning
% Data Analytic
% Big data 
% Cloud and Edge Computing
% Block Chain

% ======================================================================================================
%
\subsection{Enabling Technologies That Power Digital Twin}
Digital Twin can be characterised as a warehouse of data and a virtual representation model of a real-world object with enabling or augmenting technologies to process the data. Its true power lies primarily due to the utilization of enabling technologies\cite{sousaELEGANTSecurityCritical2021}. The use of machine learning, blockchain, cloud computing, and big data analytics has enhanced the capabilities of digital twins, leading to better decision-making processes and improved security. In this section, we will examine the primary technologies that have been explored in the literature for enhancing the capabilities of Digital Twins to provide security services.

\subsubsection{AI and Machine Learning}
From our review, we observe Machine learning (ML) techniques are the most extensively studied enabling technology integrated with Digital Twin. 
In this section, we will examine several research contributions that demonstrate the potential of using ML with digital twin technology to detect and prevent security threats in various domains, such as power systems, industrial control systems, and network security. These contributions include integrating ML tools into cybersecurity training, using deep learning techniques for power system fault detection and drone security, employing generative adversarial networks (GANs) for the generation of synthetic flow-based network traffic, and using machine learning techniques for intrusion detection in industrial control systems.
  
One of the contributions discussed in a paper by Vakaruk et al.\cite{vakarukDigitalTwinNetwork2021} is the integration of machine learning (ML) techniques into cybersecurity training. The authors propose integration of ML to enhance the training processes of the SPIDER cyber range, a playground for cybersecurity training. The ML models classify input traffic as either malicious or benign and provide a confidence level for the prediction, which is reported to a dashboard for inspection and analysis by a human user.

Incorporating deep learning techniques into Digital Twin systems for power system fault detection is another contribution. Danilczyk et al.\cite{danilczykSmartGridAnomaly2021} utilized a Convolutional Neural Network (CNN) module for detecting power system faults and found it to be highly effective, with 95\% accuracy. This was better than other ML methods like MLP and LSTM RNN. Similarly, Wu et al.\cite{wuDeepLearningDriven2022} used ML for drone security, deploying with Digital Twin to address security issues and detect different attacks and intrusions. The proposed model was tested and compared against existing models, and it was found to have higher accuracy as evidenced by a low root mean square error. 

In a study by Rebecchi et al.\cite{rebecchiDigitalTwin5G2022}, the application of Generative Adversarial Networks (GANs) was proposed for the generation of synthetic flow-based network traffic that can imitate both normal and malicious activities. The generated data serves as training input for ML models, which are used by "Blue Teams" to detect security threats, such as crypto mining attacks.

The use of ML algorithms in the simulation and optimization environment of a smart grid is another contribution discussed by Atalay et al.\cite{atalayDigitalTwinsApproach2020} The authors suggest ML algorithms for the classification of abnormal behaviour during the system's lifecycle.

In a paper by Salim et al.\cite{salimBlockchainEnabledSecureDigital2022} deep learning techniques were used to analyze packet headers for detecting botnet activities that are associated with external IP addresses. The capabilities of deep learning algorithms to identify suspicious activities and malicious nodes within a network allow digital twins to accurately monitor and detect malicious behaviour.

In a study by Sousa et al.\cite{sousaELEGANTSecurityCritical2021}, ML was used for security analysis with the primary objective of detecting anomalous activity related to Denial of Service (DoS) and Distributed Denial of Service (DDoS) attacks. The developed model examines incoming traffic and categorizes it as either normal or malicious traffic.

In Li et al.'s\cite{jiaqiliSpaceSpiderHyper2022} work, AI was used as part of advanced technologies, including big data, to provide a better overall educational experience for those participating in the competition and training. The goal was to improve the evaluation and enhance the training effect.

Finally, Varghese et al.\cite{vargheseDigitalTwinbasedIntrusion2022} used machine learning techniques for intrusion detection in an industrial control system (ICS). The study used eight supervised ML-based Intrusion Detection System (IDS) techniques and evaluated their performance using a labelled dataset. The performance of the algorithms was evaluated based on accuracy, precision, recall, and F1-score. The study also used an ensemble approach, stacking multiple classifiers, to design a signature-based IDS. Similarly, In a paper by Akbarian et al.\cite{akbarianIntrusionDetectionDigital2020}, a digital twin-based intrusion detection technique using machine learning was proposed. The attack classification approach used was Support Vector Machine (SVM), which can handle both binary and multi-class classification. The intrusion detection mechanism involves a combination of a Kalman filter for attack detection, a particle swarm optimization algorithm for noise estimation, and an SVM algorithm for attack classification.


 

\subsection{Blockchain and Smart Contract}
The integration of Digital Twin and Blockchain is also the most extensively studied area next to Machine learning. Research explores the benefit of blockchain to provide secure and uncorrupted data that is shared between various stakeholders. In this section, we compile three studies that use blockchain to improve security and accountability in the industrial internet of things (IIoT) and smart grid environments.  


 Kumar et.al\cite{kumarBlockchainDeepLearning2022} proposed a digital twin in IIoT network that includes various layers, such as IoT Device Layer, Edge-Blockchain, and Cloud-Blockchain Layer. The primary aim of blockchain is to ensure the integrity and authenticity of data transmitted over public communication channels by leveraging smart contracts. The study employed the Ethereum test network for scalability analysis and incorporated IPFS off-chain storage system to securely store encrypted IIoT transactions. Salim et.al\cite{salimBlockchainEnabledSecureDigital2022} explored the use of blockchain technology in securing IIoT environments against cyberattacks, particularly botnets. They proposed a Blockchain-enabled Digital Twin Framework that synchronizes and authenticates data with their corresponding IoT devices using a private blockchain network, leveraging smart contracts to authenticate the communication between the digital twins and packet auditors. Lopez et.al\cite{lopezDIGITALTWINSINTELLIGENT2021} focused on the use of blockchain technology in the smart grid to enhance the security and accountability of data. Blockchain solutions were proposed to synchronize information between partners securely, ensuring data ownership and traceability. Overall, the integration of blockchain technology with digital twin systems has shown potential for enhancing data security and authenticity in IIoT environments and the smart grid.

\subsubsection{Cloud and Edge Computing}
Researchers are exploring different ways to deploy digital twins to improve performance and reduce costs. In this section, we briefly discuss various studies that have explored the deployment of digital twin systems using cloud and edge computing. 

In one study \cite{akbarianSecurityFrameworkDigital2021}, a digital twin is deployed in the cloud to take advantage of its unlimited computational and storage resources. An intrusion detection system is placed near the controller in the cloud, allowing it to monitor the signals sent to the controller and assess its health. Another study \cite{grasselliIndustrialNetworkDigital2022} describes how virtual components of the digital twin can be deployed as Virtual Network Functions (VNFs) on Virtual Machines (VMs) to replicate elements of the real infrastructure such as firewalls, intrusion detection/prevention systems, and industrial application gateways. A cloud infrastructure managed with OpenStack and OSM provides dynamic control over the compute, storage, and network resources needed for the instantiation of the digital twin. Gupta et.al. \cite{guptaHierarchicalFederatedLearning2021} propose deploying digital twins on edge devices to reduce the gap between physical objects and their digital representations hosted on cloud servers. They design a Federated Learning (FL) based Anomaly Detection (AD) model and deploy it on an edge cloudlet associated with a patient's digital twin. The use of edge cloudlet computing enhances data privacy and improves model performance. In another study \cite{saadImplementationIoTBasedDigital2020}, a digital twin solution is introduced that is composed of two parts, with the second part built as a function on the cloud. The deployment of digital twins on cloud and edge devices offers different benefits and researchers are exploring different approaches to take advantage of these benefits.

% ===========================================================================
\subsubsection{Big Data and Data Analytics}
By incorporating data analytics techniques, digital twins can provide powerful insights\cite{jiaqiliSpaceSpiderHyper2022} into the behaviour and performance of physical systems. In this section, we provide a review of studies that leverage big data processing and analytics to augment Digital twins.  
 

Salvi et al.\cite{salviCyberresilienceCriticalCyber2022} introduced a conceptual model that integrates data analytics and causal analysis to detect potential attacks on critical cyber infrastructures in real-time. Wang etal.\cite{wangDTCPNDigitalTwin2022}, proposed a platform that integrates big data processing capabilities to efficiently obtain, store, and search data of specific objects, which can be used for simulation and network security analysis. Hussaini et al.\cite{hussainiTaxonomySecurityDefense2022}  discussed how machine learning and deep learning-based data analytics can strengthen digital twins' monitoring capabilities against various attacks. Li et al.\cite{jiaqiliSpaceSpiderHyper2022} used big data and artificial intelligence to process and analyze large amounts of data to provide deeper insights into network security scenarios. Dietz et al. \cite{dietzIntegratingDigitalTwin2020} presented security analytics tools for processing log data for security incident detection in digital twin simulations, while Almeabed et al.\cite{almeaibedDigitalTwinAnalysis2021}  developed a framework based on data processing and analytics capability for vehicular digital twins. These studies illustrate the potential of big data processing and analytics to improve the capabilities and security of digital twins, allowing for more effective modeling and analysis of real-world systems.
