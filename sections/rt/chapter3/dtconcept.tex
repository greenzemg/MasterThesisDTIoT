% ======================================================================================================
% NOTES, TODOS
% Explain how digital twins evolve 
% Why it was incepted in the first place
% What is the potential gain of digital twins...monitoring, optimization..etc
% Prepare a table that shows the definition of DT and the corresponding paper.
% ======================================================================================================
%
\subsection{Exploring the Concept of Digital Twin}

The concept and definition of digital twins have been subject to various interpretations by researchers and scholars, contingent upon the specific context. Nevertheless, the essential components of a digital twin can be generally characterized by 3 main components: two states(physical and digital), inter-connectivity (the channel between the two states), and process( a mechanism for collecting and examining data). In this section, we endeavor to provide a comprehensive definition of digital twins by synthesizing a definition derived from a  review of 30 research papers on the topic. The definition and respective reference are listed in Table \ref{tbl:dtconcept}

Here we give a complete definition of digital twins from a collection of various definitions of research publications: \textit{ Digital Twin is a virtual representation of a physical object or system that mirrors its real-world counterpart through real-time updates and tracking of its entire life-cycle. It is designed to model the physical characteristics and behaviors of the object using digital technology, mapping the physical operating environment to virtual space for interaction and providing valuable insights through collecting asset-centric data, analytic capabilities, and simulations. Digital twins are used for monitoring, simulating, optimizing, and predicting the state of the physical object. They have a standard structure, end-to-end connectivity, communication protocol with backward compatibility, and standard data format for communication between the twins.}

\begin{table}[H]
\scriptsize
\centering
\caption{\label{tbl:dtconcept} Definition of digital twin in the literature}
\begin{NiceTabular}{p{10cm}|p{4cm}}
\CodeBefore
% \rowcolors[gray]{2}{0.8}{}[cols=1-2,restart]
\Body
\toprule
    \textbf{DT definition} & \textbf{Reference(s)} \\
    \midrule
     Digital twins are virtual representations of industrial assets that provide valuable insights through collecting asset-centric data, analytic capabilities and simulations & \cite{dietzIntegratingDigitalTwin2020, eckhartEnhancingCyberSituational2019} \\  
     \hline
    A virtual representation of a physical system, process, or product that is synchronized with its real-world counterpart & \cite{gehrmann_digital_2020, rebecchiDigitalTwin5G2022} \\ 
    \hline
    It is used as a virtual representation of a physical entity, modelling its components and properties & \cite{vakarukDigitalTwinNetwork2021} \\
    \hline
    A system that continuously monitors the physical state of an environment through wide sensor arrays and compares it to simulation models to gain deeper insights into its operating condition & \cite{williamdanilczykANGELIntelligentDigital2019, xuGametheoreticApproachSecure2020, danilczykSmartGridAnomaly2021, veledarDigitalTwinsDependability2019, kumarBlockchainDeepLearning2022, hadarCyberDigitalTwin2020} \\
    \hline
    A virtual representation of a physical system, process or product that is synchronized with its real-world counterpart & \cite{gehrmann_digital_2020, luongnguyenDigitalTwinIoT2022, lopezDIGITALTWINSINTELLIGENT2021} \\ 
    \hline
    A technology to map the physical operating environment to virtual space for interaction. & \cite{wuDeepLearningDriven2022}  \\ 
    \hline
    Evolving digital profile of the historical and current value of physical object or process. & \cite{becueCyberFactorySecuringIndustry40with2018} \\
    \hline

    Virtual representation of physical objects or systems that can be used to monitor and control the real-world counterparts & \cite{almeaibedDigitalTwinAnalysis2021, chukkapalliCyberPhysicalSystemSecurity2021, dietzEmployingDigitalTwins2022}\\
    \hline
    virtual replica of physical object with standard structure, end-to-end connectivity, communication protocol with backward compatibility, and standard data format for communication between the twins & \cite{atalayDigitalTwinsApproach2020} \\

    \hline
    DT is a mapping between physical object and virtual entity that receive data in real-time to predicate the state of the physical object & \cite{dinglingsuzehuiquDetectionDDoSAttacks2022} \\
    
    \hline
    A virtual Model designed to accurately map a physical object or process & \cite{wangDTCPNDigitalTwin2022, sousaELEGANTSecurityCritical2021} \\
    
    \hline
    a method to describe and model the physical characteristics and behaviors of physical objects by using digital technology & \cite{wangSoCbasedDigitalTwin2020} \\
    
    \hline
    A virtual space for representation of real world object and an information flow to keep them synchronize  & \cite{giovannipaolosellittoEnablingZeroTrust2021}\\
    
    \hline
    A digital twin is a virtual representation of a physical object that tracks and mimics its entire life-cycle through real-time updates & \cite{vargheseDigitalTwinbasedIntrusion2022, dietzUnleashingDigitalTwin2020} \\
    
    \hline
    Digital Twin is a virtual replica of physical system that precisely mirror the internal behavior of system for monitoring, simulating, optimizing and predicating the state of the system & \cite{akbarianSecurityFrameworkDigital2021, akbarianIntrusionDetectionDigital2020} \\
    
    \hline
    a digital twin is defined as an integrated system that combines computational, communication and physical aspects of Cyber Critical Infrastructures (CCIs) to provide increased cyber situational awareness & \cite{salviCyberresilienceCriticalCyber2022, pirbhulalNovelFrameworkReinforcing2022} \\

    \hline
    
    
\bottomrule
\end{NiceTabular}
\end{table}