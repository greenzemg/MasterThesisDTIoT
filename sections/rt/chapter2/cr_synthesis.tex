% ======================================================================================================
% NOTES, TODOS
% ======================================================================================================
\subsection{Data Synthesis and Analysis }

\label{Chapter2} % For referencing the chapter elsewhere, use \ref{Chapter1} 
In conducting this literature review, our primary objective is to answer two research questions related to the use of Digital Twin to enhance security in (I)IoT applications for Industry 4.0. The first research question aimed to identify existing solutions that leverage Digital Twin to improve security in the context of (I)IoT applications, while the second question sought to identify the specific mechanisms used to secure communication between Digital Twin and IoT sensors.

In this paper, we followed the three-phase approach outlined by Kitchenham and Charter \cite{kitchenham_guidelines_2007} to perform a systematic review of the literature. We leveraged automating tools, including Parsif.al for designing our review protocol, VOSviewer for conducting bibliometric analyses, and Logseq for data collection and to facilitate the review process.

We began by searching six different digital libraries and collecting a total of 725 papers. We then applied inclusion and exclusion criteria, as well as a quality assessment phase, to arrive at a final selection of 69 items that we found relevant for this review.

To ensure consistency in our review process, we developed a data extraction form as described in table \ref{tbl:extraction}. This approach allowed us to systematically analyze each of the selected papers and provide valuable insights into the application of Digital Twin technology in Industry 4.0 use cases. Using this method, we were able to address our research questions and provide comprehensive findings.


% ======================================================================================================
% NOTES, TODOS
%
% ======================================================================================================
%
\section{RQ1: How Digital Twin is used to secure the (I)IoT application in Industry 4.0}

The integration of operational technology and IT systems in Industry 4.0 increases the risk of cyber attacks. However, technologies like Digital Twin offer opportunities to improve security\cite{dietzHarnessingDigitalTwin2022}. In fact, the literature review suggests that Digital Twins can be applied to enhance security in IoT applications across multiple domains. The table \ref{tbl:lit-bench} summarizes some of the research that uses Digital Twin to improve the security of (I)IoT applications in various domains. The research papers are analyzed using benchmarks, including the target sector (use case), the purpose of DT, the enabling technology (integrated with DT), the contribution (category of methodology), and the type of study (characteristics of the study). The table demonstrates the wide range of sectors in which Digital Twins can be applied to improve security in industries including satellite, energy, power grid, intelligent transport systems, water, agriculture, the automotive industry, and manufacturing.

The Enabling technologies used in the studies include big data,  AI(machine learning), cloud computing and data analytics, edge computing, blockchain, and NFV, among others. These technologies enable researchers to build DT-based systems to improve security for various use cases.

The contributions of the studies have been classified as frameworks, platforms, algorithms, and architectures. We classify a contribution as a platform if the research presents a tool that can be deployed and used to provide security services like intrusion detection. If the authors provide only a high-level overview of the proposed solution, we classify it as an architecture. On the other hand, if details are presented for each component of the architecture, we classify it as a framework.

Regarding the study types, Table \ref{tbl:lit-bench} shows that there are theoretical studies, case-based and experimental-based studies, and review-type studies. The theoretical studies tend to focus on proposing frameworks and architectures, while the case-based studies and experimental studies tend to evaluate the proposed frameworks and algorithms. 


% \begin{adjustwidth}{-0.5in}{}
% \begin{minipage}{\textwidth}

\begin{table}[H]
\tiny
\centering
\caption{\label{tbl:lit-bench} Overview of References by Use Case, Purpose of Digital Twin, Enabling Technology, Contribution Category, and Study Type.}
% \resizebox{\linewidth}{!}{
\begin{NiceTabular}{p{0.5cm}|p{3cm}|p{2.5cm}|p{2cm}|p{3cm}|p{1.5cm}}
\CodeBefore
% \rowcolors[gray]{2}{0.8}{}[cols=1-2,restart]
\Body
\toprule
    \textbf{Ref}  & \textbf{Use Case}& \textbf{Purpose} & \textbf{Enabling Technology }  & \textbf{Contribution Category} & \textbf{Study type} \\
    \midrule

    \cite{jiaqiliSpaceSpiderHyper2022} & Satellite & Simulation & Bigdata and AI & Platform & - \\
    \hline
     
     \cite{danilczykSmartGridAnomaly2021} & Smart grid & Anomaly detection & Machine Learning & Architecture & Experiment \\
     \hline

    \cite{shitoleRealTimeDigitalTwin2021} & Energy & Testing & - &  Platform & Case-Study \\
    \hline

    \cite{saadImplementationIoTBasedDigital2020} & Power grid & Anomaly detection & Cloud Computing and Data Analytics & Framework and Algorithms & Experiment based \\
    \hline
    
    \cite{akbarianIntrusionDetectionDigital2020} & ICS & Simulating and Testing & Machine Learning & Algorithm & Experiment \\
    \hline

    \cite{dietzHarnessingDigitalTwin2022} & ICS & Simulation & - & Framework and Algorithms & Experiment \\
    \hline

    \cite{eckhartEnhancingCyberSituational2019} & CPS & Monitoring, Incident Handling, Testing & - & Framework & Theoretical \\
    \hline

    \cite{maillet-contozEndtoendSecurityValidation2020} & Water, Agriculture & Simulation and Testing & Data Analytics & Architecture & Case study and Experiment \\
    \hline

    \cite{giovannipaolosellittoEnablingZeroTrust2021} & Smart Grid & device policy enforcement & - & Architecture & Theoretical \\
    \hline

    \cite{dietzEmployingDigitalTwins2022} & ICS & Testing and Security Assessment & - & Framework & Experiment \\
    \hline

    \cite{sousaELEGANTSecurityCritical2021} & - & Intrusion Detection & Machine Learning & Architecture & Experiment \\
    \hline

    \cite{xuEfficientAuthenticationVehicular2021} & Automotive Industry & - & - & Framework and Algorithm & Theoretical \\
    \hline

   \cite{glenandbensonjamesandguptamaanakandsandhuravicatheyEdgeCentricSecure2021} & Intelligent Transportation & Access Control & Edge Computing & Architecture & Case-study \\
   \hline

   \cite{wangDTCPNDigitalTwin2022} & Enterprise Network & Simulation & NFV, Big data processing & Platform & Experiment \\
   \hline

   \cite{franciaDigitalTwinsIndustrial2021} & ICS & Testing, Vulnerability assessment & - & - & Experiment \\
   \hline

   \cite{lopezDIGITALTWINSINTELLIGENT2021} & Smart Grid & Detection & Blockchain & Architecture & Theoretical \\
    \hline
    
   \cite{adrienbacueDigitalTwinsEnhanced2022} & Aerospace & Simulation(Attack) & - & - & Case-study \\
    \hline
    
   \cite{veledarDigitalTwinsDependability2019} & Automotive industry & Predictive analytics & - & Platform & - \\
   \hline

   \cite{holmesDigitalTwinsCyber2021} & - & - & - & - & Review paper \\
   \hline

   \cite{vargheseDigitalTwinbasedIntrusion2022} & ICS & Intrusion Detection & Machine Learning & Framework & Experiment \\
   \hline

   \cite{rajivfaleiroDigitalTwinCybersecurity2022} & - & - & - & - & Review paper \\
   \hline

   \cite{hossenDigitalTwinSelfSecurity2021} & Power Grid & Model & - & Algorithm & Experiment \\
   \hline

   \cite{luongnguyenDigitalTwinIoT2022} & Intelligent Transport System & Testing and Simulating & - & Platform & Case-study and Experiment \\
    \hline

    \cite{almeaibedDigitalTwinAnalysis2021} & Automotive Industry & - & Analytics & Framework & Case-study \\
    \hline

    \cite{becueCyberFactorySecuringIndustry40with2018} & Manufacturing sector & Simulation for Testing and Training & - & - & Theoretical \\ 
    \hline

    \cite{wuDeepLearningDriven2022} & CPS of Drones & Simulation & AI - Deep Learning & Model & Experiment \\
    \hline

    \cite{salviCyberresilienceCriticalCyber2022} & Power Grid & - Situational Awareness & Data Analytics & Model & Theoretical \\
    \hline

    \cite{chukkapalliCyberPhysicalSystemSecurity2021} & Agriculture sector & Anomaly detection & Machine Learning & Framework & Experiment \\
    \hline

    \cite{hadarCyberDigitalTwin2020} & Enterprises & - & Analytics & - & Experiment \\
    \hline

    \cite{kumarBlockchainDeepLearning2022} & IIoT Network & Simulation, Intrusion Detection & Blockchain, Deep Learning & Framework & Experiment \\
    \hline

    \cite{sugumarAssessmentMethodDetecting2019} & Water Treatment Plant & Assessment , Simulation & - & Model & - \\
    \hline

    \cite{williamdanilczykANGELIntelligentDigital2019} & Smart Grid & Data Visualization & - & Framework & Experiment \\
    \hline

    \cite{masiSecuringCriticalInfrastructures2023} & Intelligent Transport Systems & - & - & Architecture & - \\
    \hline

    \cite{kandasamyElectricPowerDigital2022} & Smart Grid & Training & - & Platform & Case-study \\
    \hline

    \cite{hussainiTaxonomySecurityDefense2022} & Smart Industries & Simulation and Intrusion Detection & Architecture & Case-study \\
    \hline

    \cite{akbarianSecurityFrameworkDigital2021} & ICS & Intrusion Detection & Cloud Computing & Framework, Algorithm & Experiment \\
    \hline

    \cite{xuGametheoreticApproachSecure2020} & CPS & Fault detection, Monitoring & - & Framework & Experiment \\
    \hline

    \cite{atalayDigitalTwinsApproach2020} & Smart Grid & Testing & - & Framework & Theoretical \\
    \hline

    \cite{houDigitalTwinRuntime2022} & Satellites and Space & Penetration Testing & - & Framework and Algorithm & Theoretical \\
    \hline

    \cite{vakarukDigitalTwinNetwork2021} & Telecom & Training & Machine Learning & Platform & Theoretical \\
    \hline

    \cite{rebecchiDigitalTwin5G2022} & 5G Network & Simulation - Training and Testing & Machine learning & Architecture & Experiment \\
    \hline

    \cite{gehrmannDigitalTwinBased2020} & ICS &  Data sharing & - & Model, Architecture & Case study \\
    \hline

    \cite{chengzhelaiSPDTSecurePrivacyPreserving2022} & Transportation & - & Cloud & - & - \\
    \hline

    \cite{alessandradebenedictischristiancarmineespositoalessandrasommaAdoptionSecureCyber2022} & CPS & Anomaly detection & - & Framework & Theoretical \\
    \hline

    \cite{olivares-rojasCybersecuritySmartGrid2022} & Power grid & Testing & - & Framework & Theoretical \\
    \hline

    \cite{suhailSituationalAwareCyberphysical2022} & Automotive industry & Testing & Blockchain & Framework & Use-case \\
    \hline

    \cite{dietzUnleashingDigitalTwin2020} &  ICS & Simulation, Testing & - & - & - \\
    \hline
\bottomrule
\end{NiceTabular}
% }
\end{table}



% % ======================================================================================================
% NOTES, TODOS
% Explain how digital twins evolve 
% Why it was incepted in the first place
% What is the potential gain of digital twins...monitoring, optimization..etc
% Prepare a table that shows the definition of DT and the corresponding paper.
% ======================================================================================================
%
\subsection{Exploring the Concept of Digital Twin}

The concept and definition of digital twins have been subject to various interpretations by researchers and scholars, contingent upon the specific context. Nevertheless, the essential components of a digital twin can be generally characterized by 3 main components: two states(physical and digital), inter-connectivity (the channel between the two states), and process( a mechanism for collecting and examining data). In this section, we endeavor to provide a comprehensive definition of digital twins by synthesizing a definition derived from a  review of 30 research papers on the topic. The definition and respective reference are listed in Table \ref{tbl:dtconcept}

Here we give a complete definition of digital twins from a collection of various definitions of research publications: \textit{ Digital Twin is a virtual representation of a physical object or system that mirrors its real-world counterpart through real-time updates and tracking of its entire life-cycle. It is designed to model the physical characteristics and behaviors of the object using digital technology, mapping the physical operating environment to virtual space for interaction and providing valuable insights through collecting asset-centric data, analytic capabilities, and simulations. Digital twins are used for monitoring, simulating, optimizing, and predicting the state of the physical object. They have a standard structure, end-to-end connectivity, communication protocol with backward compatibility, and standard data format for communication between the twins.}

\begin{table}[H]
\scriptsize
\centering
\caption{\label{tbl:dtconcept} Definition of digital twin in the literature}
\begin{NiceTabular}{p{10cm}|p{4cm}}
\CodeBefore
% \rowcolors[gray]{2}{0.8}{}[cols=1-2,restart]
\Body
\toprule
    \textbf{DT definition} & \textbf{Reference(s)} \\
    \midrule
     Digital twins are virtual representations of industrial assets that provide valuable insights through collecting asset-centric data, analytic capabilities and simulations & \cite{dietzIntegratingDigitalTwin2020, eckhartEnhancingCyberSituational2019} \\  
     \hline
    A virtual representation of a physical system, process, or product that is synchronized with its real-world counterpart & \cite{gehrmann_digital_2020, rebecchiDigitalTwin5G2022} \\ 
    \hline
    It is used as a virtual representation of a physical entity, modelling its components and properties & \cite{vakarukDigitalTwinNetwork2021} \\
    \hline
    A system that continuously monitors the physical state of an environment through wide sensor arrays and compares it to simulation models to gain deeper insights into its operating condition & \cite{williamdanilczykANGELIntelligentDigital2019, xuGametheoreticApproachSecure2020, danilczykSmartGridAnomaly2021, veledarDigitalTwinsDependability2019, kumarBlockchainDeepLearning2022, hadarCyberDigitalTwin2020} \\
    \hline
    A virtual representation of a physical system, process or product that is synchronized with its real-world counterpart & \cite{gehrmann_digital_2020, luongnguyenDigitalTwinIoT2022, lopezDIGITALTWINSINTELLIGENT2021} \\ 
    \hline
    A technology to map the physical operating environment to virtual space for interaction. & \cite{wuDeepLearningDriven2022}  \\ 
    \hline
    Evolving digital profile of the historical and current value of physical object or process. & \cite{becueCyberFactorySecuringIndustry40with2018} \\
    \hline

    Virtual representation of physical objects or systems that can be used to monitor and control the real-world counterparts & \cite{almeaibedDigitalTwinAnalysis2021, chukkapalliCyberPhysicalSystemSecurity2021, dietzEmployingDigitalTwins2022}\\
    \hline
    virtual replica of physical object with standard structure, end-to-end connectivity, communication protocol with backward compatibility, and standard data format for communication between the twins & \cite{atalayDigitalTwinsApproach2020} \\

    \hline
    DT is a mapping between physical object and virtual entity that receive data in real-time to predicate the state of the physical object & \cite{dinglingsuzehuiquDetectionDDoSAttacks2022} \\
    
    \hline
    A virtual Model designed to accurately map a physical object or process & \cite{wangDTCPNDigitalTwin2022, sousaELEGANTSecurityCritical2021} \\
    
    \hline
    a method to describe and model the physical characteristics and behaviors of physical objects by using digital technology & \cite{wangSoCbasedDigitalTwin2020} \\
    
    \hline
    A virtual space for representation of real world object and an information flow to keep them synchronize  & \cite{giovannipaolosellittoEnablingZeroTrust2021}\\
    
    \hline
    A digital twin is a virtual representation of a physical object that tracks and mimics its entire life-cycle through real-time updates & \cite{vargheseDigitalTwinbasedIntrusion2022, dietzUnleashingDigitalTwin2020} \\
    
    \hline
    Digital Twin is a virtual replica of physical system that precisely mirror the internal behavior of system for monitoring, simulating, optimizing and predicating the state of the system & \cite{akbarianSecurityFrameworkDigital2021, akbarianIntrusionDetectionDigital2020} \\
    
    \hline
    a digital twin is defined as an integrated system that combines computational, communication and physical aspects of Cyber Critical Infrastructures (CCIs) to provide increased cyber situational awareness & \cite{salviCyberresilienceCriticalCyber2022, pirbhulalNovelFrameworkReinforcing2022} \\

    \hline
    
    
\bottomrule
\end{NiceTabular}
\end{table}


% ======================================================================================================
% NOTES, TODOS
% Telecommunication -> 1, 2
%     5G network 
% Power grid, smart grid Energy sector-> 3, 4, 5
% Transportation -> 1, 2
% Agriculture sector
% Automotive -> 1, 2
% Water treatment -> 1
% Space industry -> 1
% General
%     Industry Control system -> 1, 2
%     Cyber-Physical System -> 1, 2, 3, 
%     IIoT network -> 1
%     Digital Enterprises  -> 1
%     smart manufacturing -> 1

% ======================================================================================================



\subsubsection{DT Use-Case Based on Targeted Industry}

In our study, we classified the papers based on the industry they targeted to identify domain-specific challenges. Our approach to classifying the papers was carefully designed to minimize any potential biases. Our finding revealed that DT based security solution has been widely adopted in several industries ranging from Smart Grid(SG) to Smart Homes(SH) for various purposes. Intrusion detection \cite{akbarianIntrusionDetectionDigital2020}, anomaly detection, training and testing \cite{rebecchiDigitalTwin5G2022,becueCyberFactorySecuringIndustry40with2018}, botnet detection \cite{salimBlockchainEnabledSecureDigital2022}, etc. are among the main functions provided through this digital technology. 


To avoid bias classification, we followed a systematic way of identifying the target industry based on the following condition. 

\begin{itemize}
    \item If the authors explicitly specify the industry for which their proposed solution is targeted, we consider this to be the target industry of the paper.
    \item In cases where the target industry was not explicitly mentioned, we looked at the experiments and use cases presented in the paper to identify the targeted industry.
    \item If we couldn't find the target industry through the previous methods, we searched for any discussions of Industrial Control Systems (ICS) or Cyber-Physical Systems (CPS) in the paper, and if found, we categorised the paper under ICS/CPS.
\end{itemize}


It is important to note that our classification methodology excluded review papers and studies that solely focused on providing security mechanisms for digital communication between DT and (I)IoT. 



Using the above criteria, we found the following main industry sectors illustrated in Figure \ref{fig:use-dt}. \\
% Power grid, Agriculture, health, Smart home, Transportation, Autonomous Vehicle, Water Treatment, Space industry, Smart Factory, Automotive industry, 

\begin{center}
 

\smartdiagram[constellation diagram]{
  DT,
  Power grid, Agriculture, Health, Smart home, Transportation, Autonomous vehicle, Water Treatment, Space industry, Smart Factory, Automotive industry
}
\captionof{figure}{Use Cases Of DT}
\label{fig:use-dt}
\end{center}
\subsubsection*{Power Grid}
In the study by Danilczyk et al.\cite{williamdanilczykANGELIntelligentDigital2019} proposed a framework named "Automatic Network Guardian for Electrical Systems (ANGEL)" which uses real-time data visualization to enhance the security and resiliency of microgrids. The framework models both the cyber and physical layers of the microgrid, allowing it to detect discrepancies between simulated and physical systems under various operating conditions. The framework's two-way coupling between the simulation and the physical system enables it to update and improve its simulations, detect unnatural changes, and evaluate meter data accuracy, thereby improving security. ANGEL can also be equipped with machine learning to have self-healing capabilities that can mitigate component failures and cyber-attacks. While the ANGEL framework is promising, it has limitations, including potential false positives and difficulty detecting some types of malicious attacks. Additionally, the framework is still in development and has not yet been tested on a real-world microgrid system for further evaluation.

Another study by Saad et al.\cite{saadImplementationIoTBasedDigital2020} presents an IoT-based DT (DT) for microgrids that aims to improve their resilience against cyber attacks. The proposed framework is implemented as a cloud-based DT platform that acts as a central hub for the networked microgrid system. The DT is designed to model both the physical and cyber layers of the microgrid, allowing it to detect false data injection (FDIA) and denial of service (DoS) attacks. The framework utilizes observer-based What-If scenarios to take corrective action when an attack is detected, ensuring the safe and seamless operation of the networked microgrids. The proposed DT framework is validated using a practical setup of the distributed control system and Amazon Web Services (AWS), and is able to quickly detect and mitigate a range of cyber attacks. The authors argue may the fusion of deep learning and Luenberger Observer(LO) enhances the speed, accuracy, and predictability of attacks. In general, the proposed IoT-based DT framework presents a practical solution to improve the resilience of microgrids against cyber attacks.

In\cite{hossenDigitalTwinSelfSecurity2021} this paper, Hossen et al. propose a knowledge-based self-security algorithm that evaluates the incoming power setpoints for safety before implementation. The algorithm utilizes the steady-state and dynamic behaviours of the inverter, which were experimentally determined using laboratory equipment to create a DT of the inverter. The study demonstrates that this technique can help protect smart grids from man-in-the-middle attacks by thoroughly examining incoming commands via the DT before engaging them in the local controller.

The study undertaken by Atalay et al.\cite{atalayDigitalTwinsApproach2020} focuses on providing an overview of smart grid cybersecurity standards and reviews major threats to smart grid environments at the physical, network, and application layers. In this study, the authors argue that despite the prevalence of smart grids in energy distribution networks, there is a lack of standards for comprehensive security assessment, which is a critical shortcoming. With the aim to address this gap, the authors propose a Digital Twins based approach for the security testing lifecycle of smart grids, by accurately modelling the functioning of the physical grid and running security testing on the actual grid without causing disruption. This approach has the potential to become an important tool for standardisation. While the paper presents an innovative framework for security testing, it lacks experimental validation and implementation details for real-event scenarios.


In their study, Sellitto et al.\cite{giovannipaolosellittoEnablingZeroTrust2021} proposed a methodology to build a cybersecurity DT of a Smart Grid based on its architectural blueprint. The goal of the methodology is to enable the adoption of Zero Trust Architecture (ZTA) and dynamic enforcement of security policies for devices connected to the grid. The authors presented a novel approach to dynamically align the DT with its real-world counterpart, creating a maintenance-aware model for the Smart Grid. This was achieved by adopting an architectural view that gets dynamically aligned with the state of the real-world counterpart during deployment and operation time. The authors laid the foundation for a DT model that allows dynamic enforcement of security policies that reflect Smart Grid topology changes over time. 


Salvi et al.\cite{salviCyberresilienceCriticalCyber2022} targets the electrical energy sector with the aim of increasing the cyber-resilience of Critical Control Infrastructures (CCIs) using a DT implementation to address risks associated with the integration of computational, communication, and physical aspects of CCIs. It seeks to provide increased situational awareness, a common understanding of incidents, and enhanced response capacity to minimize response time and reduce the impact of cyber-attacks on organizations and society. However, the study is limited by the fact that it only focuses on the conceptual model, rather than the implementation of the DT, which may require further validation through proof of concepts in different CCI contexts. Nevertheless, this research addresses the needs expressed by key stakeholders in the electrical energy sector and presents design principles that can be applied in disaster management contexts.


A study by Danilczyk et al.\cite{danilczykSmartGridAnomaly2021} presents a deep learning convolutional neural network (CNN) as a module within the Automatic Network Guardian for Electrical Systems (ANGEL) DT environment to detect physical faults in a power system. The approach uses high-fidelity measurement data from the IEEE 9-bus and IEEE 39-bus benchmark power systems to detect if there is a fault in the power system and to classify which bus contains the fault. The anomaly detection CNN algorithm was able to identify the existence of a fault with near-perfect accuracy and classify the location of the fault with an accuracy of nearly 95\% for both systems. The long-term goal of this project is to have the DT with the anomaly detection CNN running alongside the physical smart grid. However, the study's limitation is that, due to the small timescales present in power systems, the inference speed of the network will be of critical importance. For real-time implementation, more powerful hardware would be beneficial to the overall performance of the integrated system. Despite this limitation, deep learning algorithms show significant promise in the detection and location of power system faults and can improve performance and reduce the cost of power distribution.

To overcome limitations in security studies of Smart Grids (SG) in physical test-beds Kandasmy et al.\cite{kandasamyElectricPowerDigital2022} build a digital power twin that enables the deployment of real-world attacks and countermeasures while allowing easy modification of components and configurations. The tool presented by the author named EPICTWIN, a DT for a power physical test-bed, allows users to validate the security and safe operation of critical components in a more realistic environment, reducing the gap between physical and simulated test-bed environments. They claim their tool provides an attacker designer(AD) and attack launcher(AL), unique tools that enable researchers to validate and improve defence mechanisms even without expertise in offensive security testing. Finally, the authors highlight the uniqueness of their contributions in building a DT of an existing cyber-security test-bed, presenting a procedure that can be extended to any type of system, and providing unique tools for launching systematic attacks on the twin.

\subsubsection*{Smart Factory}
Lopez et al.\cite{lopezDIGITALTWINSINTELLIGENT2021} aim in their research to analyze the evolution of digital twins in smart grid infrastructures and their role in implementing intelligent authorization policies. The authors study the application of AI technologies, including machine learning and blockchain, in the context of digital twins to manage dynamic information flows and detect cybersecurity issues in real time. They provide a mid-term and long-term analysis of the pending challenges of DTs and discuss the three-stage process of DT evolution, starting from monitoring systems with limited analysis capabilities to fully semantic, self-learning platforms. The contribution of this article lies in the analysis of the future smart grid through the evolution of digital twins, pointing out the most relevant challenges they face. The authors conclude that digital twins will play a fundamental role in driving the progress of the electricity grid toward a fully decentralized and autonomous model, governed by intelligent authorization systems. However, standardization and information security efforts are necessary, along with deep research into machine learning specifically applied to critical infrastructures and smart cities.


 In\cite{shitoleRealTimeDigitalTwin2021} paper presented by Shitole et al. aims to develop a low-cost Real-Time DT (RTDT) of an interconnected and distributed Residential Energy Storage System (RESS) controlled and monitored via Cloud based Energy Management System (CEMS), in order to analyse the cyber-security of such systems and develop appropriate Intrusion Detection Systems against cyber attacks. The proposed RTDT allows for flexibility in modifying, scaling, and replicating the system without compromising its real-time fidelity. The development procedure can be easily replicated to develop RTDT of any Cyber-Physical System (CPS) or micro-grid test-beds. The paper presents a systematic procedure for the development of the RTDT and verifies its performance through an experimental case study. The RTDT is developed using a low-cost single board computer with Simulink Desktop Real-Time, which reduces overall development cost. Overall, this paper presents a reliable and economical solution for cyber security studies on RESS through the development of an RTDT.


 Salim et al.\cite{salimBlockchainEnabledSecureDigital2022} propose a secure blockchain-enabled digital framework for the early detection of botnet formation in a smart factory environment. The proposed framework integrates a DT (DT), a packet auditor (PA), deep learning models, blockchain, and smart contracts(SC) for securing the data flow of a smart factory environment. The DT is designed to collect device data and inspect packet headers for connections with external unique IP addresses with open connections. Data are synchronized between the DT and the PA for detecting corrupt device data transmission, and smart contracts authenticate the DT and PA to ensure malicious nodes do not participate in data synchronization. Botnet spread is prevented using DT certificate revocation. A comparative analysis with existing research shows that the proposed framework provides data security, integrity, privacy, device availability, and non-repudiation.


In \cite{becueCyberFactorySecuringIndustry40with2018} paper by Bécue et al. discuss ITEA initiative CyberFactory\#1 project, which aims to develop a system of systems to optimize and ensure the resilience of digital factories and factories of the future (FoF) in the face of increasing digitization and connectivity. The project focusses on optimising the efficiency and security of the network of factories, proposing novel architectures and methodologies to address cyber and physical threats and safety concerns. It also integrates technical, economic, human, and societal dimensions. In this study Digital twins is used to support cybersecurity testing and training, together with cyber ranges, to enable risk anticipation and accurate impact prediction. The project demonstrates key capabilities in realistic environments and reflect the variety of possible new factory types and business model shifts.



\subsubsection*{Health}
An automated framework for improving cybersecurity in IoT-based healthcare applications using DT that includes innovative healthcare security techniques such as system modelling, traffic and attack generation, impact assessment, attack and response strategies, and cyber-attack prevention processes proposed by Pirbhulal et al.\cite{pirbhulalNovelFrameworkReinforcing2022} The authors investigate the applicability of DT for cyber-attacks prevention and present a strategic procedure for enhancing cybersecurity. The proposed framework can help update access control policies and enhance cybersecurity, and it provides an automated cybersecurity solution by incorporating system models and resolving known vulnerabilities and threats. However, the limitation of this research is that it is a theoretical study and needs to be validated through experiments and simulations. The authors conclude that DT is a valuable tool for enhancing cybersecurity in healthcare systems, as it provides analysis, design, and optimisation of systems to improve accuracy, speed, and effectiveness, and it can simulate security breaches and develop decision-making and mitigative responses to simulated cyberattacks. 

\subsubsection*{Smart Home}
% The design and implementation of a DT system for smart metering systems (SMS) in a smart home setting are presented in the work by Olivares-Rajos et al.\cite{olivares-rojasCybersecuritySmartGrid2022}. The system is mainly composed of a DT framework controller (DTFC) responsible for mapping real objects with its virtual representation, exchange values between DTs, notify events and alarms between all the objects, and simulating cyber threats and attacks. The SMS is mainly composed of an Smart Meter(SM) in the premises of the end-users, data concentrator (DC) that collect the SM consumption/production data, and metering database management system (MDMS) server that store all the information collected by DC. The authors conclude that the use of digital twins (DTs) is feasible in various contexts of the smart grid, particularly in cybersecurity testing.

In\cite{xiaoCommandFenceNovelDigitalTwinBased2022} this paper,  Xiao et al. propose a novel digital-twin-based security framework, CommandFence, to protect smart home systems from malicious and benign apps with design flaws or logical errors that may cause harm to the user when executed. The framework uses an Interposition Layer to interpose app commands and an Emulation Layer to execute these commands in a virtual smart home environment and predict whether they can cause any risky smart home state when correlating with human activities and environmental changes. If a sequence of app commands can potentially lead to a risky consequence, they are treated as dangerous, and the framework drops them before any insecure situation can occur. The authors fully implemented the CommandFence framework and tested it on 553 official SmartApps on the Samsung SmartThings platform, 10 malicious SmartApps created by Jia et al., and 17 benign SmartApps with logic errors developed by Celik et al. The experiment successfully identified 34 potentially dangerous SmartApps out of 553 official SmartApps, 7 out of 10 malicious SmartApps, and achieved 100\% accuracy for the 17 benign SmartApps with logic errors.CommandFence is orthogonal to the well-received permission-based access control mechanisms and can be implemented as plug-in software without any hardware upgrades. 

\subsubsection*{Transportation}

Cathey et al.\cite{glenandbensonjamesandguptamaanakandsandhuravicatheyEdgeCentricSecure2021} presented a novel edge-centric access control architecture for IoT environments using techniques called Tag Based Access Control(TBAC), which utilises digital twins to separate data based on tags assigned on the fly, limiting access to authorised users and applications. The proposed architecture is lightweight, supports low-latency and real-time security mechanisms, and improves system security and efficiency by minimising data sharing and granting individual access to data subsets. The paper demonstrates the usefulness of TBAC in smart environments such as manufacturing and internet-connected vehicles.

A DT based tool called Testing and Simulation(TaS) presented in\cite{luongnguyenDigitalTwinIoT2022} paper by Nguyen et al. for testing and simulating IoT environments in order to improve testing methodologies and evaluate the possible impact of IoT systems on the physical world. The tool supports functional and nonfunctional testing and can be used to detect and predict failures in evolving IoT environments. The tool has been tested and validated through experiments performed in the context of the H2020 ENACT project. The contribution of the paper lies in the design of a tool that allows the real-time connection of the physical system to a new software version deployed in the DT, enabling verification that changes made in the code does not impact existing software functionality. The tool has been applied in different domains, showing that it is generic and can be used to achieve different test objectives. Although TaS automates several steps in the test process, the author points to limitations regarding testing scenario generation that could be improved.

The authors in this \cite{aryaDetectionMaliciousNode2023a} work proposed a framework that utilizes Digital Twin (DT) in the context of a Vehicular Ad-hoc Network (VANET) to identify and prevent malicious nodes. They employed machine learning techniques to distinguish between normal and attack traffic. By analyzing incoming packets, the physical Road Side Unit (RSU) parsed IP addresses and compared them against a blacklist. If an IP address matched the blacklist, the packet was considered malicious and discarded. The approach demonstrated a high F-1 score, indicating its effectiveness in detecting malicious nodes in VANET. Thus, the combination of DT, machine learning, and blacklist-based filtering proved valuable for the detection and prevention of malicious nodes in VANET infrastructure.








\subsubsection*{Autonomous Vehicle}
In \cite{almeaibedDigitalTwinAnalysis2021}, Almeaibed et al. proposed a standard framework for the creation of vehicular digital twins that streamlines data collection, processing, and analytics. The authors also highlight the importance of DT security through a case study that showcases how radar sensor readings can be altered by hackers, potentially leading to collisions. The paper concludes by providing insights on the implementation of digital twins in the autonomous vehicle industry and addressing privacy, safety, security, and cyber attack mitigation.

Another research that focuses on the autonomous vehicle to tackle safety and security issues in connected cars and Autonomous Driving is presented by Veledar et al.\cite{veledarDigitalTwinsDependability2019}.
With the scope of IoT4CPS, a guideline for secure integration of IoT into autonomous driving (AD), the authors suggested three main steps for designing Digital Twins to address security vulnerabilities in AD. The proposed three steps are: Firstly, identifying assets, modelling them, and defining security and safety objectives. Secondly, designing security and safety evaluation metrics. Lastly, performing threat modelling and test case demonstrators based on security and safety risk assessment and forecasting.

A study by Marksteiner et al. \cite{marksteinerUsingCyberDigital2021} which is funded by Austrian Research Promotion Agency (FFG) and the ECSEL Joint Undertaking, with support from the European Union's Horizon 2020 program, proposes an automated approach for cybersecurity testing in a black box setting. The methodology combines pattern-matching-based binary analysis, translation mechanisms, and model-checking techniques to generate meaningful attack vectors with minimal prior knowledge of the system being tested. It is designed to meet the security requirements outlined by UNECE regulation R155 for vehicular systems  

Xu et al.\cite{xuEfficientAuthenticationVehicular2021} have introduced a conceptual framework called the Vehicular DT (VDT), designed to aid in the fusion, calculation, and communication of data in autonomous vehicles (AVs). The VDT, which is stored on the cloud, is constantly updated in real-time to match the AV it represents. It can also connect with other digital twins to obtain necessary information. To maintain secure communication between the AV and the DT, the authors propose an authentication protocol that combines the secret handshake scheme and group signature. This protocol provides anonymity for honest members while allowing for traceability if necessary, and also ensures the authenticity of messages sent between the AV and the DT. The result of the performance analysis shows that the authentication protocol had a less computational cost while satisfying necessary security requirements effectively. 

\subsubsection*{Water Treatment}
% In \cite{sugumarAssessmentMethodDetecting2019}, Sugumar et al. examine the use of timed automata models for detecting cyber-attacks on critical infrastructure through a design-centric anomaly detection method. The authors create a DT model that replicates the behavior of real-world systems, such as water treatment plants, and use an attack launcher to test the deployed security method's effectiveness against various attacks, including scaling and pulse attacks. The experiment results demonstrate that the proposed approach can accurately detect cyber-attacks outperforming other methods that involve simulations or direct testing on operational testbeds. However, the authors do not address the potential limitations of their approach, such as scalability when applied to more complex systems with more intricate components. Additionally, they only evaluated a subset of potential attack templates, including scaling and pulse attacks and did not consider other types of attacks, such as random or ramp attacks, which could pose a threat to critical infrastructure systems. 

The authors In\cite{maillet-contozEndtoendSecurityValidation2020}, introduce an approach for the integration, verification, and validation of security in IoT devices. The approach is based on the DT concept and involves creating a comprehensive virtual representation of a physical device, composed of black box and white box models at different abstraction levels. By using this approach, the cost impact of adding security to physical devices is reduced, while still ensuring the security and functionality of the device. This approach provides a new way to think about integrating security in the IoT and has the potential to improve the overall security and efficiency of connected devices. To validate their approach they conducted two use case studies based on H2020 critical infrastructure of water management project.

\subsubsection*{Space Industry}

In \cite{adrienbacueDigitalTwinsEnhanced2022} this research, the authors highlight the utilization of digital twins (DTs) in the aerospace manufacturing industry where the Industrial Internet of Things (IIoT) is being integrated with Airbus Defence and Space factories. They conducted a case study to show how DT based simulation solutions can be used for simulating attacks and designing countermeasures without affecting the internal operation of manufacturing. The study's results demonstrate that DTs can effectively aid the industry in enhancing cybersecurity while adopting connected and collaborative manufacturing techniques.

Hóu et al.\cite{houDigitalTwinRuntime2022} propose a method for improving the capability of detecting cybersecurity issues in satellite communication using run-time verification based on digital twins. This involves monitoring and evaluating software or hardware system against user-defined properties. The proposed method uses state synchronization and encryption for secure communication between the physical twin and DT and incorporates a cryptographic algorithm into their state synchronization protocol to guarantee the correctness of the state. However, the framework has some weaknesses, such as the lack of discussion on the security protocols used for secure communication and the absence of security and performance analysis.

% Hou et.al. \cite{houDigitalTwinRuntime2022} proposes a method that integrates digital twins with runtime verification for the secure monitoring and verification of satellites. The approach employs state synchronization and encryption for secure communication and a Linear Temporal Logic (LTL)-based verification engine.
In \cite{jiaqiliSpaceSpiderHyper2022} Li et al. claim adding contribution by defining characteristic hyper-large scientific infrastructures and evaluation indicators of traditional large scientific infrastructures. Due to security risks facing the space Internet, the paper proposes constructing a hyper-large scientific infrastructure called Space Spider, which simulates the space Internet's entire life cycle and creates a system for space Internet attack and defence. Additionally, the paper introduces Spiderland, an open experimental platform for studying space Internet applications and security.

\subsubsection*{Enterprise Network}
 Wang et al.\cite{wangDTCPNDigitalTwin2022} suggest a DT Cyber Platform based on NFV (DTCPN) to address the challenges in developing large-scale networks, such as complex network management and operation, and high risk and overhead of on-the-fly optimization of product network. The DTCPN combines the advantages of DT and NFV technology to eliminate complex and inaccurate modeling process, support Real-Virtual interaction, and provide high fidelity. The platform is designed to facilitate the design, analysis, testing, and evaluation of network technologies and devices in a rapid, accurate, and efficient way. The article concludes that DTCPN has technical advantages that can play a significant role in network security, network management, and network applications. Further optimization and enrichment of the DTCPN's design and functions are planned for the future.

This\cite{hadarCyberDigitalTwin2020} research paper proposes a novel method for automatically gathering and prioritising security controls requirements (SCRs) for rapid risk reduction in active networks. It introduces a cyber DT, based on attack graph analytics, that associates network information with attack tactics, evaluates the efficiency of implemented SCRs, and automatically detects missing security controls. The paper presents a framework and methodology to construct a contextual cyber DT, ranking the risk impact of security controls, and prioritising SCRs to reduce risk impact as quickly as possible. The paper also provides visualizations of a field experiment conducted via an active network, demonstrating successful results in reducing cyber impact and identifying missing security controls for future implementation. The proposed cyber DT simulator offers several new risk reduction methods for automatically selecting SCRs and can be used as a valuable tool for existing cybersecurity evaluation and future cybersecurity budget proposals.

\subsubsection*{ICS/CPS Environment Use Case}

This\cite{vargheseDigitalTwinbasedIntrusion2022} research from Varghese et al. introduces a DT-based security framework for industrial control systems (ICS) that can simulate attacks and defence mechanisms. Four process-related attack scenarios are tested on an open-source DT model of an industrial filling plant. The study proposes a real-time intrusion detection system based on a stacked ensemble classifier that combines predictions from multiple algorithms. This model outperforms previous methods in terms of accuracy and F1 Score, detecting intrusions in close to real-time (0.1 seconds). The proposed framework extends the capabilities of an existing ICS DT framework with an ML-based IDS module and provides a platform for developing intrusion detection and prevention systems.


In\cite{masiSecuringCriticalInfrastructures2023} Masi et al. discuss the  use of DT (DT) technology to improve the cybersecurity of critical infrastructures. The paper presents a Cybersecurity View that can be derived from an Enterprise Architecture (EA) approach to cybersecurity. This view facilitates the identification of adequate cybersecurity measures for the system while improving the overall system design. The methodology proposed in this paper can be applied to the whole system life-cycle: from design/construction to production/exploration and phaseout. The paper addresses two main challenges: the custom-built nature of Industrial Automation and Control Systems (IACS) and the impedance between the EA models used in industrial automation and the models used in visual threat modeling. To address these challenges, the paper proposes the adoption of a reference architecture framework suitable for IACSs and uses a set of rules to build a cybersecurity view of IACS that is amenable to translation into a visual threat modeling language.  The practical usefulness of the proposed methodology is demonstrated through two real-world use cases: the Cooperative Intelligent Transport System (C-ITS) and the Road tunnel scenario. 

Dietz et al.\cite{dietzEmployingDigitalTwins2022} discusses the security issues of industrial control systems (ICS) and proposes an approach for introducing security-by-design system testing with the help of a DT. The authors argue that proper system testing can reveal the system’s vulnerabilities and provide remedies, and that security measures should be carried out as early as possible, especially to render systems secure-by-design. The authors implement a DT representing a pressure vessel and demonstrate how to carry out each step of their proposed approach, identifying vulnerabilities and showing how an attacker can compromise the system by manipulating values of the pressure vessel with the potential to cause over-pressure, which, in turn, can result in an explosion of the vessel. Overall, the DT presented in this study as a tool for security-by-design system testing in industrial control systems.


In another study, Dietz et al.\cite{dietzUnleashingDigitalTwin2020} discuss the challenges and opportunities presented by Industry 4.0 (I4.0) in relation to industrial security. As traditional operational technology (OT) systems are increasingly integrated with general-purpose IT systems, which creates novel attack vectors in industrial ecosystems, the author argues that I4.0 technologies, such as digital twins (DTs), can contribute to industrial security by providing virtual entities that represent physical industrial systems. They also added DTs offer opportunities for security, such as simulation and replication of system behavior, and can play an important role in mitigating and avoiding risks associated with critical infrastructures. DTs can also provide comprehensive information about the asset's status, history, and maintenance needs, and can support an immediate reaction to security incidents. In conclusion, the author suggests that DTs can be an important tool to strengthen industrial security in the context of I4.0.


To enhance cyber-situation awareness for operators Eckhart et al.\cite{eckhartEnhancingCyberSituational2019} propose a digital-twin cyber situational awareness framework for cyber-physical systems (CPSs). The paper builds upon and extends the previous research on leveraging the digital-twin concept for securing CPSs. The proposed framework provides advanced monitoring, inspection, and testing capabilities that support the operations staff in gaining situation perception, comprehension, and projection. The framework enables real-time visualisation and a repeatable, thorough investigation process on a logic and network level. The technical use cases illustrate the added value of the proposed framework for improving cyber situational awareness regarding CPSs, such as risk assessment, monitoring, and incident handling. However, the paper acknowledges that further development effort is required to improve the visualization of digital twins and to complete the record-and-replay feature. 


Dietz et al.\cite{dietzIntegratingDigitalTwin2020} propose a security framework that leverages DT-based security simulations to enhance Security Operations Center (SOC) and Security Information and Event Management (SIEM) systems in mitigating the expanding attack surface in industrial environments. The authors demonstrate how the framework can simulate attacks, analyze their impact on virtual counterparts, and create technical rules for implementation in SIEM systems. In general, the framework comprises five activities: asset modeling, attack modeling, simulation execution, result analysis, and action implementation. The paper concludes by highlighting the contribution of the proposed framework  to SOC security strategies and suggests future work to evaluate its effectiveness and performance. Additionally, the authors recommend extending the framework to integrate with cyber threat intelligence (CTI) to provide more utility to SOC analysts.

The paper by Grasselli et al.\cite{grasselliIndustrialNetworkDigital2022} presents the implementation of a DT for industrial networks to facilitate cyber-security testing and validation without interfering with the real cyberphysical system. The proposed methodology involves the use of technologies such as Cloud Computing and Network Function Virtualization (NFV) and is supported by the ETSI NFV Management and Orchestration (MANO) framework to automate the deployment of the DT. The authors describe the different steps involved in the lifecycle management of the DT, which include the preparation phase, commissioning phase, operation phase, and de-commissioning phase. The paper also includes a quantitative evaluation of the time needed to perform these actions. Overall, the paper highlights the potential of DT technology in addressing cyber-security concerns in Cyber-Physical Systems.


% In\cite{xuGametheoreticApproachSecure2020} paper, Xu et al. discuss the issue of data-integrity attacks in Cyber-Physical Systems (CPSs), particularly the Sensor-and-Estimation (SE) attack where the attackers tamper with sensing or estimated information of CPSs. The authors propose a framework that uses a Chi-square detector in a DT (DT) to monitor the estimation of the physical system and collect evidence to detect any attack. They also use a Signaling Game with Evidence (SGE) to find the optimal attack and defence strategies. The proposed framework is designed to mitigate the impact of the attack on physical performance and to guarantee the stability of CPSs. Analytical results show that proposed defensive strategies can effectively restrict attackers' ability to carry out stealthy estimate attacks.


Sousa et al.\cite{sousaELEGANTSecurityCritical2021} introduces an off-premises approach to designing and deploying digital twins (DTs) for securing critical infrastructures. The proposed solution involves the use of high-fidelity replicas of Programming Logic Controllers (PLCs), which provide a faithful environment for security analysis and evaluation of potential mitigation strategies. The authors highlight that while on-premises implementation can be costly, DTs offer a reliable option for security analysis and evaluation. However, adapting security and safety monitoring mechanisms to synchronize with the DT replica can be challenging. To address this issue, the paper presents an off-premises approach that uses real-time, high-fidelity emulated replicas of PLCs along with scalable and efficient data collection processes. The approach includes the development and validation of Machine Learning models to mitigate security threats such as Denial of Service (DoS) attacks. The results of the experiments demonstrate that DTs provide a faithful environment for security analysis and evaluation of potential mitigation strategies against high-impact threats such as distributed DoS attacks.

The use of digital twins as security enablers and data sharing for Industrial Automation and Control Systems (IACS) discuss in detail by Gehrmann et al.\cite{gehrmannDigitalTwinBased2020}. The authors identify design-driving security requirements for DT-based data sharing and control and propose a state synchronisation model to meet these requirements. They also evaluate the security and performance of the proposed architecture through a proof-of-concept implementation with a programmable logic controller (PLC) software upgrade case. The paper concludes that a DT-based security architecture can be a promising way to protect IACS while enabling external data sharing and access, but further research is needed to fully implement and evaluate the proposed architecture.

Motivated by the increasing connectivity of Industrial Control Systems(ICS) which makes them more vulnerable to cyber attacks Akbarian et al\cite{akbarianIntrusionDetectionDigital2020} proposes a DT based solution consisting of two parts: attack detection and attack classification. The intrusion detection mechanism uses a combination of a Kalman filter is used  to estimate the correct signals of the system and remove the destructive effects of attacks and noises, which helps detect the occurrence of attacks. Support Vector Machine (SVM) is then used for the classification of the system's state as Normal, Scaling attack or Ramp attack. The proposed anomaly detection algorithm is evaluated through Matlab simulation.

Akbarian et al.\cite{akbarianSecurityFrameworkDigital2021} propose a similar security framework to prior work\cite{akbarianIntrusionDetectionDigital2020} for industrial control systems (ICS) to address the vulnerability of these systems to cyber attacks, particularly when controlled over the cloud. Like their prior work, their proposed framework consists of two parts: attack detection and attack mitigation. The detection part is an intrusion detection system that is deployed in the digital domain, which can detect attacks in a timely manner. To mitigate the effects of attacks, a local controller is added to the factory floor close to the plant. The research paper also evaluates the proposed security framework using a real testbed, which shows that it can detect attacks on a real system in a timely manner and keep the system stable with good performance even during attacks.

A study by Francia et al.\cite{franciaDigitalTwinsIndustrial2021} proposes the use of digital twins in Industrial Control Systems (ICS) to enhance security testing, vulnerability assessment, and penetration testing at low cost and without disrupting operational physical systems. The authors identify key challenges to ICS security, including the convergence of IT and OT, supply chain insecurity, and the difficulty of OT security testing due to operational disruption. The study presents a proof-of-concept system involving a Programmable Logic Controller (PLC)-based bottle-filling system. The authors suggest future directions such as creating additional modular digital twins for various environments, expanding the DT testbed for more elaborate ICS integrations and security testing, and automating the process of creating security scenarios for effective utilisation of digital twins in security training and education.

A framework that utilizes DT as a simulation tool to generate Cyber Threat Intelligence (CTI) which can provide valuable threat information for organizations to improve their security posture is presented in this study\cite{dietzHarnessingDigitalTwin2022}. By combining a general CTI process with DT security simulation capabilities, the authors demonstrate the successive steps using the STIX2.1 standard and provide utility tools to assist the CTI generation process. They also conduct an attack simulation with a prototypical DT application to evaluate the framework and provide tool-based guidance on the CTI process steps. The experimental results show that STIX2.1 CTI report can be systematically constructed and customised according to the use case. 


A paper by Bitton et.al \cite{bittonDerivingCostEffectiveDigital2018a} suggests a method for creating a cost-effective digital twin for Testing ICS environment. The proposed method consists of two modules: a problem builder that takes facts about the system under test and converts them into a rules set that reflects the system's topology and digital twin implementation constraints; and a solver that takes these inputs and uses 0-1 non-linear programming to find an optimal solution (i.e., a digital twin specification), which satisfies all of the constraints. The proposed method maximizes the impact of the digital twin within budgetary limitations by evaluating the number and types of security penetration tests that it supports. The cost of a test is determined by the costs of the participating components (i.e., the direct cost of implementing them in the digital twin), as well as the test's execution costs (e.g., security expert's time/salary). The output of the proposed method specifies the digital twin configuration, i.e., which components of the ICS should be implemented and at which implementation level.

Xu et.al \cite{xuDigitalTwinbasedAnomaly2023a} propose anomaly detection DT based on LATTICE approach, which is an extension of the ATTAIN method proposed in the authors' previous work. LATTICE introduces curriculum learning to optimize the learning paradigm of ATTAIN. It attributes each sample with a difficulty score and feeds it into a training scheduler, which samples batches of training data based on these difficulty scores. This allows the model to learn from easy to difficult data. The authors also use five publicly available datasets collected from five real-world CPS testbeds including water treatment and gas pipeline to evaluate LATTICE and compare it with three baselines and ATTAIN. Additionally, the authors build the digital twin model (DTM) as a timed automaton machine and use GAN as the backbone of the digital twin capability (DTC) to provide ground truth labels to improve the anomaly detection capability of LATTICE.


In \cite{epiphaniouDigitalTwinsCyber2023a} this paper, the authors propose and recommend the use of Digital Twin (DT) to increase the cyber resilience of cyber-physical systems (CPS) of Critical National Infrastructure(CNI). They suggest that DT can be combined with a cyber range to analyze how the system to be engineered behaves under attack. The DT can also run attacks demonstrated against the resilience metrics, which may facilitate the design of the security and safety mechanisms of CPSs. The authors also present a proof-of-concept of holistic cyber resilience testing using DT at the port of Southampton that integrates cyber standards and security descriptors with emerging modeling techniques to represent the impact of cyber attacks and resilience efforts effectively. Therefore, the paper suggests that the integration of cyber modeling and simulation with digital twins and threat source characterization methodologies can lead to a cost-effective security and resilience assessment.

\subsubsection*{Miscellaneous Use Cases } 
% Todo:  what we are discussing in this section. 
This section discusses some of the miscellaneous use cases of DT in various fields such as Smart Home, 5G Network, IIoT Network, Smart Factory, and Drone Network. These use cases show how DT can be applied to enhance the security, privacy, and efficiency of different systems. From enhancing security for smart homes to predicting attacks in drone networks.



\textbf{\textit{5G Network}}: Wang et.al. \cite{wangDigitalTwinNetwork2022a} introduces a proposal for utilizing digital twin technology to establish fundamental security functions and develop an autonomous solution for provisioning security capabilities in 5G network slices. The aim is to achieve dynamic and KPI-driven provisioning of security measures for network slices. By leveraging digital twin technology, the paper suggests creating a virtual replica of the network slice, enabling the monitoring and management of security functions. This approach allows for the autonomous provisioning of security capabilities tailored to the specific requirements and key performance indicators (KPIs) of each network slice. Ultimately, the goal is to enhance the security of 5G network slices by dynamically adapting security measures based on their performance targets and characteristics.

% Vakaruk et al.\cite{vakarukDigitalTwinNetwork2021} discuss the need to train cybersecurity experts in preventing network attacks in the mission-critical industrial environment. They detailed the integration of machine learning (ML) tools into the SPIDER cyber range platform, which is a cybersecurity training system for experts in next-generation network cybersecurity. The SPIDER platform uses a highly virtualized synthetic traffic generation environment called Mouseworld to inject realistic traffic into a 5G network infrastructure, including attack activity within it. The Smart Traffic Analyzer (STA) component within the Mouseworld is used to train ML modules that can be utilised later as components of the SPIDER platform for detecting attacks in the injected traffic. The platform also applies Generative Adversarial Networks (GANs) to generate synthetic network traffic data (attacks and well-behaved connections) that reproduce the statistical distribution of real traffic. 

\textbf{\textit{IIoT Netwrok}}: To improve communication security and data privacy for DT powered Industrial Internet of Things (IIoT) network, Kumar et al.\cite{kumarBlockchainDeepLearning2022} introduces a framework that combines blockchain and deep learning. A New DT model is presented that  can simulate and replicate security-critical processes in a virtual environment, along with a blockchain-based data transmission scheme that uses smart contracts to ensure data integrity and authenticity. They also presents a Deep Learning scheme that utilizes Long Short Term Memory-Sparse AutoEncoder (LSTMSAE) technique to extract spatial-temporal representation and Multi-Head Self-Attention (MHSA)-based Bidirectional Gated Recurrent Unit (BiGRU) algorithm to detect attacks. The practical implementation of the framework demonstrates a significant enhancement in communication security and data privacy for the DT empowered IIoT network.



\textbf{\textit{Drone Network}}: With the aim of improving security of the CPS drone network Wu et al.\cite{wuDeepLearningDriven2022} study the use of DT as a simulation aid with deep learning.  
The author presents an attack prediction model  using improved long short-term memory (LSTM) networks and differential privacy frequent subgraph (DPFS) to ensure data privacy. The constructed model is simulated using the Tennessee Eastman process, and the results show that it has higher prediction accuracy and better robustness compared to other models. The DT technology is utilised to map the operating environment of the drone in the physical space, fully analyse the information security issues of the drone system in the virtual space, and detect multiple attacks and intrusions. However, the study has limitations due to the fact only 3 types of attacks(FDIA, replay attacks and DoS) took into consideration. in addition, only temperature sensor is under attack. Other factors like location, time and intensity of drone system are not considered for an attack.  




\textbf{\textit{Automotive industry}}: A framework called Trusted Twins for Securing Cyber-Physical Systems (TTS-CPS) that utilizes blockchain-based Digital Twins (DTs) to strengthen the security of Cyber-Physical Systems (CPSs) is presented by Suhail et al.\cite{suhailSituationalAwareCyberphysical2022}. The aim of TTS-CPS framework is to ensure the trustworthiness of data generated based on DT specification through Integrity Checking Mechanisms (ICMs). The authors argue that the framework helps to establish more understanding and confidence in the decisions made by underlying systems through storing and retrieving Safety and Security (S\&S) rules from the blockchain. In the paper, the authors demonstrate the feasibility of the TTS-CPS framework in an assembly line of automotive industry through a prototypical implementation supporting simulated network topology, Programmable Logic Controllers (PLCs), Human Machine Interfaces (HMIs), and physical devices. 

\textbf{\textit{Agriculture}}: In\cite{chukkapalliCyberPhysicalSystemSecurity2021} Chukkapalli et al. introduce a security surveillance system for a smart farm that keeps track of the data generated by sensors and alerts the farm owners. The system includes the collected sensor data, a smart farm ontology for creating knowledge graphs, and DT modules for anomaly detection. The researchers first use the collected data to generate knowledge graphs with the smart farm ontology and then use the DT to train the anomaly detection model using Principal Component Analysis. The authors demonstrate that the DT-based anomaly detection model can be used to detect various anomalies in the smart farm.

\textbf{\textit{Nuclear Power Plants}}: The authors in this \cite{guoCyberSecurityRisk2021a} paper proposes the use of Digital Twin technology to enhance the security of physical protection systems (PPS) in nuclear power plants. They develop a cyber security test platform based on digital twin technology, which allows for the evaluation of security measures without impacting the actual physical system. The digital twin technology combines multi-dimensional information perception, intelligent algorithms, and other tools to enable intelligent cognition and iterative optimization of real objects. The paper identifies threats from external and internal factors, referencing the national standard for classified protection of cybersecurity. 3D modeling is employed to digitize each physical object of the PPS, providing an intuitive display and enabling the association of important system information. The use of digital twin technology results in the creation of a cyber security test platform that facilitates the verification of various protection measures. Only measures that pass the test platform can be deployed in the real environment. Additionally, the test platform can be used for training purposes related to PPSs and cyber security.

% % ======================================================================================================
% NOTES, TODOS
% Modeling and simulation 
% AI and Machine learning
%     Deep learning
% Data Analytic
% Big data 
% Cloud and Edge Computing
% Block Chain

% ======================================================================================================
%
\subsection{Enabling Technologies That Power Digital Twin}
Digital Twin can be characterised as a warehouse of data and a virtual representation model of a real-world object with enabling or augmenting technologies to process the data. Its true power lies primarily due to the utilization of enabling technologies\cite{sousaELEGANTSecurityCritical2021}. The use of machine learning, blockchain, cloud computing, and big data analytics has enhanced the capabilities of digital twins, leading to better decision-making processes and improved security. In this section, we will examine the primary technologies that have been explored in the literature for enhancing the capabilities of Digital Twins to provide security services.

\subsubsection{AI and Machine Learning}
From our review, we observe Machine learning (ML) techniques are the most extensively studied enabling technology integrated with Digital Twin. 
In this section, we will examine several research contributions that demonstrate the potential of using ML with digital twin technology to detect and prevent security threats in various domains, such as power systems, industrial control systems, and network security. These contributions include integrating ML tools into cybersecurity training, using deep learning techniques for power system fault detection and drone security, employing generative adversarial networks (GANs) for the generation of synthetic flow-based network traffic, and using machine learning techniques for intrusion detection in industrial control systems.
  
One of the contributions discussed in a paper by Vakaruk et al.\cite{vakarukDigitalTwinNetwork2021} is the integration of machine learning (ML) techniques into cybersecurity training. The authors propose integration of ML to enhance the training processes of the SPIDER cyber range, a playground for cybersecurity training. The ML models classify input traffic as either malicious or benign and provide a confidence level for the prediction, which is reported to a dashboard for inspection and analysis by a human user.

Incorporating deep learning techniques into Digital Twin systems for power system fault detection is another contribution. Danilczyk et al.\cite{danilczykSmartGridAnomaly2021} utilized a Convolutional Neural Network (CNN) module for detecting power system faults and found it to be highly effective, with 95\% accuracy. This was better than other ML methods like MLP and LSTM RNN. Similarly, Wu et al.\cite{wuDeepLearningDriven2022} used ML for drone security, deploying with Digital Twin to address security issues and detect different attacks and intrusions. The proposed model was tested and compared against existing models, and it was found to have higher accuracy as evidenced by a low root mean square error. 

In a study by Rebecchi et al.\cite{rebecchiDigitalTwin5G2022}, the application of Generative Adversarial Networks (GANs) was proposed for the generation of synthetic flow-based network traffic that can imitate both normal and malicious activities. The generated data serves as training input for ML models, which are used by "Blue Teams" to detect security threats, such as crypto mining attacks.

The use of ML algorithms in the simulation and optimization environment of a smart grid is another contribution discussed by Atalay et al.\cite{atalayDigitalTwinsApproach2020} The authors suggest ML algorithms for the classification of abnormal behaviour during the system's lifecycle.

In a paper by Salim et al.\cite{salimBlockchainEnabledSecureDigital2022} deep learning techniques were used to analyze packet headers for detecting botnet activities that are associated with external IP addresses. The capabilities of deep learning algorithms to identify suspicious activities and malicious nodes within a network allow digital twins to accurately monitor and detect malicious behaviour.

In a study by Sousa et al.\cite{sousaELEGANTSecurityCritical2021}, ML was used for security analysis with the primary objective of detecting anomalous activity related to Denial of Service (DoS) and Distributed Denial of Service (DDoS) attacks. The developed model examines incoming traffic and categorizes it as either normal or malicious traffic.

In Li et al.'s\cite{jiaqiliSpaceSpiderHyper2022} work, AI was used as part of advanced technologies, including big data, to provide a better overall educational experience for those participating in the competition and training. The goal was to improve the evaluation and enhance the training effect.

Finally, Varghese et al.\cite{vargheseDigitalTwinbasedIntrusion2022} used machine learning techniques for intrusion detection in an industrial control system (ICS). The study used eight supervised ML-based Intrusion Detection System (IDS) techniques and evaluated their performance using a labelled dataset. The performance of the algorithms was evaluated based on accuracy, precision, recall, and F1-score. The study also used an ensemble approach, stacking multiple classifiers, to design a signature-based IDS. Similarly, In a paper by Akbarian et al.\cite{akbarianIntrusionDetectionDigital2020}, a digital twin-based intrusion detection technique using machine learning was proposed. The attack classification approach used was Support Vector Machine (SVM), which can handle both binary and multi-class classification. The intrusion detection mechanism involves a combination of a Kalman filter for attack detection, a particle swarm optimization algorithm for noise estimation, and an SVM algorithm for attack classification.


 

\subsection{Blockchain and Smart Contract}
The integration of Digital Twin and Blockchain is also the most extensively studied area next to Machine learning. Research explores the benefit of blockchain to provide secure and uncorrupted data that is shared between various stakeholders. In this section, we compile three studies that use blockchain to improve security and accountability in the industrial internet of things (IIoT) and smart grid environments.  


 Kumar et.al\cite{kumarBlockchainDeepLearning2022} proposed a digital twin in IIoT network that includes various layers, such as IoT Device Layer, Edge-Blockchain, and Cloud-Blockchain Layer. The primary aim of blockchain is to ensure the integrity and authenticity of data transmitted over public communication channels by leveraging smart contracts. The study employed the Ethereum test network for scalability analysis and incorporated IPFS off-chain storage system to securely store encrypted IIoT transactions. Salim et.al\cite{salimBlockchainEnabledSecureDigital2022} explored the use of blockchain technology in securing IIoT environments against cyberattacks, particularly botnets. They proposed a Blockchain-enabled Digital Twin Framework that synchronizes and authenticates data with their corresponding IoT devices using a private blockchain network, leveraging smart contracts to authenticate the communication between the digital twins and packet auditors. Lopez et.al\cite{lopezDIGITALTWINSINTELLIGENT2021} focused on the use of blockchain technology in the smart grid to enhance the security and accountability of data. Blockchain solutions were proposed to synchronize information between partners securely, ensuring data ownership and traceability. Overall, the integration of blockchain technology with digital twin systems has shown potential for enhancing data security and authenticity in IIoT environments and the smart grid.

\subsubsection{Cloud and Edge Computing}
Researchers are exploring different ways to deploy digital twins to improve performance and reduce costs. In this section, we briefly discuss various studies that have explored the deployment of digital twin systems using cloud and edge computing. 

In one study \cite{akbarianSecurityFrameworkDigital2021}, a digital twin is deployed in the cloud to take advantage of its unlimited computational and storage resources. An intrusion detection system is placed near the controller in the cloud, allowing it to monitor the signals sent to the controller and assess its health. Another study \cite{grasselliIndustrialNetworkDigital2022} describes how virtual components of the digital twin can be deployed as Virtual Network Functions (VNFs) on Virtual Machines (VMs) to replicate elements of the real infrastructure such as firewalls, intrusion detection/prevention systems, and industrial application gateways. A cloud infrastructure managed with OpenStack and OSM provides dynamic control over the compute, storage, and network resources needed for the instantiation of the digital twin. Gupta et.al. \cite{guptaHierarchicalFederatedLearning2021} propose deploying digital twins on edge devices to reduce the gap between physical objects and their digital representations hosted on cloud servers. They design a Federated Learning (FL) based Anomaly Detection (AD) model and deploy it on an edge cloudlet associated with a patient's digital twin. The use of edge cloudlet computing enhances data privacy and improves model performance. In another study \cite{saadImplementationIoTBasedDigital2020}, a digital twin solution is introduced that is composed of two parts, with the second part built as a function on the cloud. The deployment of digital twins on cloud and edge devices offers different benefits and researchers are exploring different approaches to take advantage of these benefits.

% ===========================================================================
\subsubsection{Big Data and Data Analytics}
By incorporating data analytics techniques, digital twins can provide powerful insights\cite{jiaqiliSpaceSpiderHyper2022} into the behaviour and performance of physical systems. In this section, we provide a review of studies that leverage big data processing and analytics to augment Digital twins.  
 

Salvi et al.\cite{salviCyberresilienceCriticalCyber2022} introduced a conceptual model that integrates data analytics and causal analysis to detect potential attacks on critical cyber infrastructures in real-time. Wang etal.\cite{wangDTCPNDigitalTwin2022}, proposed a platform that integrates big data processing capabilities to efficiently obtain, store, and search data of specific objects, which can be used for simulation and network security analysis. Hussaini et al.\cite{hussainiTaxonomySecurityDefense2022}  discussed how machine learning and deep learning-based data analytics can strengthen digital twins' monitoring capabilities against various attacks. Li et al.\cite{jiaqiliSpaceSpiderHyper2022} used big data and artificial intelligence to process and analyze large amounts of data to provide deeper insights into network security scenarios. Dietz et al. \cite{dietzIntegratingDigitalTwin2020} presented security analytics tools for processing log data for security incident detection in digital twin simulations, while Almeabed et al.\cite{almeaibedDigitalTwinAnalysis2021}  developed a framework based on data processing and analytics capability for vehicular digital twins. These studies illustrate the potential of big data processing and analytics to improve the capabilities and security of digital twins, allowing for more effective modeling and analysis of real-world systems.


% % ======================================================================================================
% NOTES, TODOS
% intrusion detection 
%     in security operation center, anomaly detection, 
% Saftey
% ICS security
% Security testing and validation
% Training and cyber range
% Modeling
% Simulation
% Threat Intelligence
% Asset management
% patch management
%     secure software updates 
% Risk management.

% ======================================================================================================
%
\subsection{Use Case of Digital Twin in Industry 4.0}

\subsubsection{Threat detection and response}
The use of digital twins as a tool for anomaly detection is the more widely used application of DT than other use cases of DT across various domains, including industrial control systems(ICS), smart grid, and cyber-physical systems(CPS). It can be used for detecting of abnormal process events \cite{xuGametheoreticApproachSecure2020} or deliberately injected malicious content \cite{saadImplementationIoTBasedDigital2020} in real-time \cite{vargheseDigitalTwinbasedIntrusion2022}. By continuously monitoring the virtual representation of the system, digital twin can also be equipped with tooling for prompt intervention and resolution of issues \cite{akbarianSecurityFrameworkDigital2021}. Intrusion or anomaly detection enabled digital twin can be used in preventing security breaches such as Network intrusion, DDoS attack, Botnet infection and son on. 

William et.al \cite{danilczykSmartGridAnomaly2021} demonstrates how DT-based deep learning convolutional neural network (CNN) module can be used to detect and locate power system faults by incorporating it into their previous proposed Digital Twin framework called ANGEL\cite{williamdanilczykANGELIntelligentDigital2019} that was only used as a visualization tool. In this work, the architecture of the deep learning algorithm was tested on IEEE 9 and 39 bus power systems(standard test models used in power system analysis and research), which showed successful results in detecting and classifying faulty buses. in the paper by Xu et al.\cite{xuGametheoreticApproachSecure2020}, the authors propose a solution to the problem of stealthy estimation attacks on Cyber-Physical Systems (CPSs) by integrating a chi-square detector--a statistical method used to detect differences or deviations between observed data and expected data -- into the digital twin of the CPS. The chi-square detector monitors the system's behavior and raises an alarm if an unusual deviation is detected. Further, the authors apply the Signaling Game with Evidence (SGE) method to determine optimal attack and defense strategies. 

In \cite{chukkapalliCyberPhysicalSystemSecurity2021}, Chukkapali et.al proposes a security surveillance framework for detecting deviations in the CPS ecosystem of smart farms. The framework incorporates an anomaly detection model supported by digital twins. In work done by Varghese et.al.\cite{vargheseDigitalTwinbasedIntrusion2022} use a more advanced machine learning model and a stacked ensemble classifier as a real-time intrusion detection system in CPS. it is based on the offline evaluation of eight supervised machine learning algorithms and has been shown to be superior in terms of accuracy and F1 Score, effectively detecting intrusions in close to real-time (0.1 seconds). Different from the previous two papers, Salim et.al \cite{salimBlockchainEnabledSecureDigital2022}shows the integration of Digital Twin with Blockchain can be used for botnet detection. In this work, the authors address a growing concern of botnets in industry 4.0 by proposing a blockchain-enabled digital twin that can detect botnet spread at an early stage. The digital twin employs certificate revocation to impede the spread of botnets.  

 Saad et.al.\cite{saadImplementationIoTBasedDigital2020} presents an IoT-based digital twin platform aimed at enhancing the resiliency of cyber-physical networked microgrids against cyber attacks. The platform provides centralized oversight and the ability to detect false data injection, denial of service, and coordinated attacks, with corrective action based on what-if scenarios.  
 
 Lopez et.al\cite{lopezDIGITALTWINSINTELLIGENT2021} explores the use of digital twin in real-time to anticipate faults and detect security issues in CPS. The digital twin has the ability to update access control policies, thus mitigating an attack or fault.  
 
 Another approach, proposed by Hussaini et.al \cite{hussainiTaxonomySecurityDefense2022} discusses the importance of intrusion detection techniques in improving the security aspect of CPS. The author suggest the use of advanced data analytics techniques to detect intrusions.  
 
 Akbarian et.al.\cite{akbarianSecurityFrameworkDigital2021} employs a digital twin-based security framework for ICS with attack detection and mitigation components. The framework features a digital IDS and a mitigation method to preserve system stability during attacks. The study was evaluated through experiments on a real testbed.  

In general, the authors propose using digital twins to model a system's behavior and monitor it for anomalies or deviations that may indicate an intrusion. Deep learning techniques\cite{williamdanilczykANGELIntelligentDigital2019}, data analytics techniques\cite{hussainiTaxonomySecurityDefense2022}, and machine learning algorithms\cite{williamdanilczykANGELIntelligentDigital2019, chukkapalliCyberPhysicalSystemSecurity2021, vargheseDigitalTwinbasedIntrusion2022} are commonly used to analyze the data generated by the digital twin.

\subsubsection{Vulnerability assessment and validating security measures}
% Using DT as a simulation for attack and defense 
% risk evaluation
% Assessment tools
% penetration test

Many researchers have also explored the use of digital twins in testing, verification, and assessment of cybersecurity issues. It involves creating a comprehensive virtual representation of a physical system, which can be used to simulate and verify its behavior in different scenarios. This approach offers numerous advantages over traditional methods of testing and verification, including reduced cost\cite{franciaDigitalTwinsIndustrial2021, jiaqiliSpaceSpiderHyper2022, shitoleRealTimeDigitalTwin2021, maillet-contozEndtoendSecurityValidation2020}, increased accuracy\cite{sugumarAssessmentMethodDetecting2019, atalayDigitalTwinsApproach2020}, and reduced disruption \cite{franciaDigitalTwinsIndustrial2021, atalayDigitalTwinsApproach2020, adrienbacueDigitalTwinsEnhanced2022} to real-world operations.  
In this paper, we will summarize and expand on eleven different contributions from researchers in the field of digital twins for vulnerability assessment, Testing, and Verification use cases.  

A contribution by Dietz et.al\cite{dietzEmployingDigitalTwins2022} explores the use of digital twins for security testing of industrial control systems (ICS). The study demonstrates how a digital twin of a programmable logic controller (PLC) can be used to detect potential threats in a simulation environment. A proof-of-concept using a pressure vessel was implemented to show the feasibility of this approach.  

Atalay et.al \cite{atalayDigitalTwinsApproach2020} proposed a digital twins-based approach for accurately modeling the functioning of a real-world power grid while avoiding service disruptions caused by running tests on it. This paper aimed also to fill the security evaluation standard gap in the smart grid sector and includes elements such as virtual tiers and libraries of profiles/system parameters for security evaluation. However, the authors noted the lack of validation for their proposed framework in real scenarios.  

Eckhart \cite{eckhartEnhancingCyberSituational2019} proposed a cyber situational awareness framework for a comprehensive and current view of the cyber situation, including visualization and playback for deep inspection of recorded events.  

Adrien Bacue et al \cite{adrienbacueDigitalTwinsEnhanced2022} conducted a case study to show how digital twins can be used for simulating attacks and designing countermeasures without affecting the internal operation of an industry.  

In\cite{franciaDigitalTwinsIndustrial2021}, Francia et.al investigated the use of digital twins for ICS security testing with cost-effective, non-disruptive methods. The authors conducted an experiment to showcase the use of digital twins for ICS security testing by creating a proof-of-concept system using a bottle-filling system controlled by a PLC. They used packet generator and transmitter tools to inject network packets with various protocols into the communication channel between digital twins and their physical counterparts.  

Wang et.al \cite{wangDTCPNDigitalTwin2022} presented a platform called Cyber Digital Twin based on Network Function Virtualization (NFV) for testing network security and management. As an example, they demonstrated a Distributed Denial of Service (DDoS) attack and defense mechanism using their proposed platform.  

In \cite{shitoleRealTimeDigitalTwin2021} Shitole argued that the use of real-time CPS testbeds for security studies has become popular, but these testbeds have limitations such as high costs and inflexibility. In this study, the authors offered a cost-effective platform called Real-Time Digital Twin for security testing of CPS systems by creating a virtual representation of the physical system.  

Jiaqi Li et.al \cite{jiaqiliSpaceSpiderHyper2022} proposed a simulation and verification platform called "Space Spider", which is a digital twin-based hyper large scientific infrastructure to simulate attacks and defenses for verification of core technologies used in space internet.  

A novel approach for the integration, verification, and validation of security in devices based on the digital twin concept was introduced by Maillet et.al\cite{maillet-contozEndtoendSecurityValidation2020}. This approach involves creating a comprehensive virtual representation of a physical device, composed of black-box and white-box models at different abstraction levels. By using this approach, the cost impact of adding security to physical devices is reduced while still ensuring the security and functionality of the device.  

In \cite{sugumarAssessmentMethodDetecting2019} Sugumar et.al explored the use of digital twin models for assessing the effectiveness of anomaly detectors by launching attacks like scaling and pulse. The authors claimed that their proposed solution can be used quickly and accurately when compared to other methods involving simulation or direct testing on operational testbeds. However, they did not evaluate other types of attack templates such as random or ramp attacks, which could potentially be used against critical infrastructure.  

% \subsection{Network Segmentation}

\subsubsection{Threat Intelligence}
% collecting data, processing, and analytics. 

One of the well-known use cases of Digital Twins is the capability of providing visualization abilities through the collection of real-time data using sensors from the physical object. The collected data can then be used for various purposes, including cyber threat intelligence. Despite the benefit of threat intelligence for security measurement, the contributions in this area are limited. In the following, we will explore three contributions that demonstrate the use of Digital Twins based on data collection, visualization, and data processing for threat intelligence.  
In \cite{williamdanilczykANGELIntelligentDigital2019}, William Danilczyk et.al propose a framework called ANGEL to model the cyber and physical layers of a microgrid and offers real-time data visualization. It uses machine learning algorithms to allow for the simulation of multiple states concurrently, enabling the Digital Twin to make intelligent decisions based on the physical system's state.  

Almeaibed's \cite{almeaibedDigitalTwinAnalysis2021} research suggests the use of analytical techniques for processing incoming data and generating reports on security and safety concerns. This is demonstrated through a case study of how hackers can manipulate radar sensor readings and potentially cause collisions.  

Dietz et.al \cite{dietzHarnessingDigitalTwin2022} outline systematic steps for creating a structured threat report through digital twin security simulations. They present the process, define requirements, conduct attack simulations, and use the STIX2.1 standard and custom tools to generate Cyber Threat Intelligence (CTI) reports. The results show that shareable CTI reports can be created using digital twin security simulations.  
\subsubsection{Security Awareness and Training}
% Improve cybersecurity skills of employee through cyber range. 
% DT as simulation 

The integration of digital twins and cyber-ranges has proven to be an effective solution for enhancing cybersecurity skills through training and simulation. For example, vakaruk et.al \cite{vakarukDigitalTwinNetwork2021} proposed a machine learning-augmented CyberRange tool, SPIDER, to train and enhance cybersecurity skills. The tool features a DTN solution that provides an orchestration and management component within the SPIDER cyber range environment, capable of emulating various scenarios.  
- In \cite{becueCyberFactorySecuringIndustry40with2018} Becue et.al studied the potential of using digital twins in combination with cyber ranges for cybersecurity training and simulation. The authors focused on the design of the cyber range to test the weaknesses of the digital twin by generating traffic and attacks, which allows the assessment of the impact of these attacks on the system and provides insight to support the operator in decision-making. The training offered through the cyber range focuses on building personalized cybersecurity competence through realistic scenarios.  

Another integration of Digital Twin and cyber-range aimed for training and research in the field of cybersecurity proposed by Kandasamy et.al \cite{kandasamyElectricPowerDigital2022}. The study offers users the ability to simulate real-world attacks and defense techniques, as well as make changes to components or configurations with ease, making it simpler to repeat the environment for security studies on smart grids. By utilizing both tools, researchers can acquire valuable information on the potential threats posed by smart grids, leading to the development of stronger security strategies for future use. 

To address the shortage of cybersecurity skills in 5G networks, Rebecchi et.al \cite{rebecchiDigitalTwin5G2022}, propose a solution through the use of an emulative approach. The authors built a realistic 5G environment through a platform composed of several interacting components, including an Emulation Scenario Editor, Knowledge Base, OSS Northbound API, and Executable Service Graphs. The goal of the platform was to provide training scenarios for ethical hackers and DevOps engineers.  
% \subsection{Intrusion Detection}









% ======================================================================================================
% NOTES, TODOS
%
% ======================================================================================================
%
\section{RQ2: What are the security schemes presented in the literature }

To ensure the reliability and security of Digital Twin based systems, it is essential to have secure communication between the physical and digital components. In this section, we will examine the various methods proposed in recent studies to secure the communication between Digital Twin and its physical counterpart. The studies reviewed will cover topics such as access control systems, cryptography, authentication protocols, privacy protection mechanisms, quantum networking, and blockchain-based data transmission with the aim to provide an overview of the current state of research in securing communication in cyber-physical systems based on Digital Twin and IIoT components.

Gehrmann et al.\cite{gehrmannDigitalTwinBased2020} discuss the implementation of a single central access control system that is based on policies defined using standard frameworks such as XACML and tokens like SAML and OAuth. These policies help regulate who has access to what information and ensure the security of the communication.  

In\cite{xuGametheoreticApproachSecure2020} Xu et al. describe a system with two channels for communication between sensors and other components. The first channel is used for communication between sensors and a Digital Twin and is protected using cryptography like Message Authentication Code (MAC) or Digital Signature (DS). This channel is considered secure because cryptography helps protect the integrity of the data flow between the sensors and the digital twin. The second channel, used for communication between sensors and a physical estimator, does not use cryptography because it would negatively impact the performance of the physical system.  

To solve the security problems such as communication trust and privacy protection, the authors in\cite{xuEfficientAuthenticationVehicular2021} propose a secured vehicular digital twin communication framework that utilizes anonymous authentication. To achieve this, the authors present a concrete authentication protocol based on a secret-handshake scheme and group signature, which solves the issues of unforgeability and conditional traceability. The proposed framework provides secure communication between iTwins(DT) and their physical lords, as well as between iTwins(DT) themselves, ensuring the privacy and security of the information transmitted. The proposed protocol has been validated and found to meet basic security requirements while having low computation cost.  

Jingyi Wu et al.\cite{wuDeepLearningDriven2022}presents a method that focuses on the privacy and confidentiality of data used for training detection models in drones of cyber-physical systems. The authors use differential privacy techniques to improve the accuracy and efficiency of the analysis of drone data while ensuring the protection of sensitive information. 

Kumar et al.\cite{kumarBlockchainDeepLearning2022} suggest a blockchain-based data transmission scheme that employs a Proof-of-Authentication (PoA) mechanism, which is implemented through the use of smart contracts. This helps to validate the legitimacy and integrity of data collected from Internet of Things (IIoT) nodes, improving communication security and data privacy within a decentralized IIoT network.  

In \cite{salimBlockchainEnabledSecureDigital2022} Salim's work involves securing the communication between IoT devices and Digital Twins using a private blockchain, smart contracts, and deep learning for network traffic monitoring. The private blockchain and smart contracts help ensure the data flow between physical devices and DTs is secure and tamper-proof. The deep learning model helps detect early signs of botnet behavior and alerts the security vendor to take action to isolate infected devices, maintaining the security of the communication and the integrity of the data.  


A study conducted by Zhigan Lv et al.\cite{lvDigitalTwinsBased2022} aims to enhance the communication security between industrial Internet of Things (IIoT) devices and Digital Twins (DTs) by using quantum communication technologies. The authors introduce a channel encryption scheme based on quantum communication, using entanglement states and quantum teleportation. Further, they propose an Adaptive Key Residue algorithm based on quantum key distribution mechanism. The goal is to improve the security of communication between IIoT devices and DTs.

Lai et al.\cite{chengzhelaiSPDTSecurePrivacyPreserving2022} present a scheme for secure and privacy-preserving traffic control data sharing using digital twins. The scheme incorporates a group signature with time-bound keys for data source authentication and efficient member revocation during the data uploading phase, ensuring secure data storage on the cloud service provider. Moreover, the scheme includes an attribute-based access control technique for flexible and efficient data sharing during the data sharing stage. The primary objective of this scheme is to guarantee effective and secure data sharing for traffic control purposes

in\cite{debenedictisAdoptionSecureCyber2022} De Benedictis addresses the security and trustworthiness of the communication between the digital twin and physical device through various technologies and HW and SW solutions such as Trusted Execution Environment platforms and Physically Unclonable Functions (PUFs) for device authentication. In addition,  Blockchain technology, which provides secure, immutable and auditable data storage for the exchanged critical data is investigated. 

In Olivares-Rojas et al.'s recent study \cite{olivares-rojasCybersecuritySmartGrid2022}, a Digital Twin framework was proposed to avoid sniffing attacks and ensure secure communication between physical and virtual objects. To achieve this, the authors utilized RSA signatures, which provide robust encryption that is difficult to compromise.
