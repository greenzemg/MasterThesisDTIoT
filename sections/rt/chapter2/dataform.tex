% ======================================================================================================
% NOTES, TODOS
% ======================================================================================================

\subsection{Data Extraction Form}
Kitchenham and Charter \cite{kitchenham_guidelines_2007} state that a well-designed data extraction form is helpful to gather information from primary studies to address research questions. To achieve this in our study, we utilised the web-based tool, \textit{parsif.al}, to structure and design the data extraction form used to collect data from the selected articles. The data extraction form used in this systematic literature review is presented in Table \ref{tbl:extraction}.

% Note that, the data extraction form is, and also was in our case, continuously evolving during the full-text review process. 


\begin{table}[h]
\small
\centering
\caption{ Data extraction form.}
\label{tbl:extraction}
\begin{NiceTabular}{p{4cm}p{8cm}}
\toprule
    \textbf{Data Point} & \textbf{Options/Explanation} \\
    \midrule
    \textbf{Aim of research} & Summarized version of the aim of the paper. \\ 
    \textbf{Targeted sector} & The studied or targeted Industry 4.0 sector \\
    \textbf{DT purpose} & The use case of the proposed DT \\ 
    \textbf{Enabling technology} & Technology integrated with DT to provide security service \\ 
    \textbf{Security mechanism} &  Authentication and encryption mechanisms used   \\ 
    \textbf{Contribution category} & Framework, Algorithms, Architecture, Model, Platform \\
    \textbf{Study type} & Paper with case-study, Experiment based, Theoretical concept, Review paper science  \\
\bottomrule
\end{NiceTabular}
\end{table}