% ======================================================================================================
% NOTES, TODOS
% ======================================================================================================
% describe how to perform the detail assessment
% use numeral scale 
% Are the aim of the article clearly stated?
% Is the implementation detail explained adequately 
% does the study has direct link to research question 1 and/or question 2. 
% 
\subsection{Quality Assessment Checklist}
Beside the inclusion and exclusion criteria, it is important to evaluate the quality of the research study [kitchen]. Quality assessment checklist define the detail assessment criteria for selecting paper during screening abstract and introduction of the paper. In this stage, articles that are not relevant to the research question and the objective of the study are identified and removed [kofod]. 

Below, we outline quality assessment questions as a checklist. 
\begin{itemize}
    \item \textbf{QA1:} Does the study clearly define the aim and objective of the research?
    \item \textbf{QA2:} Does the study fully explain a methodology to secure IoT/IIoT application using Digital Twin
    \item \textbf{QA3:} Does the study has direct link to research question 1 and / or question 2?
    \item \textbf{QA4:} Does the study conducted an experiment(test) to validate the hypothesis? 
    \item \textbf{QA5:} Does the study provide detail implementation of authentication scheme?
    
\end{itemize}
The quality assessment questions outlined above were evaluated using a numerical metric scoring system, with a range of 1-5, where 1 represents an irrelevant article and 5 represents a highly relevant and qualified article. The level of agreement in answering the quality checklist questions was determined through a categorical classification system, comprising of "Agree," "Somewhat Agree," "Neutral," "Somewhat Disagree," and "Disagree." These categories were assigned corresponding weight values, with "Agree" being assigned the highest weight value of 5 and "Disagree" being assigned the lowest weight value of 1. This scoring system helped to make sure that we were evaluating the research studies in a fair and consistent way, so that the studies that were most relevant and of the highest quality were selected for the review.
