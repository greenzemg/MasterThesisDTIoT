% ======================================================================================================
% NOTES, TODOS
% ======================================================================================================

\subsection{Defining PICOC}
PICOC stands for Population, Intervention, Comparison, Output and context. It is a widely used technique in medical and social science studies to define the focus of the research[carrera]. The techniques can still be applied for research in computer science to formulate and structure research questions. In this subsection we define the  our PICOC criteria for this systematic literature review.

\textit{Population:} The motivation for us to conduct this research is the security related problem we identified in communication between digital twin and IoT devices. The problem domain or "Population" for this research cloud be any business operations that use Digital Twin and IoT/IIoT such as smart cities, smart homes, industry, critical infrastructures, smart manufacturing, and so on. In this sense, the "Population" part of PICOC is analogous to the following terms: Digital Twin, (Industry)Internet of things, Industry 4.0, smart manufacturing, cyber-physical systems, critical infrastructure. 

\textit{Intervention:} Our intervention to tackle this problem is to implement a light weight NIST standard cryptographic authentication scheme for power and storage constraint devices. With this regard, we define the term authentication as intervention. 

\textit{Comparison:}Before deigning and implementing intervention for specific problem, it is important to identify the existing solution in the literature. Result of reviewing, comparing, and analyzing existing solution discussed in relevant research literature can be used as an input for designing and implementing the intervention methodology. With this regard, in this study, we will identify and compare authentication schemes or security mechanism used in securing a data flow between Digital Twin and IoT application. 

\textit{Outcome:} Secure remote access with integrity and confidentiality of communication, efficient and performant cryptographic schemes that can run on constraints device is the expected outcome of this research.\\

\textit{Context:} This research shall have academic and industrial context[help]. 