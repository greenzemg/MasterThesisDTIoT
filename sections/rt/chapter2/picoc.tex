% ======================================================================================================
% NOTES, TODOS
% ======================================================================================================

\subsection{Defining PICOC}
PICOC stands for Population, Intervention, Comparison, Output and context. It is a widely used technique in medical and social science studies to define the focus of the research\cite{carrera-rivera_how-conduct_2022}. However, in\cite{carrera-rivera_how-conduct_2022, kitchenham_guidelines_2007} Kitchenham and Carrera showed that this technique can still be applied for computer science related research to formulate and structure research questions. In this subsection, we define our PICOC criteria for this systematic literature review.

\textit{Population:} The motivation to conduct this research is the security related problem we identified in the communication between Digital Twin and constrained  (I)IoT devices deployed in the smart industry to collect sensor data. Hence, the problem domain or "Population" for this research is (I)IoT devices used with Digital Twin to enhance security in Industry 4.0. Industries that use Digital Twin and (I)IoT devices, such as smart cities, smart homes, smart grids, smart health, smart manufacturing, etc. In this sense, the "Population" part of PICOC in this review refers to the following terms: Digital Twin, (Industrial)Internet of Things, Industry 4.0, Smart Manufacturing, Cyber-physical Systems, and Critical Infrastructure. 

\textit{Intervention:} Our intervention to address this problem, the security issue of digital communication between Digital Twin and (I)IoT, is to implement a lightweight NIST standard cryptographic authentication/encryption scheme for power and computation constraint (I)IoT devices. In this regard, we use the term "authentication" as an intervention.

\textit{Comparison:} Before designing and implementing an intervention for a specific problem, it is important to identify the existing solution in the literature. The results of reviewing, comparing and analysing the existing solution discussed in the relevant research literature can be used as input to design and implement the intervention methodology. With this regard, this study will identify and compare authentication schemes or security mechanisms used in securing a data flow between Digital Twin and (I)IoT. 

\textit{Outcome:} Secure remote access with integrity and confidentiality of communication, efficient and performant cryptographic schemes that can run on constrained devices is the expected outcome of this research.\\

\textit{Context:} This systematic literature review is focused on the Industry 4.0 environment, targeting Digital Twin solutions deployed in smart industries to enhance security. However, the second part of the study is dedicated to designing and implementing authentication schemes for the Smart Grid which is considered as one instance of Industry 4.0.