% ======================================================================================================
% NOTES, TODOS
% ======================================================================================================
\subsection{Search Queries and Search Strategy}

 Each of the selected database comes with their own way of performing advance searching. The search field and filtering option are different from one database to another. Having this into consideration, we followed a search strategy we called three stage searching mechanism. The detail at each stage is explained as follow. 

The first stage involved searching for papers whose titles included the main key term "Digital Twin." The second stage involved narrowing the search results by including security-related terms such as "authentication," "security," "encryption," and "cryptography" in the abstracts of the papers. In the final stage, if the number of retrieved papers exceeded 30, the search results were further narrowed by incorporating industry and IoT-related terms in the full text of the research papers. 

\begin{tcolorbox}[colback=black!5!white, sharp corners=all, colframe=white!95!black]
\textbf{Web of Science}
\tcblower
("digital twin*" OR "digital-twin*") (Title) AND ( "authenticat*" OR "cryptography" OR "security" OR "encrypt*" ) (Abstract) and English (Languages) and Article or Proceeding Paper (Document Types) and Engineering or Computer Science (Research Areas)
\end{tcolorbox}

In Web of Science, "Topic"(i.e Title, keyword, and Abstract) filed is used to search for digital twin and internet of things terms. The security related terms like authentication, encryption, cryptography and encryption as well as terms related to industry are searched on all fields. From the search results of our query, document types such as book chapters, early access,and editorial are excluded. In other words, only document of type article and conference papers are selected. We run the search query over all available years which are published under category of computer science and 30 articles are returned as a result.
\begin{tcolorbox}[colback=black!5!white, sharp corners=all, colframe=white!95!black]
\textbf{Scopus}
\tcblower
( TITLE ( ( "digital twin*" OR "digital-twin*" ) ) AND TITLE-ABS-KEY ( "authenticat*" OR "cryptography" OR "security" OR "encrypt*" ) AND TITLE-ABS-KEY ( ( "industr*" OR "Industry 4.0" OR "factor*" OR "manufactur*" OR "smart manufacturing" OR "cyber-physical system*" OR "cyber physical System*" OR "infrastructure*" OR "industrial control system*" ) ) ) AND ( LIMIT-TO ( SUBJAREA , "COMP" ) OR LIMIT-TO ( SUBJAREA , "ENGI" ) ) AND ( LIMIT-TO ( DOCTYPE , "cp" ) OR LIMIT-TO ( DOCTYPE , "ar" ) ) AND ( LIMIT-TO ( SRCTYPE , "p" ) OR LIMIT-TO ( SRCTYPE , "j" ) ) AND ( LIMIT-TO ( LANGUAGE , "English" ) )
\end{tcolorbox}
We employ the same search methodology and search terms for Scopus as we do for Web of Science. In the first stage, the "Topic" i.e. title of the article , was used to search for papers containing the terms "digital twin" or "digital-twin". This initial search yielded a total of 5851 references, which were then subject to exclusion criteria. In the second stage, the abstract and keywords sections of the papers were searched for terms related to security, such as "authentication", "encryption", "cryptography", and "security". This narrowed the results down to 240 papers. The final stage involved further refining the search by incorporating keywords related to industry and the Internet of Things. Only articles and conference papers were selected, and documents such as book chapters, early access, and editorials were excluded. The search was conducted over all available years in the category of computer science, resulting in a total of 151 articles at the final stag 

\begin{tcolorbox}[colback=black!5!white, sharp corners=all, colframe=white!95!black]
\textbf{IEEExplore}
\tcblower
("Document Title":"digital twin*" OR "Document Title":"digital-twin*") AND ("Abstract":"authenticat*" OR "Abstract":"cryptography" OR "Abstract": "security" OR "Abstract":"encrypt*") \\

Filters Applied: Conferences Journals
\end{tcolorbox}
In the case IEEE, we use a different search strategy from the above two cases. while we perform search for terms  authentication and industry related on full text , the "All Metadata" field is used to perform the others search terms on title, keywords, and abstract. Articles from conferences and journals are selected and articles under category of early access and magazines are excluded. Finally, Our query resulted in 37 research studies.   

\begin{tcolorbox}[colback=black!5!white, sharp corners=all, colframe=white!95!black]
\textbf{ACM}
\tcblower
[[Title: "digital-twin*"] OR [Title: "digital twin*"]] AND [[Abstract: "security"] OR [Abstract: "authenticat*"] OR [Abstract: "encrypt*"] OR [Abstract: "cryptography"]]

\end{tcolorbox}
ACM is a bibliographic database that exclusively focuses on publishing computing literature. In ACM we run the search query within the search field of "Anywhere". And only research articles are included, and as a result 12 research studies are retrieved. 




