% ======================================================================================================
% NOTES, TODOS
% ======================================================================================================
\subsection{Study Selection and Refinement}
% 74 selected papers -> 14 not relevant and 3 duplicate studies submitted to different journals  excluded during full review of the papers. 

Even thought, we initially selected 73 papers for analysis. Upon further examination, we discovered three studies that were duplicates with different metadata but had similar content, and had been submitted to different journals. These duplicates were not identified by the tools we had used to exclude them. In addition, through quality assessment checklist, we also excluded fourteen papers for data extraction phase.

The reasons for the exclusion of these papers were as follows:

\begin{itemize}
    \item When the paper discussed how to secure the digital twin itself, rather than securing IoT applications using digital twin technology or securing the communication channel between DT and (I)IoT. 
    \item If the paper lacked a clear objective and aim.
    \item Some were not related to securing (I)IoT applications with an Industry 4.0 use case (for example, a study that used a digital twin to secure a data centre).
    \item Study sourced from book chapter. 
    \item The study was not relevant to any of the research questions.
    \item A study that focuses on securing (I)IoT devices that are not associated with any industry use case. 
\end{itemize}

As a result of this refinement and selection process, we were left with final set of 56 papers that were used for data extraction and analysis. In the following chapter, we provide a review of the 56 papers focusing to answer two research questions: How is digital twin used to improve the security of (I)IoT applications and what security mechanisms are used to secure the communication channel?    



