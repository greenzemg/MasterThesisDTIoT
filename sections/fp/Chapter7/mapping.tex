
\section{Ditto Java Base Payload Mapping}

In the context of Eclipse Ditto, data storage and transfer are facilitated through a format known as the Ditto protocol. This protocol utilizes a JSON structure, employing key-value pairs to represent and transmit information.

To seamlessly integrate with Ditto's capabilities, the connectivity microservice bundled with Ditto offers an extension specifically designed for intercepting incoming data. This extension allows for the mapping of data from its original form to a format that Ditto can understand and store in its underlying MongoDB database. Using the Ditto payload mapping feature, we can decrypt incoming encrypted payload messages and convert them into a format that Ditto can process and store.
 
With the payload mapping feature in Ditto's connectivity microservice, we can do the following: receive encrypted data from the IoT device, decrypt and authenticate it, and convert it into Ditto protocol messages. This helps ensure that the data sent between the IoT device and Ditto is secure and authentic.

To implement our custom mapping functionality to encrypt and decrypt, we perform the following steps:
\begin{itemize}
    \item[-] Implement and build a Java class as Jar file for the encryption and decryption functionality. This class will provide the ASCON or AES-GCM encryption and decryption operations needed for secure communication and data handling. 
    \item[-] Develop a custom message mapper class that will handle the conversion of incoming device messages to the appropriate Ditto protocol format. This class will integrate with the aforementioned encryption and decryption functionality to ensure data integrity and security during the mapping process.
    \item[-]Configure the Ditto connectivity microservice to recognize and load our custom message mapper. This configuration step ensures that incoming messages are routed to our custom mapper for processing, enabling seamless integration of our specific data transformation requirements within the Ditto framework.
\end{itemize}







